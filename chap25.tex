\chapter{Conclusion}

The details given so far has been about the last 300 to 50 years old. After
that, because of partition between India and Pakistan, some of the Maheshwari
community migrated from Pakistan to India at different times. Many Maheshwaris
living in Thar moved to Sindh's cities for business and livelihood, however,
they maintained contact with Thar. 

Many changes have happened in these last 50-55 years. New horizons are opening
with everyday science inventions. Electronic equipments, transportation, higher
education, cinema, radio, T.V., computers, internet, etc. have advanced and
there is a clear influence of Western culture over the younger generation. It
is natural that they do not prefer our old culture in today's modern world. 

Urbanization have opened the gates to progress and development but because of
this, our culture, systems, customs, food has been polluted which is not
appropriate. Criticizing simple food and traditional dressing has given way to
new style dressings and food which is not a modern culture but in a way danger
to our culture.

These days, respect for the elderly is diminishing. Too much freedom to
children has resulted in anarchy. Old principles are being forgotten and
loyalty to the self and community is dying. False showoff, pretentiousness,
disapproval and corruption is on the rise. Modesty is on the fall. Unity is not
maintained in the community. 

For this we cannot escape blaming today's young generation. In the last 50
years, we have entered a new world, became wealthy and in the process, have
gradually changed our behavior which has left an impression on the next
generation and has expanded. 

In Thar, Maheshwaris lived in a small towns with limited diversity which
maintained to culture and traditions but in today's big cities and diverse
group, some of the influence of others is clearly seen. 

Change is a demand of time. We must also change but during these changes we
must adopt good practices and leave the rest. We are being pulled towards a
mentality of poor character and losing all the good qualities because of
showiness.

Time is still in our hands. We are responsible to build our character and can
turn back from here. We pray that the Maheshwari community which is a treasure
of qualities and character with its exemplar and illustrative past must stay as
it is for the time coming.

My heartfelt apologies if during these descriptions and discourses, something
inappropriate for the elders or youngsters has been uttered. With this, I
finish this section of the book. 
