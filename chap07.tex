\chapter{Religious Culture}
Religion links soul to the supreme. Religion connects the world with the
creator. Religion shows the way to rendezvous with the God.

There are many religions in this world and people observe their religion. Hindu
religion is the religion of Hindus. It is the oldest religion of all--it is a
\textit{sanatan} (eternal) religion and is also known as a vedic religion.

Thar Maheshwaris are Hindus and followed the Hindu religion. Maheshwari
community originated from the God Mahesh. So, Lord Shiwa was worshipped but
Maheshwaris were Vaishnavs. Every town had a temple. Temple is a leading symbol
of our culture and heritage. Eternal religion is incomplete without temples.
Temples are so integrated in Indian life that it is hard to believe any Indian
has not visited one in his/her lifetime. Many \textit{Murlidhar} temples were
built by Maheshwaris. Beautiful marble idols were brought from Jaipur and were
installed in temples. Shri Ganesh and Hanuman idols would also have a place in
such temples. Towns had \textit{Mataji} temples and \textit{Madhi} where people
would go for worship during \textit{Navratri} fastivities. Other Matajis such
as \textit{Sheetala Mata} idols would also be present in such temples.
\textit{Bhojaks} would maintain and take care of temples and were called
\textit{sevaks}. They would take care of temple's cleanliness, God's bathing,
decoration of the idol, prayers etc. Occasionally, people would donate gold and
silver which was taken care of by the sevaks. Musical instruments such as
drums, hand-symbals etc. would also be found in temples. These were used
occasionally during devotional songs and during festivities such as Holi.

There was not much literacy among Maheshwaris in Thar but there was tremendous
belief in almighty. Sanskrit knowledge was negligible but because of Gujarati
knowledge, people used to read Ramayana, Mahabharat, Bhagvadgita and Bhagvat.
Other religious text were also read. Such Gujarati books were mostly published
by the Akhand Anand Publishers in Ahmedabad city of Gujarat. When people
visited Gujarat for business, they would bring these books with them back home.

Women would visit temple almost everyday. They would donate some grains and
money. They offered 16 \textit{Akshats} (unbroken rice grains) and milk to
\textit{Shivalinga}. After bowing to the god in temple, they would hold a small
mirror put in temple towards god and then towards themselves. From a small bowl
of \textit{tilak}, they would dip a small stick and do a tilak on their
forehead to the tip of their nose. They would circumnavigate the temple. They
would chant devotional songs there. On special festivals such as
\textit{panthevari}, \textit{Holi}, \textit{Janmashtami}, \textit{Shivratri},
etc. related devotional songs would be chanted by community and towns people.

Every 11th and full moon day of the month, woman and some men would observe
fasts. On full moon day meals were taken after seeing and worshipping the moon.
\textit{Ganesh Chaturthi}, \textit{Hanuman Jayanti}, Mondays and Saturdays were
also observed as fasts by some people.

Towns will always have organization of religious stories called \textbf{katha}
in the evening. Usually, elderly women would go to listen to kathas. If there
is a touring playgroup for \textit{ramleela}, men-women would visit without
fail. Occasionally, \textit{Chobas}, \textit{Pandas} and \textit{Gosais} would
visit towns and would receive donations.

Households would organize \textit{Satyanarayan katha} to which neighbors,
relatives and friends were invited to listen. They would listen, and enjoy
sweets and meals in the end.

In addition to the dieties in temples, other forms of god as described in the
Geeta\textit{ji} were also worshipped. Khetwal (kshetrapal), trees and plants
(especially basil, peepal), Holika, Baliraja, Sun, Moon, snake, lamp and other
symbols were also worshipped. Some households had a \text{thaan} (place) for
their ancestors and they would put aside some food for them before eating
meals.

Women threw grains for pegions every morning. Would offer a token-sacrifice to
fire in the afternoon. Separate roti was cooked for cow-dog. Wandering sadhus,
if they come for donation, she would offer them some food. Every Saturday,
\textit{varatiyas} would be offered oil. Ant's mounds were offered flour
and grains. People would donate food and clothes after eclipse. After
eclipse, bathing with same cloth on would be practiced in all seasons.
Potable water would also be changed.

In the Chhachhro town, celibate holyman Devnarayan and holyman Ladhuram's
lectures had significant influence on people's religious inclinations. Aryan
Society also influenced people's cultural and religious influences, interest
and practices in Thar.

Travel was difficult in those days. After travelling over camel for two days
people would reach nearest railway station. Despite these difficulties,
Maheshwaris would go pilgrimage to Pushkar, Mathura, ShrinathDwara, Haridvar,
Jagannathpuri and Narayan Sarovar. Specially, Pushkar and Haridvar where they
would do ossification of their loved ones with the assistance of holymen. Such
pilgrims, when they returned from their pilgrimage would be visited by the
people from town to wish them on a successful pilgrimage. They would offer them
coconut and rupee as gift and the pilgrims would offer sacred sweets (usually
brought from the pilgrimage) and some utensil in return.

With such a tremendous faith, religious sentiments were maintained over years
by the community which was an inseparable part of Maheshwaris life.

\section{Devotional Songs~(\textit{Bhajans})}
No devotional songs in Thari/Dhatki language were found. Perhaps no devotional
songs were made in Dhatki language or were made but did not spread and last
long.

Ancient devotional songs such as by Surdas, Narsingh Mehta, Mira, Tulsidas,
Ramayan's \textit{chopai} etc. were sung. Morning songs, prayers were also
sung.

