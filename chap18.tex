\chapter{Animals and Insects}
Cow, buffalo, goat, sheep, camel, horse etc. were the main domesticated animals
in Thar. There were stray cats, dogs and donkeys. Every Maheshwari family had
at least one milk yielding animal. Some families had 2-3 cows whereas higher
status families had cows and buffaloes yielding milk. Maheshwaris did not own
goats.

There wasn't much of forest but still there were rabbits, foxes, bears, hyena
etc. There were no large cats.

A very few people owned camels and horses for transportation. Camels were
available on rent as and when required. Pushkarna Brahmins would rent out
camels. Additionally people from Meghwal, Bheel and Koli communities also rent
out camels.

Thar towns had more cows than buffaloes. When cows are in milk yielding phase,
household's women would milk them or else in some cases, they would hire a
cowboy \textbf{dhanar} for milking their cows. They would tie a special rope in
hind legs before milking. First they would let a calf drink milk to let cow
happy and yield more milk. The udders were washed and then milked in a metal
pot. After milking two teets were left for the calf and when the calf becomes
one month old, one teet was left. Female calf was fed more milk compared to
male calf.

Cows and buffaloes were fed wet beans. Also pearl millet hay was given.
Buffaloes were fed cotton plant remains. Pieces of straw were also fed. Water
was given daily in summer and every other day or third day otherwise. The
cattles were taken to the well where there was a place for them to drink water.

Cowbells were tied in the neck so that they do not destroy fields. If a cow is
wandering a field the cowbell would ring and alert the owner of the field or
the cowboy. If the owner of the field catches a wandering cow, he would tie it
up and the cow owner would have to pay a government fine to get it back.

Cows and buffaloes were herded in a place by the cowboy for feeding in the
wild. In summers and winters cows would go alone to feed. In monsoon the cowboy
would also go along. When the herd returns back in the evening, the owners
would call them by name. The cows would return the call with their moo sound
and would run towards the owner.

Cows were milked twice a day in Monsoon whereas just once in summers and
winters. Cows stayed in a shed overnight. In monsoon, to help ward off germs
and flies, smoke was made by burning dried grass or tree branches. Flies were
ward off with cloth. Cows also moved their head and tail to ward off any germs,
bugs and flies.

Oxes were used in agriculture. Maheshwaris would take oxes to Saurashtra and
Kutch region of Gujarat to sell them. Every town had 1-2 or more stray oxes
(they were called \textbf{allahiya dhaga} meaning nobody's or god's oxes). Such
oxes were branded with a hot metal brand attached with a rod for
identification. Every community and town had cattles branded with a distinct
pattern so that they would be identifies quickly and easily. For this purpose,
special types of cuts were also made in cattles ears.

There was not much theft of cattles. Specialized people would find the
footmarks and would bring the lost cattle back. Livestock's diseases were cured
with traditional methods by the owner themselves. There were no veterinary
hospitals in Thar. Thar's camels were famous.

A leathersmith would take away leather from dead animals and would use it to
make shoes, containers for ghee, water bags etc.

\section{Insects}
In addition to animals, there were many small and big insects. In the sandy and
hilly terrain, there were different kinds and species of such insects. There
were many different kinds of snakes such as black headed \textbf{vaasing}
snake, \textbf{lundi}, \textbf{khap}, \textbf{funkan} (\textbf{piyan} or
\textbf{buchhdi} which would blow poisonous air). Two-headed \textbf{bodhi},
\textbf{bala} which lived under the soil. In winters they lived mostly under
the soil and in summer many showed up outside. There was a saying as:

``\textbf{Vasant punchmi ra vaja vagiya, naag vichchu sahe jagiya}"

Legged venomous insects were scorpions, centepedes and \textbf{hunkhun} etc.

\textbf{dimbhu}, bees, mosquitoes, Wasps were some of the flying insects. If
wasp or bee stings they would cause inflammation and swelling. 

Other legged animals were Bengal monitor (useful for thieves), monitor,
mongoose, cicadas, lizards, spiders, frogs, locusts, grasshoppers, bedbugs,
ants (black and red), termite, maggots (found in stale grains), moth, jatrophas
(red insects found in rainy season), chancala etc were also found. These were
non-venomous.

Insects seen in monsoon had a very short lifespan. Winged insects seen around
lantern would die and a heap of wings were seen in the morning. 

Holymen would chant mantras and sprinkle water in order to cure scorpion and
snake bites. Snakes and scorpions would not come near fire. If snake has
bitten, it was called \textbf{kakhjyo} or \textbf{kakh chhibyo ahe}.

Locusts would seen in July and August. A group of locusts were capable of
eating up and destroy all the harvest of a field at a time. If they sit on a
tree, it would go barren. Locust lay 108 eggs in soil which were called
\textbf{teh} and \textbf{fakka} when they grow. When locust flies, the whole
sky would get covered by them. If a plague of locust is moving in a particular
direction, government people would warn the villages in that direction by
telegrams. Government would run big campaigns to destroy locusts. Locusts were
buried in land. Working officer would get a certificate and a prize. Some
communities and muslims would collect locusts in bags and would fry and eat
them.

