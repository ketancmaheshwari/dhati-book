\chapter{Sports}
Just as festivals, celebrations, fairs and plays provide us with entertainment value, sports play an important role in providing pleasure and relaxation. A brief description of games and sports played in Thar are as follows:

\section{Children}

(1) \textbf{Ingni Mingni}: 4-5 children would sit with inverted palms. A monitor would chant as per below and would touch his/her finger on each of the inverted palms on each letter.

``\textbf{ingni-mingni-ganthiya-gora-heeng-bahida-kirsan-kaka-lahe-gopala-chhinu-chhod-nakha-dor-itak-mitak-karnara-pura-chhutak}"

The palm that is touched by the finger at the time of speaking \textbf{chhutak} would open the palm. The chant begins again. If the finger comes on an open palm the child would take the palm behind to hide it. Then the monitor would ask question about whereabouts of the hand. The child would answer and bring the palm to the front and show it.

(2) \textbf{Champeta}: This game was played with five small stones, usually by little girls. The stones are called \textit{panchika}. They would hold the panchika in their palm and would toss it up and catch them. Then she would put one or more panchikas on floor and toss the rest again. She would quickly pick the one or more from the ground while others are still in air. Then she would quickly catch the ones in the air as well. They toss panchika and catch it with inverted palm and toss again and catch it with open palm. In a next round they would toss panchika up and touch some part of body and quickly catch them in palm. Alongside they would chant as follows:

``\textbf{
    ikda/bida/munga/eka/jinthi/hado/hundi/dhakni/doiyo/apura-sapura/saleba-choba/hek wari/biji-wari/teeji-wari/chothi wari/pahelka karira katora/bijka karira katora/teej ka karira katora/choth ka karira katora/pehlak bethi/bijak choki/tijak ubharlo/chothak pirlo/panchak khurkhuriyo khaja/dahik dabadko/khilo khaja/giriyo godo/khadiya/rayak bethi rambhlo/thak bethi thambhlo/isarya visarya/gerni/ekam/bijak/tijak/chothak/panchak
}"

(3) \textbf{Dolls (Gudda-Guddi)}: children would make colorful dolls from the rags from old or unused clothes taken from parents. They used rosary peas (chanothi) for eyes. Dolls were decorated with different household items. Dollhouse was also made and dolls were married. 

(4) \textbf{Skipping}: Girls were fond of skipping rope. A girl alone will hold the rope and skip. Alternatively, two girls woul hold the rope from either side and a third girl would skip. 

(5) \textbf{Jhurdiya}: Mango seeds were drilled and tied with threads such that both ends of the thread could be held in either hands and the seed would rotate in between. Young boys loved to play this.


\section{Boys}

(1) \textbf{Marbles}: 4-6 boys would hold a marble each and make a horizontal line and throw the marble. The one whose marble goes most further would get a first shot. He would aim and try to hit other marbles and so on. Marbles games were played in winters.

(2) \textbf{Top}: 2-3 boys each tie their tops with rope and throw on the ground or catch in the palm or even over the nail of thumb. Alternatively, they throw it over on the ground in a small circle.

(3) \textbf{Hide and Seek}: All will hide and one will try to find them. The first person found would take turn finding others. This game was also called thief-police.

\textbf{Moi-Dandia}: 

\textbf{Kabaddi}:

\textbf{Aatapata}:

\textbf{Seven stones (satolia)}:

\textbf{Bhatabhati}:

\textbf{Wrestling (Mull)}:

\section{Indoor}

\section{Card Games}

\section{School Games}

