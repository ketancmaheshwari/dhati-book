\chapter{Sports}
Just as festivals, celebrations, fairs and plays provide us with entertainment
value, sports play an important role in providing pleasure and relaxation. A
brief description of games and sports played in Thar are as follows:

\section{Children}

(1) \textbf{Ingni Mingni}: 4-5 children would sit with inverted palms. A
monitor would chant as per below and would touch his/her finger on each of the
inverted palms on each letter.

``\textbf{ingni-mingni-ganthiya-gora-heeng-bahida-kirsan-kaka-lahe-gopala-chhinu-chhod-nakha-dor-itak-mitak-karnara-pura-chhutak}"

The palm that is touched by the finger at the time of speaking \textbf{chhutak}
would open the palm. The chant begins again. If the finger comes on an open
palm the child would take the palm behind to hide it. Then the monitor would
ask question about whereabouts of the hand. The child would answer and bring
the palm to the front and show it.

(2) \textbf{Champeta}: This game was played with five small stones, usually by
little girls. The stones are called \textit{panchika}. They would hold the
panchika in their palm and would toss it up and catch them. Then she would put
one or more panchikas on floor and toss the rest again. She would quickly pick
the one or more from the ground while others are still in air. Then she would
quickly catch the ones in the air as well. They toss panchika and catch it with
inverted palm and toss again and catch it with open palm. In a next round they
would toss panchika up and touch some part of body and quickly catch them in
palm. Alongside they would chant as follows:

``\textbf{
    ikda/bida/munga/eka/jinthi/hado/hundi/dhakni/doiyo/apura-sapura/saleba-choba/hek
    wari/biji-wari/teeji-wari/chothi wari/pahelka karira katora/bijka karira
    katora/teej ka karira katora/choth ka karira katora/pehlak bethi/bijak
    choki/tijak ubharlo/chothak pirlo/panchak khurkhuriyo khaja/dahik
    dabadko/khilo khaja/giriyo godo/khadiya/rayak bethi rambhlo/thak bethi
    thambhlo/isarya visarya/gerni/ekam/bijak/tijak/chothak/panchak }"

(3) \textbf{Dolls (Gudda-Guddi)}: children would make colorful dolls from the
rags from old or unused clothes taken from parents. They used rosary peas
(chanothi) for eyes. Dolls were decorated with different household items.
Dollhouse was also made and dolls were married. 

(4) \textbf{Skipping}: Girls were fond of skipping rope. A girl alone will hold
the rope and skip. Alternatively, two girls would hold the rope from either side
and a third girl would skip. 

(5) \textbf{Jhurdiya}: Mango seeds were drilled and tied with threads such that
both ends of the thread could be held in either hands and the seed would rotate
in between. Young boys loved to play this.


\section{Boys}

(1) \textbf{Marbles}: 4-6 boys would hold a marble each and make a horizontal
line and throw the marble. The one whose marble goes most further would get a
first shot. He would aim and try to hit other marbles and so on. Marbles games
were played in winters.

(2) \textbf{Top}: 2-3 boys each tie their tops with rope and throw on the
ground or catch in the palm or even over the nail of thumb. Alternatively, they
throw it over on the ground in a small circle.

(3) \textbf{Hide and Seek}: All will hide and one will try to find them. The
first person found would take turn finding others. This game was also called
thief-police.

\textbf{Moi-Dandia}: One player would dig a small slit on ground and put the
\textit{moi} horizontally and toss it by putting a larger stick underneath. 4-5
players standing in front would try to catch it. If it falls down, the first
player would try and hit it by lifting it up with stick. The game will progress
by next person's turn. Then, the moi would be hit by placing it on fingers,
eyes, elbows, or legs. Then a measurement is taken as to how far did the moi
went. In another version, moi would be tossed and caught with the stick. This
was called \textbf{ilo-bilo}. Based on which part of the body the moi is
placed, the shot was named after, eg. \textbf{muth} (on fist), \textbf{chhalo}
(vertical fist), \textbf{kalkatto} (two fingers), \textbf{thunth} {elbow},
\textbf{godo} (knee), \textbf{pug} (leg), \textbf{ankh} (eye).

\textbf{Kabaddi}: Two team stand opposite to each other in a court. One person
from a team would come running in the other team's court while chanting
`kabaddi-kabaddi'. If he comes back after touching any number of persons and
without stopping to breathe those persons are out. If the persons can hold him
and he runs out of breathe then he is out. 

\textbf{Aatapata}: Large boxes are drawn over ground with stick. Two teams
would be made. One person would take turn to be \textbf{vanjhi}. This person
can go to any of the lines and if he gets caught, he is considered out.

\textbf{Seven stones (satolia)}: Seven stones are stacked. Two teams are formed
and each team stands on either side of the stack. One person from a team takes
turn and tried to break the stack with a ball. If the stack is broken the team
belonging to that person have to quickly restack the stones before the ball
could hit them. If the stack is not broken by the throw, someone from the other
team will take turn to break the stack. Meanwhile, if while restacking the
stones, if someone hits the person with ball, the team is lost.

\textbf{Bhatabhati}: Hitting each other with a soft ball. Whoever has the ball
will hit any person around.

\textbf{Wrestling (Mull)}: Two man would remove their shirt, fold the dhoti and
try to fell each other with heels while wrestling. This felling was called
\textbf{malh vidhan}. The fallen person is considered lost. People from faraway
who are expert in the game would visit Thar's villages and towns and would
practice the game. These matches were called ``\textbf{malakhdo}".

\section{Indoor}

\subsection{Sea shell Games} \begin{enumerate}

\item Women would play this during festivities. Big shells were called
\textbf{dayla}. Hot lead was poured in these daylas. Then the shells were held
in the hand and would be thrown on ground. Depending on which shells have fell
up or down, the scoring was done.

\item Shells were thrown in a circle on ground. Then they were carefully taken
out. If a shell touches any other then the person is considered lost. The
person who is able to get maximum shells out wins.

\item Four shells are thrown on ground. Then one down shell is used to hit
another down/up shell. Points are scored per hit. No points if missed. All four
up would considered eight points and all four down would be considered four
points. If three are down then it is a void chance.

\item A small pit is dug in the ground and shells are thrown over it. Those
that fall inside the pit are considered to be won by the thrower. This game was
also played with money.

\end{enumerate}

\subsection{Card Games}
In card games, young children would play stack, chance, etc. Adults would play
games such as coat, 2-3-5, choice, etc. People would also play gambling such as
poker and would lose or win money. Women would specially play money with cards
during festivities. 

Free people would play in the summer afternoons or winter evenings/nights.

\subsection{Other Games}

\textbf{Lion-Goat}: This game was played by young boys. There were seven pieces
considered as goats and two pieces considered lions. They were arranged in a
triangular shaped board with many points along the lines of the triangle. If
the lions could skip over goats, the goat is considered died. If the lion do
not have a place to skip or the goat is not in any of the surrounding points to
lion then lion is considered lost. If lion can skip all goats, then lion is
considered won. 

\textbf{Nine Corners}: Two squares were drawn inside a larger square and some
lines were drawn. Each person would have nine pieces to start with.  Each
person would take a turn and place their pieces one by one. If a persons pieces
fall in a straight line the other person will give him his one piece. The game
will continue until one person has exhausted with pieces.

\textbf{Chomal}: This game had 25 squares. 4 people would sit on each of the
two sides. Each person would play with 4 pieces each. Each person would throw 4
shells to get a score and accordingly move their pieces in the game. If someone
gets over other's piece, the piece is considered lost, and has to return to the
starting point. The game progresses until all pieces reach the center of the
board.

\textbf{Chopat}: The board was drawn by making boxes over a cloth. Four people
would play with 16 tokens. The moves are obtained by throwing two special dices
made up of ivory. The dice had 1,2,5, and 6 points painted over it. The token
would move as many points obtained by the dices. In the end, the tokens would
go on to the central box. This game would also be played with shells in place
of dices.

\section{School Games}

Students would play cricket, football, volleyball in high schools. Teachers
and government employees would play tennis. Matches between schools were
organized.

In Mithi, in 1964, after the arrival of R.S.S, chess and carom board games were
also started. These games were played by the R.S.S. volunteers. They would also
teach stick/baton games.

