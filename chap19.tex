\chapter{Disease and Cure}
In today's world cities and villages have polluted air and water, crowds and
impure food, noise pollution which was unthinkable in the old days in Thar.
With open villages and towns, clean air and water, food grown in own fields
without any synthetic manure or fertilizers, food made of flour made in home
mills, cow and buffalo's pure milk and milk products such as butter, curd,
buttermilk, simple lifestyle and regular festivities kept busy men and women of
Thar very healthy with minimum necessities.

Only the district headquarters had hospitals. There were rarely any hospitals
in the small towns or villages. People would treat minor illnesses themselves
with the home remedies. Home remedies included oil message and oral medicines
to eat dry or drinkable solutions. Household solutions would be used in most
cases and no expensive medicines were used. Only in the cases of major diseases
such as tuberculosis (T.B.) and typhoid were dreaded. Childbirth would happen
at homes. Caesarian section procedure was unheard of. However, in some cases
where the woman was too young, deaths related to childbirth occurred.

Influenza was seen in Thar in the year 1918 and it was spread all over. There
were not many hospitals in Thar and people seldom took benefits of a hospital.
Foreign medicines were dreaded. There were not enough savings to go to cities
for treatment. This influenza would kill 5-7 people everyday in towns. The
atmosphere was of dread throughout Thar. Thari people would call this disease
``\textbf{loos}" and because of the year it happened, it was remembered as
``\textbf{panchoter ri loos}" (flu of the 75 (lunar calendar)) . 

Now let us see how minor diseases were treated in those days:

\textbf{Constipation}: Senna powder and dried black resin were soaked in water
overnight, which was warmed and an extract was given.

\textbf{perspiration/dizziness}: Salt water would induce vomit. Mint and fennel
seeds were given. \textbf{Gomaandar} powder was given.

\textbf{Cough}: Half cooked pearl millet flour roti was applied to chest and
back. Salt, carom, or pearl millet was heated in a cloth bag and applied to
chest or back.

\textbf{Cold and Fever}: Warm oil massage on head and forehead. Crown flower
(\textit{akdo}) leaves were strapped to head overnight. Hot jaggery soup was
given in the morning.

\textbf{Flu}: Patient was sat, a whole cloth was wrapped over and water was
sprinkled over a hot brick so that vapours would be produced. Patient would
inhale these vapours. This process was called \textbf{bafaro}.

\textbf{Periodic Fever}: Maheshwaris would invite 4-7 people from similar
ancestry as self and would discuss the fever.

\textbf{Pneumonia}: Hot ash from the stove was applied on chest. Soaked rice
paste was wrapped/strapped along the chest.

\textbf{\textit{Sidkar}}: If someone comes from outside with heat and sun and
immediately drinks cold water, then sidkar would happen. This is why one should
drink water after a short while. Sidkar would cause headache and uneasiness. A
massage from an experienced person will also help. An experienced person would
put warm ash in patients hands between index finger and thumb, over temple,
over chest and earlobes, over knees and shoulders and would say
``\textbf{sidkar bhaje}". This would help. This was similar to accupressure as
practiced today.

\textbf{Black Cough}: A copper coin was tied along the neck. Milk obtained from
donkey was given. A square of hair from head were removed and a cloth with
medicine was applied. This was called \textbf{chatti}.

\textbf{Eye Pain}: \textbf{raswal} was applied. If the pain is severe,
\textbf{chimed} seeds were crushed and fine yellow powder was applied to eyes
which was called \textbf{bharan} nakhyo. Neem leaves were strapped over eyes.
Soaked sand plates were made which were put on earthen pots for cooling. They
were then applied over eyes so that the heat is absorbed. 

\textbf{Red Eye}: Water was taken in a plate and then a cloth was dipped in oil
and was burned. Patient would watch it. Oil drops would fall in the plate and
the person who is helping would say, ``\textbf{chop bhaje, chop bhaje}". 

\textbf{Leg Sprain}: Thick roti of wheat floor was cooked on one side and oil
and turmeric was applied on the other side. This was strapped in place where
the pain was maximum. This was called ``\textbf{dagad badho}".

\textbf{Pain}: Massage with sesame seed oil. Warmed castor leaves were tied.

\textbf{Boils}: Patient would bathe in hot water with neem leaves. Buttermilk
was added to a copper vessel and would be rubbed with a copper coin. The
produced rusty paste was applied to the boil area. If the boil is big and
without a mouth then berry leaves were crushed and applied. Coal and dried
bedellium (gugal) would be rubbed over the boil so that it bursts releasing pus.

\textbf{Throat Pain}: On eating sweet and sour food together sometimes pain
occurs in the sides of the neck with small marble like growth. This was treated
with massage, hot packs and hot saline water gargles.

\textbf{Throat Discomfort}: Throat discomfort because of cold weather. Can get
as worse as making it difficult to swallow water. Hot salted water gargles
helped.

\textbf{Lower Back Pain}: Pain in lower back because of heavy lifting. Massage
and leg massage by a boy born with breech birth was done.

\textbf{Hepatitis/Yellow Fever}: Mantra treated water was given and mantras
were chanted around the patient. Dried chickpeas and buttermilk was given.

\textbf{Skin Disease}: Hot ash was rubbed over body and thick blanket was
wrapped.

\textbf{Nosebleed}: Cold water was sprinkled over head and rested. Cotton plugs
used in nose.

\textbf{Wound}: Turmeric was used. If wound is small urine was used.

\textbf{Ear Infection}: Child's urine was used. Fresh neem leaves and oil was
poured in the infected ear.

\textbf{Mumps}: Spices and flour was mixed and a paste was made which was then
applied over cheeks.

\textbf{Legs Pain/Arthritis}: Legs would be inserted in hot sand.

\textbf{Asthama}: Opium cigarettes were smoked.

\textbf{Muscle Sprain/Nerve Pinch}: Pain around spine in upper back was treated
by pressing with thumb and applying oil massage. 

\textbf{Spine disorder in children}: Because of accidental handling children
may get this disorder where they would vomit and the growth will be stunted.
This was called ``\textbf{satt pai}". This was treated by massaging over the
spine.

\textbf{Seizures}: Patient would be laid down on their stomach and the back
would be gently kicked. Seizure would be tied. Oil applied over stomach. 

\textbf{Panic}: Sudden panic was treated by pouring cold water over eating dog
to transfer panic to dog.

\textbf{\textit{najar}}: Salt was moved around and over the head of the person
who is believed to be a victim of najar. This salt was then put on a three-way
junction or into a stove. The person doing this would not speak anything. This
was called ``\textbf{najar utari}". Sometimes some people were considered with
a bad eye/najar. Then, a little bit of sand was taken from the left heel of
their footstep and was taken over the najar victim. 

\textbf{Tummy Ache}: If tummy ache occurs every time someone eats then special
person would do a massage and mantras were chanted.

\textbf{Pox}: During a pox (small or chicken) the room where a patient is
resting would be tied with neem leaves at the door. After the pox infection
becomes mild, on the next Tuesday, the person would be taken to the temple of
Goddess Sheetala and worship was done. The patient would ride over donkey back
to temple. The patient was given previous day's stale food. This disease
happened once in one's lifetime. 

In addition to these, \textbf{chor tup} (a kind of fever which occurred during
night only), \textbf{aanjni}, \textbf{Chhapako} (blemishes and spots over face
and body), tonsils, teething of children, \textbf{kachchoradi}, \textbf{rain
jhhai}, \textbf{miriyo}, \textbf{petejari}, poked eye, etc diseases were also
common and they had household treatments. There were no hospitals, diseases
were present but people were generally content.

When patients went to hospital and get medicines in bottles, they would tell
compounder to add some salt in the medicine like so: ``Uttama (name of
compounder), jara dawa mi salat (salt) nankhe". The compounder would lovingly
say ``yes" weather he really adds or not. This loving behavior often caused
half the disease to go away.

If the disease is serious or major then the patient would be taken to
Mirpur-Khas or Jodhpur hospitals. They would not go to Hyderabad or Karachi.

Scorpion or snakebite will be treated by holymen with mantras. The
\textit{funkan} animal would blow venomous air in victim's mouth and would hit
the victim with its tail. This kind of victims were given piece of alum to chew
on. 

People would assume holy restrictions in order to get rid of diseases. They
would wish for some token offering at the temple if the disease goes away. Once
the disease is gone, they would donate money and food. They would offer food to
cow, dogs and crows. They would offer grains to birds and invite married
daughters for meals. Holy men were given clothes, grains etc.

