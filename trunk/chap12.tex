\chapter{Groom's Marriage Procession~(\textit{varghoda})}
\chaptermark{varghoda}
It is a custom to have a marriage procession for groom during wedding which also gives an estimate of the family's prestige. However, the modern day marriage processions were originally called ``ghodi chadhan" (horse climbing).

After migrating from Marwar, wherever they stayed, Maheshwaris organized their children's weddings but they were much simpler because of a painful migration.

After some time, about after a couple of generations in Thar, Maheshwaris settled down and remembered the old style marriage procession. With some savings and established business, they started the custom again.

Groom's marriage procession was one such custom which was resumed by one Chhatmalji from the Rathi family. In order to start a marriage procession, one must get permission from the panchayat of the town. Then, hire some camels, tie drums on each side of these camels and with musical sounds, take the procession from groom's town to bride's town. Offer parties and gifts to the community families on the towns in the way and offer 30 camels to the holymen or an equivalent amount of money.

This custom was understood by Chhatmalji after inviting knowledgeable people from Marwar and organized first marriage procession in Dhat, of his son Dhanji. From this anecdote, a new proverb started as follows:
\begin{quote}
Chhiti kari Chhatmal, varghode ri vaat,
Rathi thari jaat, jjayo na jaapse
\end{quote}

Narsingdas belongs to this Chhatmalji's family and their ancestry was called Narsingra. Chhatmalji's great grandson was Ramji Shah and his ancestry was called Ramjira.

A second marriage procession was organized by 
