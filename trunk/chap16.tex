\chapter{Fairs}
We humans seek time out of our routine to activities that will give us pleasure
and help us forget our sorrows and tiredness. In addition to daily such
activities, society has organized annual festivities and fairs that we can
enjoy in our liesurely time. Fairs are organized on religious and social
festivals and reflect society's culture and traditions. One forgets themselves
and gets enjoyment for a while.

Many fairs were organized in Thar and people--men, women and children would
enjoy irrespective of caste-community or creed. Some such fairs are described
below which were particularly organized around the towns where Maheshwaris
lived:

(1) On the \textit{otta} (outside of house) of Hardas Bhagat in the eastern
Mithi a fair is organized over the dune. This fair is organized on
\textbf{Bhadarva, sud} second day and participated by many people from towns
all around. Earlier this was a one-day fair which became a 3-day fair after the
partition.

A Pushkarna Brahmin named Hardas lived in Chhachhro. In the year 1781, one
night, he saw in dream that some kind of catastrophe is going to befall upon
the town of Chhachhro. There will be riots and bloodbath. Next day he told
about his dream to people of the town and mentioned that he is planning to
leave the town. Anyone is free to join him in this exodus. With Hardas Bhagat,
Maheshwari community's leader Mukhi Malji Mansukhani (Rathi)'s ancestor moved
to Mithi with family. Soon the predicted catastrophe happened. Madadkhan Pathan
attacked and badly wrecked Chhachhro.

Hardas Bhagat was a saint. A fifth generation of his brother Asandas, Roopchand
lives in Khithal in Alwar district of Rajasthan. Descendents of Rupchand's
uncle Farasram, Harishkumar and brothers live in Palanpur and Kadi in Gujarat.
As per tradition, the fair is organized in Palanpur and Mithi.

Malji Mansukhani's descendents are living in Mithi, Palanpur, Vadgam, Bhuj and
Ahmedabad.

(2) Six miles south of Chelhar near the way to Mithi a fair is organized in the
town of Harehar in honor of goddess Malhan Mata. The fair is organized on the
7th day of \textit{Magh sud}. As per the anecdote:

Bundi's Hada lineage king got married to a Rajput Malhan woman. Afterwards, a
holyman informed them that their ancestry belongs to the eighth century
Chauhans which is also the origin of the Hada lineage. Thus, both husband and
wife belong to the same ancestry. King thus renounced the queen and decided to
repent the mistake. The lady became a sati and Rajputs worship her as a goddess
till date. Everyone from the nearby village come to attend this fair.

(3) Four miles from Kantyo town on the way to Chhachhro and Umarkot on the
second day of \textit{Bhadarva sud} Chandopeer's fair is organized. A platform
is built their and people have immense faith in the peer. In addition to nearby
towns, Maheshwaris also visit from Chelhar and Chhachhro.

(4) 
