\chapter{Infrastructural Necessities of the Community}
\section{Dhatki (Thari) Language}
Many languages are spoken in India. Every region has a different language or
should we say regions are made languagewise. Thing every person who uses to
express his feelings is dialect. Inter-human relationships are different at
different places and depends upon geography, business and community. And so is
the dialect. Still dialect maintains the characteristics of its place of origin.
How-ever one tries to hide but in the time of trouble one would send a call of
distress in his own dialect.

It is said that every 12 miles the language changes. So the language at one end
of a region might be considerably different than that of the other end, and
sometimes it becomes even difficult to understand. Based on such languages, it
is decided what part the speaker comes from. For example: In Gujarat, people
from Kutchtch, Saurashtra, Mahesana, Surat etc. have distinct and identifiable
accent and style of speaking.

Formal language means a language for general purposes, administration, education
and social interaction. In that way, dialect is specific to a particular region
but a language spans the whole country. Indian constitution has officiated
several languages. After this introduction, let us see about the Thari/Dhatki
language.

Thar Desert (The Great Indian Desert) is considered to be spread across South
edge of Punjab to the west of Rajasthan to the Khairpur district till the south
of TharParkar District upto the Great Rann of Kutchtch. Maheshwaris migrated
from that region to the TharParkar region of Sindh and the dialect they spoke
was so called Thari from the Thar Desert. People settled in the ``Dhat'' region
called their dialect ``Dhatki''. As per the Encyclopedia Brittanica, vol. XVI,
page 781:
\begin{quote}
 DHATKI, a dialect of Rajasthani is spoken in south-eastern TharParkar District.
\end{quote}
As per the 1931 census of India (Bombay Presidency):
\begin{quote}
Thari/Dhatki is regarded linguistically as a dialect of Sindhi but enumerated as
a separate language in census. For this procedure, there is a clear authority as
THARI is recognised in Sindh as a distinct from Sindhi and has an area of its
own.
\end{quote}
George Gearson authored linguistic survey of India indicates that:
\begin{quote}
The language of TharParkar and Jaiselmer is mostly standard Marwadi. It has a
mixture of Sindhi and Gujarati to a little extent only.
\end{quote}
According to Shri Bherumal Maherchand Advani Authored ``\textit{Sindhi Boli ji
Tarikh}'', ``A new kind of language has been formed by a combination of Sindhi,
Marwadi, and Gujarati. It is called Dhatki means language considered to be
spoken in Dhat. This mixed dialect is considered an alternate to Rajasthani but
is very close to Gujarati.''

According to what is indicated in the Gazetteer of Bombay Presidency, CUTCH,
Feb, 1880, Chapter III, Population: Traers, page 50 \& 51, ``Maheshwaris arrived
in Kutchtch approximately 500 years ago via Nagor--Thar and settled in the
Abdasa Talluka. They spoke Thar-Gujarati language, used to put on turban like
the Baniyas of Thar ... etc''. (Note: In the above writing, the mention of Thar
is used in the sense of \textit{Greater-Thar} means the Great Indian Desert area
and not the `Thar' TharParkar area where Maheshwaris settled late. In around
A.D. 1300, Maheshwaris spoke ``Thar-Gujarati'' language which is likely to be a
mixture of Thar's Marwari and Kutchtch's Gujarati.)

The dialect Maheshwaris brought from Marwar and the one spoken in dhat went
under the influence of the dialects spoken in the surrounding region resulting
in many gradual changes. These surrounding languages includes Gujarati in the
east, Kutchtchi in the South Sindhi in the west and Rajasthani (Marwadi) in the
North. Like this, variations of basic dialect resulted in Dhatki language.

For some years Thar was under the administration of Kutchtch state's ``political
agent'', so the official language was Gujarati. This was also an influential
factor on the Thari dialect.

In the school's of Thar, initially Gujarati, then Gujarati and Sindhi and later
on in approximately A.D. 1940, only Sindhi was taught. Apart from that the
Baniyas of Thar used to write ``Modi'' Gujarati (basic Gujarati characters
without additional accents) in their books. This Gujarati was called ``Vaniki''
gujarati.

When Maheshwaris migrated from Marwar they came from Jaiselmer to Umarkot via
Ratokot. After that they started living in Thar according to their convenience
and started speaking Dhatki. But those who came from a different route from
Jaiselmer via Sakhkhar to Sindh region, then Sahevan, Tando Allahyaar, Tando
Adam, Badin etc. places or came after some time had influence of Sindhi language
on their dialect. Maheshwaris living in Tando Allahyaar and Tando Adam were
called ``Tandai'' and their dialect has clear influence of Sindhi. Table
\ref{tbl:difftharitandaiguj} throws some light on this fact:
\begin{table}
\begin{center}
% use packages: array
\begin{tabular}{lll}
\hline
\textbf{Thari Dialect} & \textbf{Tandai Dialect} & \textbf{Gujarati} \\
\hline
Kahaan dyo & Chavan dyo & Kaheva dyo \\ 
leela gabhbhaa & aala kapda & bheena lugda \\ 
mi sambhalyo & mu budho & me sambhalyu\\
\hline
\end{tabular}
\end{center}
\caption{Differences between Thari, Tandai and Gujarati}
\label{tbl:difftharitandaiguj}
\end{table}
In the same way the dialect spoken in one end of Thar is different than that of
the other end. This is shown in the table \ref{tbl:diffmithigad}.
\begin{table}
\begin{center}
% use packages: array
\begin{tabular}{lll}
\hline
\textbf{Dhatki in Mithi} & \textbf{Dhatki in Gadhado} \\
\hline
Paase mahin betho ahe & godhina betho ahe \\
puthyan aaye to & larinan aaye to \\
Dheba & Dhora\\
Tadha & Weri\\
\hline
\end{tabular}
\end{center}
\caption{Differences between Dhatki dialect as spoken in Mithi and Gadhado Villages}
\label{tbl:diffmithigad}
\end{table}
In the towns of Thar, Dhatki language was spoken by Maheshwaris, Brahmins,
Bhojak, Shrimalis, Khatris, Malis, Sonaras, Rajputs (Sodha), Meghwal, Bheels,
Bajeer etc.. But Lohanas and Muslims used to speak Sindhi however, they could
comprehend Dhatki. In some villages, Muslims also used to speak Dhatki. Looking
at these details, we can opine that: (1) Thari/Dhatki was basically spoken in
Marwad which was brought by Maheshwaris and other communities during their
migration. (2) Due to the influence of regional languages from all sides, their
is some mixture. (3) Dhat's region that was closer to the other region's have
more influence of their respective dialect. (4) School's language of teaching
influenced the dialect. (5) Because of an increase in service class people,
urban dialect differed from their rural counterparts.

Now let's see the technical and linguistic details of the Thari/Dhatki language:
According to Census of India-1911, Vol. 7, Bombay Presidency, page 168: Distribution of Total Population by Languages:\\
Family\hfill : Indo-European\\
Sub-Family\hfill : Aryan\\
Branch\hfill : Indian\\
Sub-Branch\hfill : Sanskritic\\
Group\hfill : North-Western\\
Language or Dialect \hfill : Thareli (Thari/Dhatki)\\
Total Population in TharParkar District = 3,95,235\\
Population Speaking Thari/Dhatki = 1,16,664\\
Male=64,794, Female=51,870\\
Total=1,16,664 ie. about 30\% of the district.
Now let us compare some Dhatki words with Sindhi and Gujarati (table \ref{tbl:words}).
\begin{table}
\begin{center}
% use packages: array
\begin{tabular}{l|l|l|l|l|l}
\hline
\textbf{Dhatki} & \textbf{Sindhi} & \textbf{Gujarati} & \textbf{Dhatki} & \textbf{Sindhi} & \textbf{Gujarati} \\
\hline
Ankh & Akh & Aankh & gaa & gaun & gaay \\
kann & kan & kaan & meens & meenh & bhains \\
nakk & nak & naak & vachhchhdo & gabho & vachchdo \\ 
dant & dandh & daant & chhoiyo & aadmi & purush \\ 
doodh & kheer & dudh & dosi & mai & stree \\ 
dahi & dahi & dahin & hek & hik & ek \\ 
makhkhan & makhan &  maakhan & bu & ba & be \\ 
gehun & kanak & ghau & tann & te & tran \\ 
mung & mund & mag & char & char & char \\ 
saag & bhaaji & shaak & panch & panj & paanch \\ 
chhah & jhan & chhas & dus & duh & dus \\ 
baap & piu & baap & meh & baarish & varsad \\ 
ma & amaa & maa & kirniyu & chhatti & chhatri \\ 
dikro & putt & dikro & kanglo & lagad & patang\\
\hline
\end{tabular}
\end{center}
\label{tbl:words}
\caption{Some words in Dhatki and their counterparts in Sindhi and Gujarati}
\end{table}

Some examples of sentences are shown in table \ref{tbl:sent}.
\begin{table}
\begin{center}
% use packages: array
\begin{tabular}{l|l|l}
\hline
\textbf{Dhatki} & \textbf{Sindhi} & \textbf{Gujarati} \\
\hline
tahjo naam ki ahe? & thunjo nalo chha aahe? & taru naam shu chhe? \\ 
maanh jo naam Mohan aahe & Mhunjo nalo Mohan aahe. & Maru naam Mohan chhe. \\ 
tu kith jaain to? & tu kithe vanji to? & tu kyan jaay chhe? \\ 
hun jaan mahin jaaun to. & maan jag me vanja tho. & hun jaanma jaun chhu. \\ 
taahje roti khaani ahe? & tokhe maani khappe? & tare jamvu chhe? \\ 
hun dhaapyal ahaan & mukhe dho aahe & hun dharai gayo chhu. \\ 
hek raja hanto. & hikdo raja ho. & ek raja hato. \\ 
ue re bu raane hante & tehnkhe ba raanyu huyu. & tene be rani hati. \\ 
hek rajkumar hanto & hikdo rajkumar ho. & ek rajkumar hato. \\ 
rajkumar vaddo thyo. & rajkumar vaddo thyo. & rajkumar moto thayo. \\ 
ooe ra lagan lya. & hunji shaadi kai, & tena lagna levana.\\
\hline
\end{tabular}
\end{center}
\caption{Some sentences in Dhatki and their counterparts in Sindhi and Gujarati}
\label{tbl:sent}
\end{table}
As seen in tables \ref{tbl:words} and \ref{tbl:sent}, the Dhatki language has
been influenced by Gujarati somewhere and Sindhi elsewhere. Some dhatki words
have been written in short form of Gujarati words. Means removing the `kaano'
accent.

As per Thar's traditions and because of affection with each other, peopl's names
were also shortened. We see some samples as presented in table \ref{tbl:names}.
\begin{table}
\begin{center}
% use packages: array
\begin{tabular}{l|l|l|l}
\hline
\textbf{Man's Full Name}  & \textbf{Shortened Name} & \textbf{Woman's Full Name} & \textbf{Shortened Name} \\
\hline
Ambaram & Ambo & Savitri & Saabi\\
Sukhdev & Sukho & Jashoda & Jassi\\
Maherchand & Mahero & Aasha & Aasi\\
Bhagwandas & Bhagu & Nirmala & Narmi\\
Hiralal & Hiro & Jaywanti & Jeti\\
Jethanand & Jetho & Draupadi & Dhuppi\\
Nandlal & Nandu & Rukshmani & Rukhi\\
\hline
\end{tabular}
\end{center}
\caption{Some Full Names in Dhatki and their Shortened Forms}
\label{tbl:names}
\end{table}
Articles appearing in Sindhi eg. jo, ja, ji and Gujarati eg. no, na, ni are
replaced by marwadi style \textbf{ro}, \textbf{ra}, \textbf{ri}. For example:\\

Sindhi: hi chhatti keh ji aahe?\\
Gujarati: aa chhatri koni chhe?\\
\textbf{Dhatki: e kirniyu ke ro ahe?}\\

Dhatki have male and female gender but no neutral gender. Sindhi's `aahe' is
`ahe' in Dhatki and its `tho' is `to'. Examples shown in table \ref{tbl:sent}.

There is no systematic literature available of Dhatki/Thari dialect. The
language being colloquial, it transferred orally from generation to generation
in the form of traditional songs, wedding songs, \textit{sawayas},
\textit{dhamalas}, \textit{shlokas}, festival songs, puzzles/riddles, proverbs
etc.. These were spoken on occassions but are increasingly getting less spoken.
Recently we heard that in Pakistan's Sindh state, ``The Sindhi Adabi Board''
tried to integrate, maintain and publish a collection of such sparse literature.
In that publication's preface some such samples are provided. \textit{sawayas},
\textit{dhamalas} etc belong to the \textit{``pushtimargiya''} genre and so the
Maheshwaris of Marwad must be belonging to that genre.

To include the Dhatki language in the Indian constitution, A Maheshwari Member
of Parliament put forth a proposal in the Indian Parliament in A.D. 1992-93 but
it was not accepted buy the parliament.

\section{Water}
Water is a primary need for humans, animals and plants. Thar being an arid land
there was no river and it was not possible to bring up any canals. Here
rainwater was the basis of life. Rainwater seeped into the soil was brought up
by digging wells. In the ancient times, there was a river called ``hakdo'' that
used to flow across the Thar which disappeared because of natural causes like
earthquake. Because of this the water table went low in the north-east and up in
the south. In Thar, the depth of a well is measured in terms of ``puras''. The
length from the toe of a man (Purush) till the finger of elongated hand was
considered to be one Puras. This is approximately six feet. The depth of well
has been registered as shown in table \ref{tbl:well} in different places.
\begin{table}
\begin{center}
% use packages: array
\begin{tabular}{l|l}
\hline
\textbf{Area} & \textbf{Well Depth in Puras} \\
\hline
Samroti (Near Diplo) & 5 \\ 
Parkar (NagarParkar Talluk) & 10 to 15 \\ 
Kantho (North of Nagar and South of Chhachhro) & 15 to 35 \\ 
Dhat (area between Mithi, chhachhro and Umarkot & 40 \\ 
chhachhro, islamkot, mithi & 20 \\ 
Bhorillo & 30 \\ 
Kantyo & 20 \\ 
Chelhaar & 35 \\ 
gadhado & 60 \\ 
\hline
\end{tabular}
\end{center}
\caption{Depth of wells in Puras in the Thar Region}
\label{tbl:well}
\end{table}

Wells were known as ``tadha'' or ``tad''. Digging well was considered to be holy
work. If a person funds to dig and build a well then the name of the well and
the place was called after that person. For example, ``Meghe ro Tadho'', ``Dane
ro Tadho'' etc..

Wells used to fetch water with different tastes. Different tastes had their
names eg. \textbf{Kharo}, \textbf{Charko}, \textbf{Baalo}, \textbf{Kasaro},
\textbf{Ugro}, \textbf{Mitho}. If sweet water was not available, people made it
do with the slightly salty water or the water with other tastes.

Wells being in the arid and sandy land in Thar, they were used to be built using
special bricks. Such bricks were known as ``nav-terahi'' bricks. Such bricks
were very useful in the circular built-up of the wells. Sometimes wells were
also built in square shapes. Waters in the wells being deep, they were not used
for agriculture but sometimes, vegetables were grown around the wells.

To fetch water from the wells coloured leather \textbf{``kos''} were used. They
were tied through thick rope or leather to a pulley and pulled by animals like
camels, ox or donkey. The water was filled into place called \textbf{``avada''}.
The end that was put into the well was called \textbf{``saaran''}. It used to be
approximately as long as the depth of the well itself. Two people used to
operate the kos. One the person who orders the camel to pull (called
``khilio-khilivaro'') and the person who held the kos. While the camel pulled
the pulley, the person holding the kos used to shout \textbf{``hau hau pachcha,
mel pachcha''}. On hearing this the khilio used to stop the camel and pulled out
the \textbf{nail} between the rope and kos in order to release the strain on the
kos. This resulted into water being flown into the avada. The
\textit{paaniharis} used to fetch water from here and the unused water used to
go into drain. One such turn was called \textit{``vaaro kaadhyo''} and the
people used to fetch water and operate the kos belonged to the Maali community.

Around the Gadhado town where waters were deep, two camels were employed to
fetch water. When one camel reaches half way the nail was pulled out and the
saran was re-tied to another camel and it used to pull the rest of the way.
Thus, the length of saaran was half of the depth of the well.

Some \textbf{``vaishnu''} (Vaishnav) who have dislike for leather used canvas
bags called \textbf{``chalsi''} for kos and cotton or \textbf{``akolia''}
(cotton-like rope made up from ``aaklo'' plant's pulp) ropes for fetching the
water. They used to do it themselves and the water was called \textbf{``bhrahma
jal''}.

While the kos is worked, the women of village came to fetch water. Women put the
pot of water on their head and used \textbf{``Sindhuni''} to support the pot.
They used to put different types of pots (\textbf{``gaggar-morio''}) on
sindhuni. These sindhuni's were decorated with mirrorwork, beads and beautiful
embroiderries. The behind of the sindhuni had its decorated tail called
\textbf{``chhugo''} or \textbf{``chhedo''}. Parents used to gift sindhuni to
their married daughters as ``dahej'' (dowry). Poor people used earthen pots.
Rich men's ladies did not go to the well to fetch water. They ordered the
\textbf{``pakhal''}. The leather pakhal could carry eight pots of water that
used to be brought by people called ``pakhali''. The water was emptied in the
household pots and the remaining water was put into \textbf{``hodi''} or cement
tanks.

Between Chhachhro and Gadhado, there were shallow wells which were called
\textbf{``veri''} or \textbf{``par''}. The depth of water in such veris depended
upon the rains. Normally, water was found at the depth of 5 to 15 puras. Like
tadha's, such veris and pars were known by the names of people who built it. For
example, jesse-ro-par, khime-ro-par, waghe-ri-veri, kumbhe-ri-veri, etc.. Some
women also had built such tadhas like rupi-ri-veri, maanbaai-ro-tadho. In
Mitthi, girls school teacher coming from Saurashtra were called baaisaaheb. She
also built one tadho and it was called baaisaaheb-ro-tadho. The taste of water
of such veris was similar to \textbf{palar} water.

In some villages, big tanks called \textbf{hod} were built for drinking or water
for livestock. Some big tanks were built to store palar water also.

If some bucket or pot fell into a deep well, it was searched using the
reflection of mirror (called \textbf{mirio}). The lost item was pulled out using
\textbf{``billi''}, a hooked device made of iron and tied to a long rope.
Sometimes the maali himself, used to tie himself to rope and went into the well
to fetch the thing.

In monsoon, water used to get logged into small lakes in villages and was used
for livestock and washing clothes. Such lakes were called \textbf{``tarai''} in
Thar. Many towns had such water and its storage capacity was measured in
terms of how many months the water will remain.

\begin{center}
% use packages: array
\begin{tabular}{l|l|l}
\hline
\textbf{Name of the Town} & \textbf{Name of Tarai} & \textbf{Month Capacity} \\
\hline
Mithi & nandhi, wadi tarai & 3-4 months \\ 
Chelhar & ranasar ri tarai & 5-6 months \\ 
Chelhar & Chhichhi ri tarai & 3-4 months \\ 
gadhado & pandhiyari ri tarai & 6 months \\ 
khiysar & - & 4 months \\ 
chhachhro & 2 mile dur tarai & 3 months \\ 
\hline
\end{tabular}
\end{center}

The depth of water in lakes was measures in terms of \textbf{``gode jitto''}
(upto knees), \textbf{``chel jitto''} (upto waist), \textbf{``kulhe jitto''}
(upto shoulders), \textbf{``mathode taar''} (a man would drown), \textbf{``othi
bod''} (a camel would drown - 2-3 mathoda).

Lakes built by people around Gadhado and Bagal were called \textbf{``Garua''}.
Rainwater was collected in such garuas. Such garuas were also known by the
people who built them. Such as Manakia-ro-garuo, lalania-ro-garuo, vahua-ro-garo
etc.. Garua's water was very sweet. There was a \textbf{``chhipo''} lake in
Chhachhro where boys and men used to go for bath.

Because of problems associated with depth of well and water fetching, girl's
parents hesitated to marry their daughters to such villages. Even in folk songs
daughters tell their parents not to marry them in villages where the wells are
deep.

In Maheshwaris, unmarried girls never used to go to fetch water. When in-laws
sent married woman to fetch water for the first time, they decorated the pot
with white paint (called \textbf{``sehdi''}) and used to make rec colored
swastikas on the pots. Good decorated sindhunis were given and auspicious time
was chosen to send the daughter-in-law to fetch water. This was called
\textbf{``vahuari na pani uthiyari''}.

Wells gave water and hence prosperity to people but at the same time some
unfortunate man or woman fed up of life used to jump into these wells
(\textbf{``tipo deita''}) and commit suicide. Such wells were little used
afterwards.
\section{Food}
It is a matter of pride that having been arrived from Marwar centuries ago and
living amongst various communities in different region with the non-vegetarian
eating habits, Maheshwari community practiced vegetarianism.

Ironsmiths, ``Khatri'', ``Maali'', ``Bajeer'', Goldsmiths, ``Meghwaal'',
``Bheel'', ``Koli'' and Islamic people had always been non-vegtarians but when
Saaraswat brahmins Shuddh started consuming non-vegetarian food, at that time
only brahmins of Pushkar, Maheshwaris, Bhojak and Shreemali brahmins stayed
strict vegetarians.

Jains used to live in Nagarparkar. Except there at none of the places in Thar
had Jains(Oswal) habited. None of the Maheshwaris lived in Nagarparkar. This
prevented the influence of Jainism on Maheshwari community and the brahmins of
Pushkar along with the Maheshwaris considered garlic and onion as
non-consumable. The reason behind this could be that these communities being the
followers of Vishnuism had limitation in the consumption of such Tamasik food.

Securing food by farming on their own fields, consuming milk, curd, butter milk,
ghee from their own cattles(cows and buffaloes) in enough quantities, these
people used to survive on simple but nutritious food. Almost at each
Maheshwari's house, there used to be milk-producing livestock. They used to own
one or more cows. Some Maheshwaris used to keep buffaloes too apart from cows.
People of other communities used to keep goats.

To produce flour of grains, every household possessed flour mill using which
woman used to grind the grains by themselves. While grinding the flour of Pearl
millets (\textbf{bajri}), eating the fresh flour termed as \textbf{Baat} stuck
to the \textbf{pulley} of flour mill used to give an immense pleasure. The women
of the house used to have a good knowledge of all the parts of flour mill such
as the \textbf{pulley}, \textbf{Makdi}, \textbf{kheel}, \textbf{kar},
\textbf{patli} etc.

The coarse, fine or medium texture of flour could be obtained by adjusting the
height of nail. In 1944-45 at Mithi, Bhagchand Lohana installed a flour mill
using diesel as fuel but none of the Maheshwaris used to go to his mill to grind
their grains. In case of increased need of flour, two women used to sit
face-to-face and grind the flour together using the flour mill.

With the help of a small flour mill also called as ``ghantulo", whole green
grams used to be grinded to produce lentils and further grinded and peeled to
produce \textbf{kormo} which was soaked in water and then kneaded in the wheat
flour with spices to prepare special and delicious chapatis commonly known as
\textbf{tikli}.

\textbf{The staple food of Maheshwaris:} For the breakfast, kids used to eat the
roti of bajri prepared a night before with curd. In winter, the bajri rotis were
warmed on coal-stove (angithi) and then crushed into pieces to mix with curd.
Home made butter used to be spread on the roti of bajri which was called as
Makhan-chakki and offered as breakfast to little kids. Adults did not use to eat
breakfast.

Before eating their lunch, women used to offer first part of their meal to the
fire. Separate chapatis for cows and dogs were prepared.(At the dawn,
feed/grains for birds used to be sprinkled at the places meant for it and these
places were called as hola-ro-chowk.

The lunch usually consisted of roti of bajri and thick chapatis of wheat flour,
\textbf{rabdi} (curry of gram flour and curd), cooked vegetables, curd and
buttermilk etc. \textbf{Khichdi} (boiled rice and lentil mix), rabdi and roti of
bajri used to be the dishes for dinner. To prepare khichdi, rice and green
grams(green lentils with peels) were mixed and then cooked. (As rice grew
costlier, sometimes the proportion of rice in khichdi was kept less than that of
green grams). In warm ghee, first cumin seeds and then the blend of buttermilk
and gram flour were added to boil in medium fire to prepare the rabdi. It got
cooked very quickly.

\textbf{``Lentil and rabdi had a fight, lentils consider itself superior, In
less time rabdi gets cooked, never gets less in quantity."}

\section{Clothing}
There is an old proverb in Gujarati ``desh tevo vesh". But, Maheshwaris did not
do any changes to the dressing styles they brought to Thar from Marwar. There
were different dresses for children, women and men. Following is their
description:
\paragraph{Small Children:} \textbf{``Jhablo"} and \textbf{``potro"} (a square piece of cloth without any stitches).
\paragraph{Young Boys:} \textbf{``cholo''} (shirt), \textbf{``suthan''}, patloon
(pyjama with tying thread), \textbf{dhotli}, cap, shorts, \textbf{waistcoat},
coat, in feet, leather slippers made by local cobbler. In winter, sweater,
monkey cap, muffler etc. was worn.
\paragraph{Unmarried Girls:} \textbf{Puthio}, ghaghro and after some age
``odhan''. Later some girls also used to wear Frocks and Patloon. On hands,
\textbf{kafur} (rubber)/aaj upto elbow, \textbf{Bilhia} or bangles ( made of
ivory) which was also called \textbf{Mahiyar}.
\paragraph{Men:} \textbf{Puthio, Dhotio, Potio} (turban), \textbf{Cholo}/shirt
and in winters \textbf{baggalbandi}. Elderly people put on blankets on
shoulders. While studying English, students and employed people used to wear
half sleeve shirts. In case of full-sleeve, they had double-cuff buttons, suits
(without blazer), blazer and occassionally neck-tie and hat were worn. Socks in
legs, and to keep socks inplace, an elastic belt with hooks was tied. Some
employed men also used to wear Dhoti and \textbf{Patko} (a kind of turban).
Dhoti used to be worn with double \textbf{laang} (the end to be tucked behind).
In case of a relative's death the dhotli used to be plated and one laang was
tucked on the front instead of back. During the wedding of a men, the plates
were kept untucked.
For bathing, baafta (a thick cotton) \textbf{Anguchcho} (towel) was used. Some
people used dhoti to dry their body and used to wear the same dhoti. Dhoti was
changed every day and used to be put in to laundry. Cloth-stitched baniyan with
a deep pocket near the belly was used to keep money safe. To keep money, a long
plastic bag with threads on both side to tie were used. Such plastic bags were
called \textbf{Vasni}. These Vasni's were tied along the waist.
\paragraph{Married Women:} Used to wear \textbf{zabbo} and \textbf{kurti}. Used
to wear ``gherdar ghaghras'' (chaniyu or heavy flared skirt). Women used clothes
type like chhint, gujj, cheero, kutchchi utlus, pent, kundhi etc.. To sew flared
skirts, cloth was cut into plates or hook shape and for the tying thread a
``chheen'' used to be made around the waist area. Such skirts were made of upto
200 plates sometimes. Newly wed girls used to wear skirts with a special type of
hand made tying threads made up of embroidered fabric and had two coins tied at
either ends. For covering head, they used a 3 feet cotton cloth which was also
used to cover face (ghoonghat). There were several names for such cloth such as
\textbf{laherio}, \textbf{sadahu}, \textbf{pomcho}, \textbf{divtho} etc. In
winters they used warm marino (a pink shawl). On almost all these clothes, they
used to put artistic clothwork such as \textbf{maakhi}, \textbf{klawat},
\textbf{goto}, \textbf{mukko}, \textbf{surmo}, \textbf{sattaar}, \textbf{tildi},
silk embroiderries, diamonds etc. In the event of a death, ladies used to put a
fold of their odhan behind the head. This practice was called to do ``pachcho
pallo''. In hands, they used to wear kafur (rubber) or ajj (ivory), these
usually covered the whole hands from sholder to elbow and sometimes upto wrist.
On feet they used to wear local jootis or ``sapatas''.
\paragraph{Widows:} Widows used to wear kanchali (a plain gown upto waist) and
sleeves upto wrist. These were called ``lambiye baahe''. Used to have black or
red cloth as scarves. The bangles and other things from the arms were completely
removed. Elderly women used to wear ghaghras made of ``fillingiai'' hand-colored
by the local khatri, ghand etc. Some cloth used to be imported from gadhado that
was also used.

Boys and men's dresses such as cholo, shirts, trouser etc. were made of baafta
cloth which was taken as a big piece and 3-4 pairs were sewn. These pairs used
to be very durable and got better on each wash. These were normally sufficient
for an entire year. They were also economical. Later, a Japanese cloth was also
    used which was called ``kelo''. One pair of shirt-pyjama used to cost 1
    Rupee. After the arrival of ``Hirakh'' type of cloth in the markets, clothes
    became whiter. Malmal and popplin was also used to sew shirts. Later China
    made double Horse Boski clothes also got popular. This cloth was used to sew
    shirt as well as used as turban. For trousers, ``duff'' cloth was preferred
    as it was smooth, durable and whiter.

Government workers used to sew shirts with detachable collors so that the
collors for durability. If the trousers worn from hips, it was a fashion to fix
it with fancy patch of cloth. Trouser's sleeves were doubled. For shirts check
pattern was popular and for pyjamas, belt was used. Men's dhoti (a type of
loincloth) was thick, with a red border and used to be imported from Patan,
which was called ``Pattani Dhoti''. The whole piece was roughly 8 yard so one
dhoti was 4 yard. Later on mill produced thin cloth dhoti also became popular.

For Turbans, clothes from Jodhpur was used. Elderly men wore white turbans.
Fathers of marrying couples used to wear pink turbans. Boys used to sew silky
trouser of a fabric called pent. Handkerchiefs were hardly used. Bushcoats were
not very popular. In the year 1946 when first RSS (Rashtriya Swayamsevak Sangh
or National Volunteer Organization) branch opened in Mithi, as a uniform of
swayamsevak (volunteer), khaki shorts, white shirts and warm black caps were
used while attending the branch.

Taylors were there in the villages but some sewing work was also done by women
at home. Boys shirts were hand sewn by them. Sewing machines were not common at
homes. Coat's and short's buttons were made of embroidered thread or by covering
cloth over two layers of alluminium pieces. In olden times, women used to use a
weaving wheel (``arrat'') to weave cotton to make threads, make cloth and color
it. This kind of cloth was called ``gharecho'' and was used to sew ghaghras.
Such wheels were very large and are still seen in some homes.

Old girls and women started to make tablecloth, covers, handkerchief, scarf
borders etc. using ``aar''(ankodi). Woollen sweaters were also made of ``sua'' a
kind of big, thick and non-sharp needle. These sweaters had different kinds of
embedded designs and patterns. Gloves and caps for kids were also made of wool
at home.

Women used to wash their cloth at home or at the local lakes. For washing hair
or clothes, some places had suitable soil that was used. There were no washermen
in Thar. Only in Mithi there was one family of Chhipa (washerman) but
Maheshwaris hardly gave their clothes for washing or ironing. Some government
employed people used the services of washermen.

Women used to do embroiderry on red ``hulwaan'' cloth. After removing some
threads from the cloth they used silky thread to make designs. Such embroidered
clothes were called \textbf{Bokani} or \textbf{Patko} which was worn in
weddings.
\section{Jewelry-Makeup} Gold and silver jewelry was used to be called
\textbf{``Toll''} by Thari people. Maheshwari women were very much fond of
jewelry. Jewelry popular in Marwar was also worn by people in Thar. Rich
Maheshwari man looked poor by his dressing but even an ordinary Maheshwari woman
looked rich with all the ornaments.

Keeping gold and silver in homes in the form of jewelry was not only the decor
for home but also the prestige. In difficult times, these jewelry was useful so
    elders always kept that in mind while buying these. Gold was valued at 15 to
    20 Rupees per Tola (11.66 grams) and silver at 50 to 60 Rupees per sher (80
    Tolas). But with limited income, the purchasing power was less.

Different jewelry of gold and silver was made by local goldsmiths. Later,
employed people used to go to Mirpurkhas to order their jewelry. Now let us see
the details of commonly used women's and men's jewelry:
\subsection{Golden Jewelry}
\paragraph{Men:} Rings, \textit{chhalla} or \textit{varnos} on fingers, on
wrist, solid, 24 carat, approximately 20 tola \textit{kado} or bracelet. Boys
used to wear \textit{tugalia}, \textit{murki} in ears. Elderly men used to wear
\textit{loong} (earflower) or gold laced \textit{gokhru} in ears. Used to put
shirt's golden button with a chain. Used to tie lockets and pendants in their
neck. On wedding, the groom used to wear heavy (approximately 20 tolas),
necklaces.
\paragraph{Women:} There were no ornaments for small children. Only black beads
woven into golden strings which were called \textit{najaria}. Used to put on
silver anklet in legs and \textit{kadholiya} in hands after they learn to walk.
School girls used to wear \textit{ali-borlo} in hair. Used to wear \textit{buli}
in nose, rings in fingers and earrings in ears. Unwed girls used to wear
\textit{bilhias} (bangles) on hand that were called \textit{mahiyar}. When a
girl weds, she use to weat a lot of jewelry. People used to say: \textbf{``tolaa
uu saththe chchadi ahhe''} More details on women's golden jewelry:
\begin{itemize}
 \item \textbf{Forehead:} Ali-borlo, during wedding, \textit{aad} and \textit{tildi}.
\item \textbf{Nose:} Nose-stud, \textit{siri}, \textit{koko}. Elderly women used
to wear \textit{Bhogli}. At the time of wedding, \textit{nath} or window. Used
to pierce in the middle of the nose and wear \textit{buli}.
\item \textbf{Ears:} Earring, small and big eardrops, \textit{durgala} or
durgala-eardrops with lace. The upper part of ear was used to pierce at three
places and small leaf shaped ornament used to be worn called \textit{pan'di}.
\item \textbf{Teeth:} Some women used to cover one or two teeth with a golden
sheet or only put a stud in a tooth.
\item \textbf{Neck:} ``Dohri'' (with 3,5, or 7 folds), ``kamthlo'' (which was
made of golden gini/coin or by moulding )
\end{itemize}

\section{Residence}
Maheshwari people basically migrated from Marwar and gradually settled in Thar's
towns and villages. Their residences were all scattered and unsteady for a while
after (usually changed the residence twice or thrice) which they settled at a
place. This long period could be considered to be about 50 or more years
(Transitional Migration Period).(\textbf{Note:} Since the partition in AD. 1947
till today the migration is still on.) As and when the family or clan's people
gathered they started making huts and temporary housings with hay and clay etc.
This was followed by acquiring land cheaply in large area. People who migrated
earlier called out on their relatives and when a sizeable number of people
gathered, they made more permanent and structured, strong residential societies.

Thar has a very little rain which leads to the building of special types of
residences. Specially since people were poor and lead a simple life, they
started using the locally available material to build houses.

People started making their own bricks by digging up the clay or buy from the
brick-makers. These bricks were thin, broad and long. Such cooked brick's
samples could still be found from the ruins of ``ghadhi". After that for
convenience, standard sized bricks of size $12" \times 6" \times 3"$ were made
in large quantities. When possible, bricks from old ruins were used to build new
homes. For construction, outside the home a pit was dug and in there clay
\textbf{gaar} was made using which bricks were laid. To plaster the walls,
strong gaar was used. To prepare strong gaar \textbf{drabh-murat} (a type of
grass) mixed with barley's hey, horse pr donkey's feces and salt-clay was wetted
for 7-8 days. After that this wet mixture was crushed and used as a plaster.
This was colored with white clay or with \textbf{sehdi}, used to do
\textbf{pochi}.

Later on when bricks were made by kilns the bricks were cooked in those kilns
and were used by the wealthy. However, some people used strong cooked bricks on
the outside of the house while softer raw bricks were used inside the house for
frugality and also such construction used to stay cool in summers.

Normally outside of the house a compound wall was erected so that some animal or
outsider can not come in and also the privacy of home was maintained. The main
door of house used to be kept broad and high so that domestic animals can come
in. If the house is at a higher level than the land, steps were made outside.
Inside the compounds some home had covered platforms called \textbf{jhelo}. If
there is no jhelo then there was \textbf{velho} or veranda where domestic
animals were tied in a corner. Further down, the main house started where there
was an \textbf{osri or chhajari} followed by the room. Size of rooms were big.
the kitchen was on the right or left corner of osri or sometimes separate
kitchen (\textbf{randhnu}) were made. Large room was called \textbf{dohlo} which
was used as a living room. Erecting walls a place was reserved for bath but
there were no bathrooms. No latrines were constructed in homes. The innermost
smallest room was used as a store room which was used to store valuables and
small stuff. 

As a part of main home, one room was made in a corner whose one door opened
inside the home and the other one opened outside to the street. This room was
called \textbf{ottak}. Elederly of the home used to sit here on a wooden cott.
Guests also used to stay here so that the honor and privacy of the women of the
home were maintained. Outside of the ottak also 2-4 cotts used to be there in
the evenings-nights.

To hang things, colored wooden nails were nailed into the walls. Pigeon holes
were made in the walls called \textbf{jaara} and cupboards with doors were also
made. Inside the cupboard, in a corner, a secret wallet was made in which a
small pot used to be fitted at the time of construction. The wall had a brick
which was not stuck in the wall behind which this secret wallet was made. These
pots were used to store jewelry and such valuables. Cash and other valuables
were sometimes put in clay pots which were covered by cloth and dug into
landpits. These places were marked by some secret signs. There were no banks in
Thar. 
Terraces were kept almost flat. For terrace construction, wooden rafters called
\textbf{sohitar} were used as support to flat wooden battens. These battens were
made up of lightweight wood such as \textbf{kirad} and \textbf{khabad}. On top a
type of grass called \textbf{Sania}, \textbf{murat} or \textbf{khip} were spread
or sometimes carpet made up of palm tree leaves were spread. On top of that dry
clay was spread for 2 to 3 inches followed by a plaster of strong clay. For
plaster, wooden or iron made \textbf{madho} tool was used. Terrace constructed
in such a way used to keep cool in summers. Terrace was given slight slope which
was usually invisible to naked eye. At the end of this slope was a drain to
drain out rain water. These drains were made up of thin iron plates and were
called \textbf{parnal}. Some people also used G.C.I. (Galvanised Corrugated Iron) terraces made up of iron
sheets.
After first world war, that is in about the year 1920, iron beams, girders and
T-sections were available cheap. Such girders with T-sections were used in place
of sohitors. $3" \times 5"$ I-Beam were laid on which an inverse `T' shaped $1"
\times 1.5"$ were used and in the middle squares of bricks called \textbf{choka}
were laid. $12" \times 12" \times 2"$ sized chokas were laid and the seams were
sealed with plaster. On top, sanias were put and \textbf{rago} was made to
construct the terrace.

Main ottak, dohla or near the main room's roof, a square pit was made on which
\textbf{baajigar} was made. It's three sides were covered and the side with the
direction of wind was uncovered. The square part could be covered with a cover.
This enabled light and air-flow in the room. This pit had horizontal rods in the
middle and a cover could be pulled to cover it when desired.

In the basement of the home, a mixer of equal parts of sand and cowdung was
prepared and spread on the floor. This was called \textbf{bhargola}. Members
from the household and children from the neighbor were called over to do step on
it (left-right) for it to settle down. This was called \textbf{otto champta}.
For this a lot of fresh cowdung is required for which people used to reserve the
whole village's cattles for a night so that somebody else can't take any
cowdung. If there is cowdung lying somewhere and there is a line around it in
the sand then nobody would touch it considering it was reserved. After the
stepping process, it was left to dry before using the floor. When there is wear
and tear on such floor because of regular usage, every third or forth day a
plaster of cowdung was done which was called \textbf{dhor dini}. This was done
daily in the kitchen to keep is neat.

On the doortops of less strong houses a thick wooden plank was kept as a lintel.
In the houses made up of bricks, the doortops were made up of brick archs which
looked aesthetic and were convenient. The lower panel of door frame was called
\textbf{tharkhan} while the upper frame was called \textbf{barsakh}. The frame
itself was called \textbf{chokhat}. Outside of the house a semi-hard otto was
made but if it was convenient a hard otto was made by laying bricks horizontally
and vertically and filling seams with plaster. This otto was called
\textbf{thalho}. To store water in home, tanks made up of bricks and cement were
also included later on.

There was a common conception that until the Temple's rooftop is not made, no
Maheshwari can make his home's rooftop. In Mithi village, there was one such
two-storied girl's highschool. One Maheshwari brother constructed such
two-storied house and in a few days his wife died at a very young age. However,
this could also be a coincidence.

In Thar, almost everybody had their own small or big homes. Nobody rented house.
If some one needed or if someone comes to a town for job purpose, he used to get
a house to live for time being. Travelers who came from outside, for them, every
town had a Maheshwari Guest house called \textbf{Dharmshala}. For the building
of such dharmshalas, some Maheshwari might have donated land or it could be a
combined effor of the Maheshwari \textbf{panchayat} or it could have been
completely built by one individual. This Dharmshala was generally known by the
individual who built it. The Dharmshala usually had an accompanying well built
as well. 

As mentioned earlier, women used to bath in the walls erected at the vedaha
while the men either used to bath behind a plank of wood or went to the well.
After the bath, they used to wash their clothes and return with the
bucketfull of water with themselves.

Latrines were rarely built inside the house. Men used to go 2-3 of them together
into the woods with a pot full of water. While returning they cleaned the pot
with sand. After coming home they used to wash the hands and the pot with clean
water. Women went to the outskirts of villages behind the bushes. Later on in
Mithi, government built a compound outside the town protected by barbed bushes
which was called \textbf{kuhi}. Women used to go in a group of 3-4 for latrine.

To assist in construction of houses, Meghwal or Bajir labors were easily
available. They were paid a few annas or a rupee per day for their labor. They
were also given Bajri Roti and buttermilk to eat for lunch.

Before the festival of Diwali, walls of home were whitewashed and painted with
white clay. Both sides of doors were painted pictures using Sehadi.

Schools, Hospitals, government offices, Government Rest Houses, Post-offices,
Courts and prison buildings were all located in the headquarters of the Talukas
and were well built.

\section{Education}
Thar region was very underdeveloped owing to the lack of transportation and
scarcity of water etc. This was the main reason it was lacking in education too.
Till AD 1901, the whole district had 3639 literate people. This ratio was 1000
to 10 from the population point of view.
As Sindh was the part of Mumbai state and since Thar area was under the
administration of Kutchtch regions political agent, Gujarati language was taught
in the schools. Initially, Gujarati was taught in second, third and fourth
grades while Sindhi was taught in first and second grades. However, soon all
Taluka headquarters had primary schools. In 1907, Tharparkar district's main
centre was moved from Umarkot to Mirpurkhas, so Sindhi language teaching was
more emphasized and Gujarati language took a place as second language. After
teaching Gujarati with Sindhi eventually in 1940-41, the Gujarati teaching
stopped completely. For Gujarati exams, a separate education inspector used to
come from Karachi.

Primary education started from kindergarten. Black stone slate and black-white
chalk-pen were used. To improve hand-writing wooden plank panels were used.
Fuller's earth (\textbf{Multani Mati}) was applied to these panels and after
drying it, alphabets were written with ink and \textbf{baru} pens. To prepare
black ink, semi-circular sticks were available which were soaked in water in a
wooden or glass inkpot. Inside these raw cotton threads were also introduced.
Accounting and book-keeping was done using this black ink and the books were
sprinkled with fine sand. Later on as blue ink, bloating paper, pen holder,
fountain pen became available, the usage of black ink stopped.

Primary education books were printed in Mimbai state (Sindh was a part of Mumbai
state). Books were printed in equal quantities for both Sindhi and Gujarati
languages. Lessons and pictures were same. Only poetries were different. Book's
cover had pictures of British queen Victoria-George Vth and queen Mary-George
VIth. Books were put in a cloth bag and hung on shoulder to take to the school.
These bags were called \textbf{bujki}.

Primary education was upto fourth grade. In 1911, in Mithi, A.V.
(Anglo-Vernacular) school started where English was taught from first to fifth
standard. After passing fourth grade Sindhi in primary school, in A.V. school,
first grade English with fifth grade Sindhi was taught. In subsequent two years,
second and third grade English with sixth and seventh grade Sindhi was taught
respectively. After passing fifth grade English, for sixth and seventh (Metric)
grade English, one had to go to Hyderabad (Sindh).

For girls, there was one school from grade one Sindhi till grade seven. In
primary fourth grade and English third grade a competitive exam was held in
which some bright students used to participate. Successful students of this exam
received a monthly scholarship of Rupees five. This scholarship was given for
three years. After third grade English, this scholarship was given for the next
four years.

Vadhwan's (Saurashtra, Gujarat) Mr. Dayashankar Ganeshji Dave was appointed the
first headmaster of A.V. school. He was alos given the authority of second class
megistrate. He stayed in Mithi for several years. The ritual of changing of the sacred thread
\textbf{janoi} on the day of \textbf{Balev} was started by him.

If boys and girls wished, they could appear for the \textbf{kameti-final}
(Vernacular Final Examination) exam. This exam name was changed in 1939 to
``Primary School Leaving Certificate Examination". The center for this was
Umarkot which was changed to Mithi in 1938. After passing this exam one could
get a job of teacher in the primary school.

The population of Mithi in 1901 was 2806. In the primary school, the number of
boys was 143 and girls was 93. Umarkot's population was 4924 in 1901. While the
number of boys and girls in its primary school was 180 and 120 respectively. In
Umarkot, boy's school used to impart technical education. They used to teach
carpenter and ironworks to students. 

The seventh grade English examination ``metric" was considered a very important
examination. This examination was organized by the Mumbai University. The centre
in Sindh was Hyderabad.

Initially, in the primary schools, wearing cap was compulsory and in the English
medium school, wearing blazer was compulsory which was abolished later on.

In those days there was no boys-girls co-education. Girls had a separate
two-storied school. As their headmistress, a lady from Kathiawad (Saurashtra in
Gujarat) was appointed. She was called ``bai saaheb". Initially, parents were
hesitant of sending their girls to schools so government used to pay them four
to eight annas as incentive (one anna is one-sixteenth of a rupee). Girls used
to get free cloth and threads for embroidery work at school. When higher
authorities came for school inspection, heads of the town or village used to
distribute \textbf{patasa} sweets. On the birth anniversary of the Queen or the
King, toffees or sweets were distributed among students.

Corporal punishment was common in schools. This included, pinching, situps,
hitting with wooden rule or thin stick. Some teachers also made students do
little household chores.

Around the year 1931-1932, in order to open a new highschool in Mithi, a Thar
Education Society was established. Main members of the society were Mr.
Murlidhar Parumal Nabisaria, Mr. Bhojraj Laxmandas Sindhi, Mr. Hemraj Mulchand
Bachani, Mr. Dungromal Narayandas Jeswani, Mr. Tikamdas Maljiram Rathi. After
the inspiration and hard work of then district collector Mr. Madhusudan Damodar
Bhatt I.C.S., finally the highschool that was going to be built in Diplo was
moved to Mithi and the work started in 1937. The people of Thar collected and
donated an amount of Rupees 25,000 for the highschool. Each Maheshwari
voluntarily donated one month of his salary.

In Chhachhro, English medium school was available until grade three. A.V. school
till fifth grade started in 1940 and a highschool started in 1945. First
headmaster in the A.V. school was Mr. Mulchand Pitambardas Khatri.

Boys from Chhachhro, Chelhar, Gadhado, etc. used to come to Mithi for studies.
Mithi highschool was under the administration of the district local board. In
the campus of highschool there was a boarding facility for Hindu boys and later
there was a hostel for Muslim boys outside of the campus. There was good
facility for dining and staying. There was a deep well in the campus of the
highschool. This went dry later on. It was called \textbf{heladio}. In front of
the school, there was a big playground. 

In fourth grade English medium, Persian was taught as a second language but
after the year 1940, Sanskrit was arranged as a second language in addition to
Persian. For Sanskrit, a knowledgeable pandit was hired part-time. Students
were allowed to opt for either of the subjects. 

Students used to get one or two paise as their pocket money out of which they
used to buy \textbf{Taluna} (cooked chickpeas lentil), \textbf{bhugada} (grams
without skin) or pippermint. It also happened that student paid 1 paisa and
asked for half the stuff in the recess and  the other half at the end of the
school.  

In those days in around 1944-45, without the availability of ice, a man
belonging to Bhojak community used to sell ice-cream made with the help of
Ammonia. Students used to buy from him and eat on campus.

Only boys used to go to highschool initially, but in the year 1944-45, after the
arrival of a lady teacher, two girl students -- Revti Waghjimal Jagani and
Nirmala Bhojraj Sindhi entered the school. They used to sit in the boy's class
on a separate bench. They got maried in a year or two and had to leave the
school.

To become an English teacher, there was a college in Hyderabad from where after
passing the S.T.C. examination, many became teachers. In order to get B.A.
degree for teachers-only Aligarh University organised an external examination
where many teachers from Thar also graduated.

In highschools, ink was used to write in notebooks. Ball-point pens were not yet
available. Blue inkpot and nib-holder had to be used. In the school every bench
had inkpots, one per student in which the peon used to fillup the ink everyday.
For this, one paise fees per month was charged. The fees for games and sports
was one anna per month. 

Occasionally plays were held in the highschool for which students used to bring
sari's \textbf{odhani} for curtains etc. Some students also brought chairs from
home. School also organized competitions for speech, singing, sports, etc.

In monsoons, students and teachers used to go on picnics-excursions to nearby
lakes. Every student used to contribute 2-4 annas and some sweets were prepared
at one of the student's home. These sweets were taken along to the picnics.

In Thar, before the partition of India, Mithi's Mr. Murlidhar Parumal Nabisaria
was the first person to get matriculation. While Mithi's Mr. Shevaram Vinomal's
daughter Draupadi studied from Mirpurkhas and became the first Maheshwari girl
to get matriculation. After metric, followed the inter which paved the way for
the L.L.B. degree. First such graduate was Mithi's Mr. Amolakhdas Parumal Kela
(Nabisaria). After that Mr. Jethanand Lilaram Lalwani and Mr. Waghjimal Gunesmal
Jagani became advocates. After that Chhachhro town's Mr. Maneklal Khetaram
Sharda became the first B.A. graduate. They started working as an English
teacher and later on joined the government's Revenue department. After matric, a
training could be taken to become post-master. Some Maheshwaris were
post-masters.

A school was opened in Mithi in 1938 wherein Patan's (North Gujarat) Pundit
Mulshankar Shastri used to teach Hindi and Sanskrit. Students attended it
voluntarily. One such school was in Umarkot. To open one such school in
Chhachhro town, Umarkot's Mr. Tikamdas Ramjimal Kakkad made a donation. The
school had pictures of both father and son on its walls. School also imparted
religious teachings of Bhagwat Geeta's Shlokas.

Studies were specially emphasized in the schools. There were no private tutions.
Students only studied from textbooks and school notes. Guides were rare. They
used to be called \textbf{tat}. If some student had one such guide he was
considered duffer. For such poor students, teachers used to teach them
additional hours.

For evening sports and games, there was a special games teacher. If some student
is found to be absent, they used  to ask reasons. If the student is sick, they
used to visit him to their homes. Teachers had great affection for their
students and every student was taken well care of by the teachers. There were no
extra curricular activities in the schools.

Students were very respectful of teachers. If some students are chatting around
in the town and if they see a teacher coming, they would stand up in respect. If
a student sees a teacher coming from the othert end of the street, he would either
stopped or would change his way.

\section{Utensils}
We human beings eat food to sustain life and to prepare food, we need utensils.
Earlier, poor people used clay utensils. Whereas, Maheshwaris, Brahmins etc.
used brass, or copper utensils. Muslims used to use enamelled iron utensils.
A detailed description of the utensils used by people in Thar is as below:
\subsection{Regular Usage}
\begin{enumerate}
\item To cook \textit{khichdi}, \textit{daal}, \textit{rabdi}, \textbf{topio
(tapeli)}, \textbf{sipri}, \textbf{degdi} and for its cover,
\textbf{rikebi(chhibu)} were used.
\item To prepare Wheat or Barley dough from floor, \textbf{paiter} was used
which was made up of wood.
\item To roast roti or \textit{fulka}, iron pan or clay pan (\textit{tavdi})
were used.
\item To assist roasting, a long flat spatula which used to be made of iron, was
used.
\item To stir vegetables, daal or khichdi, big brass spoon called \textbf{kudchhi} or
\textbf{kevi} were used. Small spoons were not used.
\item To serve and eat the cooked food, plates, bowl (\textbf{vatki}), glass and
round mugs (\textbf{loto}) were used. These utensils were made of Brass or
Bronze. Bronze was preferred.
\end{enumerate}
Slanted edged plate was called Hyderabadi plate. To fetch \textbf{Moger}, this
was specially used. Round mugs or \textbf{lotas} were named differently based on
their shapes and sizes. For example, Banarasi, Jaipuri, Moradabadi, Chauda Ghat,
Golmora, Kalasia, Vachchati etc.
People used to carry \textbf{lotas} for defacation in the woods.
In addition to copper, brass and bronze, after the world-war-II, utensils made
up of German Silver were used which looked shiny and neat. Copper and brass
utensils were enamelled in order to avoid cooked food getting `poisoned'.
However, bronze utensils did not need to be enamelled.

The container for ghee (refined butter) was called \textbf{ghilodi} while the
inner spoon, called \textbf{mirio} used to be made up of iron. In order to serve
more ghee, \textbf{tipri} was used which could carry about 65 grams (6 to 6.25
tolas) of ghee. \emph{note:} At home on Bajri roti, more ghee was served while
on chapati, the ghee used to be smeared with the help of a small piece of cloth
called \textbf{thigdi}.

To store spices, a wooden holed \textbf{chaamak} was used. While on travel,
spices were taken in little cloth bags called \textbf{kothadia}.

Knives were used to cut and chop vegetables.

\subsection{For Fetching and Storing Water}
\textbf{gaagar}, \textbf{morio} were used to fetch water. These two together
were called \textbf{bedu-hel} and were made up of brass or copper. These would
be imported from Gujarat's Visnagar town. To store water at home, clay pots, big
pots, \textbf{gagaria} and their covers (\textbf{dhak}) were used. For more
needs or during festivals, copper-brass or iron drums were used.

\subsection{For Milk Production and Storage}
Brass \textbf{chaudo}, \textbf{boghardo} or \textbf{chaudi} were used for
milking. To store milk, curd, buttermilk, fermenting (\textbf{ambedva}) and
churning a big pot, small \textbf{handi} and \textbf{dhakni} (cover) were used.
To feed milk to children, seashells (\textbf{supli}) were used instead of spoons.

\subsection{For Sweets}
To place flour or to make dough or to put sweets \textbf{tambith} (big dish)
made up of copper or brass were used.
To cook flour, to fry or for sugar syrup \textbf{kadhai} (large handled bowl)
made up of brass or iron were used.
To filter big circular filters made up of brass were used.
To crush or grind big mortar and pestle made up of wood, iron or brass.
To pull out the fried food/sweet, iron \textbf{jhaaro} (spoon with long handle
and large circular base) was used.
To distribute sweets among the community, iron beaker with handles on both sides
was used so that two people can hold it from opposite sides.
To store sweets \textbf{daboro} (round box) made up of brass was used.

\subsection{Large Utensils on Wedding/Funeral Occasions}
To prepare \textbf{mogar}, a large iron kadhai in which upto 10 kilogram of flour
could be cooked.
\textbf{kudachch} (large spoons) made up of iron which used to be 5 to 6 feet long.
To cook vegetables, curry, rice, \textbf{khichdi} (porridge-like mix of rice and
lentils) a \textbf{deg} which was made up of copper was used.

\subsection{Clay Utensils}
Some clay utensils are mentioned in the above sections. In addition, to store
flour \textbf{tasli} and to store salt or such powdered things \textbf{kulhadia}
were used.
Previously vegetables or khichdi was cooked in clay pots which used to be
tastier.
To crush \textbf{thandai} the pestle was was wooden and the
\textbf{kundo}(special clay made circular deep vessel) were used.
To grind homemade medicines stone made pestle and flat sheet were used.
In addition to above, to store grains the following utensils were used:
To store foodgrains \textbf{kothla} which were cubic or cylindrical large
vessels made up of clay. The foodgrains were filled up from the top and were
fetched from the bottomhole. This bottomhole was sealed with a cotton cloth
which was called \textbf{suhanu}.
To store less quantity of foodgrain, flour or raw vegetables \textbf{dhabra}
were used which were made up of clay mixed with paper.
To keep stuff or to put clothes for washing \textbf{taans} were used which were
made up of iron.
Buckets were iron or brass.
To grind foodgrain, the mill was stone made.
To crush rice into small pieces \textbf{jandar} made up of cooked clay was used.
\subsection{Other Necessary Utensils}
To bring burning coals \textbf{taanda} from neighbors to fire the stove, a clay or iron made
\textbf{dhupio} was used.
To hold taanda \textbf{chimto} (tongs) made up of iron were used. Such nice
tongs were used to hold or turn sweets like \textbf{jalebi} or \textbf{dohti} (a type of
donut).
To fetch \textbf{chhaas}(buttermilk) out of handi, coconut shell called
\textbf{topsi} were used. Dried coconut was used to make topsis. After removing
the skin of dried coconut, a line of water was made in the middle and then it
was broken so that it breaks into two halves. The half with eyes was thrown away
and the other half was mended and cleaned to make topsi.
To wash handis, \textbf{jhunthi} (a kind of grass) was used.
To pick hot vessels, cloth piece were used which were called \textbf{garno}.
There were no pliers. Wealthy people used silver utensils. When Maheshwaris
invited brahmins or other community people over for lunch then they used silver
vessels as they were considered unpollutable. If by mistake a brass vessel was
used then that vessel was depolluted by using taandas.
Silicon or chinese clay vessels glass utensils were never used by Maheshwaris.
Alluminium vessels came late but were not adopted by Maheshwaris. For domestic
usage, different types and sizes of utensils were stored at home and were used
depending on the occasion. Big community vessels were kept at a common place and
were brought as and when necessary.
When Maheshwaris went for business of pilgrimage to Sindh or Gujarat, they used
to buy and bring the utensils from there. They used to engrave elders name on
the utensils. When great grandparent's or grandparent's utenils were used, they
used to read their names and felt proud and happy.

The stoves \textbf{chula} were made up of clay. In winters portable stove of
clay were made which was called \textbf{angethi}. For fuel in those stoves, dry
wooden logs or dried cowdung was used. There were no coal stove, kerosene
primus, electric of gas burners or ovens.

In addition, utensils for religious activities were made up of copper or brass
such as \textbf{panch patra}, \textbf{aachmani}, lamp, and so on.

\section{Bedding}
Because of Thar's sandy and weedy soil, many bugs, insects and repltiles were
found. Among them, poisonous creatures like snakes, scorpions and such were very
dangerous and one had to be careful. People were afraid of their bites. Normally
they could not climb up the cotts but \textbf{funkan} could climb up the bed,
will blow poisonous air in the sleepers mouth and wake him up with its tail.

To be safe of such creatures, homes must have good bed. Bed made by local
carpenters were called \textbf{micha} which were very simple. Good local bed
frames were made up of \textbf{rohida} wood and its legs used to be engraved with
designs. Bedframes made in Hyderabad were called \textbf{khat}. They were popular
because of their \textbf{sangheda} wood legs, fittings and height.

Gujarat's Vadodara district's Sankheda town was famous for its bedframes. When
people visited this town for business, they fondly bought bedframes with lacquer
work in their legs. They also engraved or painted their names on the bedframes and
brought them home. Such bedframes were also given to daughters in \textbf{bihnda}.
Small such bedframes were given to the daughters at the time of childbirth.

To make a complete bed out of these bedframes, weaving was required. A special
type of rope called \textbf{waan} was used for weaving. The weavers were expert
people in the art of weaving. Two different color ropes were used for weaving
and different designs were made while weaving. To begin weaving, first, two-feet form one
side's hump, a life-knot was tied and rope was weaved around it which was called
\textbf{vadaan}.

Another style of  weaving starts from one corner and continues in angle throughout. Such
angled weaving was done without vadaan. While weaving, people used to keep
little threads below the frame where rope was winding so that the rope does not
slip away. These little pieces of rope were called \textbf{kuti}. Once the
weaving is completed, these kutis were removed. In the end the ropes were pulled
for tightness and tied into a knot. Special care was take while pulling so that
the legs do not get disbalanced. Instead of cotton or waan, sometimes the ropes
were made from the thread-yarn obtained from the \textbf{aakdo} (Calotropis) plant's flowers.

Later on instead of ropes, plates of textile or other materials were used. These
plates were called \textbf{nivar}. In the case of nivar, instead of weaving at
an angle it was weaved straight. First, between two bedposts, plates
were wound 9 to 11 times. This was followed by weaving the other plates in the
right angle of the previous plates from the other side of the bed. This style of
weaving was called \textbf{tanipeto}. With this weaving, sometimes
\textbf{chopat} game's design were also made. In a variation, 3,2 and 1 plates
were weaved with the \textbf{Bharat} material and the other 3,2, and 1 plates
were weaved with \textbf{khanat} material. This design was stronger than the
previous ones. Once the weaving is completed, the nivar was pulled and the end
was sewn. Such beds did not have vadan. These types of beds were soft and
comfortable compared to the rope beds. During usage when the bed gets lose
\textbf{like a swing}, then the ropes or nivars were pulled and tightened again.

Rugs (\textbf{faraasi}) were spread on these beds. Matress and bedsheet were put
on rug. Matress were called \textbf{pathrani} which were made by filling of 3.5
to 5 kilogram of cotton-wool. Mattress were covered with cloth cover which was
washable. This cover protected the matress. 

As a blanket with bed, quilt or duvets were made. These quilts were made up of a
filling of 1 to 1.5 kilogram of cotton-wool. Another kind of blanket was called
\textbf{sirakh}. Sirakh were thick and made with more quantity of cotton-wool.
The meaning of Sirakh is Si (cold) + Rakh (keep) or a blanket which keeps cold
away. These were also covered with a cloth cover. It was a tradition to give
silk-satin quilt to daughter in marriage. 

Such matress-quilts etc. came later. In early days, woman of house used to make
blankets from rags of cloth found in homes. These blankets were called
\textbf{rali}. Old cloth and rags were collected from home and from it a
\textbf{leh} was made. Two equal pieces of jute-cloth were taken and sewn from
three sides to make the outer cover. One end was kept open unsewn. This  was
opened inside-out and spread on the floor. On top of this leh was put as much as
required. This leh was then spread out and then a well-practiced woman would
invert the cover in such a way that the leh does not get heaped up and can get
evenly spread between the two layers of the cover. After doing this the cover
and leh (cover + leh = rali) were stitched together. Daily a little bit of
stitching was done. If the stitching was not done over a long time, the rali had to be tied
around a wooden stick. Beautiful designs were made while stitching. Oftentimes,
colorful cloth was used for the cover. Rali for kids was called \textbf{ralka}
which were made fondly for kids. Silky cloth was used to make the cover and was
adorned with shiny threads. Such ralkas were given to children of daughters and
also son's children. Rali was used as matress as well as blankets. 

In winters blankets made up of wool called \textbf{khatha}, \textbf{kambal},
\textbf{kambalia} were available. They were very protective of cold weather.

In summers, cotton sheets called \textbf{khes} were used which were brought
popularly from Sindh's Nasarpur town. Islami style \textbf{Ajrakh} kept body cool in
summers. 

To support head, pillows (\textbf{vihana-osisa}) filled up of cotton-wool were
made. Pillows were covered with pillow-covers. These pillow-covers were
embroidered with users name or phrases such as ``good night", ``happy dream",
etc. In the corner nice creepers were embroidered. Such embroidery was also done
on bedsheets.

In addition to big beds, small beds called \textbf{michli-khatli} were made for
children. People who do not have a swing \textbf{ghodiyu} used to put a little
stone under one of the legs of such bed and rocked it across to give the bed a
rocking motion so that child sleeps well.

Wealthy people had big bedded swings and they used to swing in them
(\textbf{ludta}). If some sick person has to be taken somewhere on camel then a
special kind of bed called \textbf{kajao} was tied to the camelback on which the
person could sleep comfortably.

To spread on floor, cotton \textbf{farasi} was used. Wealthy people used
\textbf{galicha}. \textbf{Karad, jaroi} weaved out of camel and goat hair was
used. These were strong and durable. These all kinds of beddings were called
\textbf{hundh} in Thar and to store them wooden \textbf{dahnchi} was made.

In the event of someone's death, people used to sleep on sheets on floor without
matress. This kind of sleeping tradition was called \textbf{ukhrande suta}.

\section{Business and Employment}
Maheshwaris used to do business, farming and livestock raising. When they
arrived in Thar, they found soil and environment similar to Marwar, so they
found it suitable and continued their original business and economical activity.

Thar's Maheshwaris used to put shops in the nearby villages of their own towns.
Every town and village of Thar had such shops of Maheshwaris. Only one
Maheshwari used to put a shop per village. They used to sell cloth, spices,
jaggery, sugar, silver jewelry, utensils etc. on such shops. In leiu of these
they used to take grains, ghee, wool, ``jiroi", ``kharad", ``khatha", blankets
etc. under the barter system of trade. Additionally, rural people had little
cash in those days. They used to lend money on interest from ``vaniyas"
(Maheshwari) on occassions like wedding-death. They had to mortgage their
jewelry against money and had to pay a good amount in interest. This interest
business used to be profitable to Maheshwaris. Shopowners used to stay at the
village shops for 4-6 months. The family stayed back in their towns. So they
managed all their household activities like cooking themselves.

In some villages where there were no such shops, Maheshwaris used to go with a
big bag of stuff to sell (called ``khadiyo" with 2 sections hanged on the either
side of shoulders). They used to shout in the village: \textbf{``ahe koi maai,
tol vatthan vari?"} -- which means, is there any lady who would like to buy
jewelry?

Some Maheshwaris had fields on the outskirts of the towns. These were given to
kolis and bheels on ``haarap" for cultivation. The grain thus produced was
partly given to them, partly used for domestic use and the rest was sold away.
In monsoon Chibhri, kaaring, Guvar etc. was available. Milking animals like
cows, buffaloes, were domisticated and used to produce enough milk, butter,
curd, butter-milk, ghee etc. for the domestic usage. Ghee was traded in big
guantities with Kutchtch and Gujarat. Thar's ghee was famous. Some Maheshwaris
used to sell oxes. Oxes from Thar used to go to Kutchtch for sale in return of
silver, wooden beds, ``gaj", ``atlas", small wooden cots (from Sankhda, District
Vadodara) etc.

Maheshwaris had shops of grains, cloth, cutlery, grocery etc. shops in cities.
They used to import wheat and rice from Sindh. Colored cloth was imported from
Gadhada, while pattani dhoti-jota were imported from Patan. Jaggery from Sindh
and sugar was imported from Sumatra.

In the headquarters of TharParkar district called Mirpurkhas, Mithi's Tejomal
Jagani had a big financial establishment. Jagani family was called ``lakhpati"
in those days.

Some Maheshwaris had shops in Naukot also.

Umarkot's Tikamdas Ramjimal Kakkad was the owner of many acres of land. He was a
big landlord. First English education started in A.D. 1912 in Thar. First batch
of metric students came out in A.D. 1919-1920. Students passing metric  in those
days and after used to work as clerk's in the government's departments such as
customs, courts and revenue. Some got special training and became postmasters or
teachers in schools. These service class people were around 2-3\% but were
better respected than others in the community. Their lifestyle was visibly
different than others.
\section{Festivals}
In order to escape from the boredome of routine life, and enjoyment,  society has made
arrangements for festivals. Celebration of festivals are arranged such that they
fall on the days according to the seasons, religious events, political events,
or social systems. These festivities blossoms the hopes and aspirations of human
heart. The stories and legends related to these fesrivals have been preserved by
heart by the women of the community. They enjoy these festivals by reciting
these legends for each festival. Ledies also perform religious ceremonies
related to the festival. These festivals are also attached with the scientific
viewpoints and the traditions associated ensures a healthy and balanced life for
the people celebrating them. 

Since the time of immigration from Marwar, Maheshwaris have strived to preserve
their identity as per traditions. To this effect, along with their language,
clothing and garments, jewelry, makeup, they also strived to preserve their
festivals.
\paragraph{Marwari Year}: A Vikram Samwat (calendar) year in Marwar starts on
the first day \textit{sud}\footnote{The waxing phase of moon is called sud and
the waning phase is called vad} of Chaitra month. Whereas in Gujarat, the month
starts on first day of Kartak sud. Means, in Gujarat the new years starts 7
months after it starts in Marwar. However, a Marwari month starts from the vad
fortnight followed by the sud fortnight whereas the Gujarati month starts with
the sud fortnight followed by the vad fortnight.
So, each month has a difference between Marwari and Gujarati Calendar as shown
in table \ref{tbl:cal}.

\begin{table}
\begin{center}
% use packages: array
\begin{tabular}{l|l|l|l}
\hline
\textbf{Gujarati Month} & \textbf{Marwari Month} & \textbf{Gujarati Month} &
\textbf{Marwari Month}\\ 
\hline
Kartak sud 2057\footnote{Gujarati new year} & Kartak sud 2057 & Magh sud 2057 & Magh sud 2057 \\
Kartak vad 2057 & Magshar vad 2057 & Magh vad 2057 & Fagan vad 2057 \\
Magshar sud 2057 & Magshar sud 2057 & Fagan sud 2057 & Fagan sud 2057 \\
Magshar vad 2057 & Posh vad 2057 & Fagan vad 2057 & Chaitra vad 2057 \\
Posh sud 2057 & Posh sud 2057 &Chaitra sud 2057 &Chaitra sud
2058\footnote{Marwari new year} \\
Posh vad 2057 & Magh vad 2057 & Chaitra vad 2057 & Vaishakh vad 2058 \\
\hline
\end{tabular}
\end{center}
\label{tbl:cal}
\caption{Gujarati and Marwari Calendar}
\end{table}
As seen in table \ref{tbl:cal}, in every fortnight, sud is the same but vad has
a difference of one month.

Thar's festivals are associated with Marwari calendar still coicides well with
the Gujarati calendar. For instance, Diwali in Gujarat falls on Aaso vad New
Moon day while in Marwar it is Kartak vad New Moon day. Both days are counted as
different on calendar but fall on the same actual day. All the festivals
celebrated by hindus of Hindustan are celebrated by us with minor differences
and additions of some new local festivals.
In the following description of festivals, Gujarati tithi is mentioned and where
the festival falls on a vad part, the Marwari tithi is mentioned in the
brackets.
\paragraph{Ishwar Gavar/Gangor.} This festival falls right after Holi. Ishwar
means Shiva and Gavar means Parvati. When the Spring season is on its full
blossom, this festival is celebrated. On the next day of Holi, that is on the
day of Dhuleti, unmarried girls would wash hair, put on new clothes and went
where the Holi was lighted. They would cool down the lighted Holi with water,
which was called \textbf{Chhanto nakhyo}. From there, they would make fistful
shapes of wet ash and brought some of them to someone's home. Here, they would establish
Gavar from the ash. This was followed by making statues of sand and seeding of
\textbf{jvara}.
After that girls used to put on new clothes and jewelry everyday and used to
offer water (\textbf{arag}) to sun before breakfast. They used to wish well for
their parents and in-laws. The song they used to sing is given in the appendix.
This routine used to continue until the the farewell of gavar.

After that fasts (\textbf{ektana}) started. In these ektana, they used to have
one meal a day. Fagan vad choth was called chothio ekano. On this day, ladies
used to make \textbf{behdalia} in which they used to stack 9 or 11 gagar, moriya
or lota etc. in ascending order. This behdaliya was carried to the temple or
well by some girl or middle-aged woman. After singing gavar songs, they used to
put some water in a small lota and sprinkled the water around. Newly wed woman
used to observe ektana and the girls used to gather other women who observed
ektanas. Similarly, on vad aatham (eighth), athalio ekano was observed. Chothio
and athalio were observed only by the unwed girls.

Athalio ektana was also called Surajio ektano. That day, after worshipping the
sun, a sun shaped \textit{tikli} was prepared and a hole was made. Sun was seen
through this hole. Then the girl used to say \textbf{sakhi, sakhi, mi surajio
diththo; diththo ehdo tuththo}, which meant, sungod, bless me and make my wishes
come true.

The non-widow married women used to observe a \textbf{dasalio ekana}
on the tenth vad day. From vad 11th day till Chaitra sud 2nd day, there were
seven such ekana were observed, which were called \textbf{bilke}. The non-widow
married women used to invite little girls for meal after these fasts.

The last ekana was observed on Chaitra sud third day belonged to
\textbf{gangor}. This ekana was observed by both unmarried and non-widow married
women. On that day, Ishwar-Gavar were married. Similar to a real wedding,
wherever Ishwar-Gavar is established, there the family became parents, sing
wedding songs and followed all rituals for this wedding. The next day on Chaitra
sud forth, similar to the farewell to a wedded daughter, the Gavar was given a
farewell and the figure was put on the bank of lake on the outskirts of the
village. Along with the Gavar, a raw earthen pot was carried which was broken at
the lake and the pieces were brought back and put inside the storage of grains
and beleived that will increase the prosperity and grains. This festival is
celebrated fondly in Marwar.
\paragraph{New Year.} Marwari new year begins on Chaitra sud first day.
(However, since Chaitra month starts from vad, the day comes after 15 days and
that is considered the beginning of the month.) There was not much celebration
of  this day in Thar. This day falls in the middle of the Ishwar-Gavar fasts.
\paragraph{Ramnavami.} Chaitra sud ninth day is the day of birth of Lord Ram
means \textbf{Ramnavami}. Ladies folks observed a fast. A Ram birth festival
used to be celebrated at noon in the temples.
\paragraph{Hanuman Jayanti.} Full moon day of Chaitra is the birthday of the son
of Goddess Anjana means Hanuman Jayanti. It was called \textbf{Ajadi}. Women
folks observed fast on this day. Clay figurines of
\textbf{Ajado-Ajadi}(Hanuman-Anjani) were made and covered with a cloth-piece. Large \textbf{rotas} of thick wheat-flour were
cooked. Separate rota per person were cooked. After breaking a coconut and
doing rituals, Ajadi's story was told-\textbf{Rota moti kor de, saagi roto aane
de}. Such rotas could be eaten by the boys of the home but the girls who did not
observe fast could not eat them. Married \textbf{niyani}(girl) was invited by
sending a \textbf{dhamu} or inviation. Ghee and powdered sugar was spread over
the rota and used to be eaten with spiced curry of \textbf{sangri}. The next
day, ajado-ajadi were left under a tree outside the town.
\paragraph{Panthewari.} During the whole Vaishakh month, girls and newly wed brides go to
temples to sing panthewari. There they used to sing Krishna devotion songs.
After making \textbf{sakhio} (Swastika) sign of grains, they used to listen to
stories. Worshipped tulsi (the basil plant) and Pipal tree (the sacred fig
tree). At the end of the month, clothes were donated to Brahmin women.
\paragraph{Akhatrij.} Vaishakh sud third day is \textbf{Akhateej}(akhatrij). In
every household two each heaps of bajra (barley) and moong and one heap of guvar, thus five
heaps were made. Niyani were given small water pots. Boys used to dance with
\textit{ghungroo} tied to their feet. \textbf{Rihans} were organized in the town
where sprouted barley and sugar were eaten. \textbf{Amal} (Opium) was used
freely. At the home of the head of the village, small wheat flour pits were made and
water was poured into these pits. These pits were named after the months
Vaishakh, Jeth, Ashad, and Shravan. It was believed that it will rain plenty
much in the month whose named pit leaks water first.
\paragraph{Vaishakhi.} On Vaishakh sud full moon day, niyanis were invited over
for meals or \textit{tikli} with sugar-ghee were sent to her place. Raw material for food
were donated to temple.
\paragraph{Nrijala Ekadashi.} Jeth sud 11th day is called \textbf{Bheem Agiaras}
or \textbf{Aamba Agiaras}. Women observe fast on this day. Some women even
refrained from drinking water. Some women used to have sharbat (a sugar-based
drink) after 12 noon and used to eat Mango. Niyanis were given mangoes and
sugar. 
\paragraph{Nani (small) Trij.} The \textbf{Nandhi theej} festival falls on Shravan sud
third day. Unmarried girls used to apply henna on the previous night. They
observed ektanu on this day.
\paragraph{Veer Pasli.} The Sunday after Nani Trij was called the day of Veer
Pasli. On this day, brother's sisters used to take grains and sesame seeds to
her parent's home and burnt them and used to say: \textbf{til badin, jav badin,
bhai ra dushman badin} which roughly translates to: the way these grains are
burning, let my brother's enemy burn like that. Sisters had meal at the brother's.
\paragraph{Pavitra Agiyaras.} On the eleventh day of Shravan sud, women used to
colour threads. Such threads were called \textbf{pavitra}. They were offered to
god in a temple as decoration.
\paragraph{Ak Chathth.} On the sixth day of Shravan sud, new-born baby and newly
wed couple were worshipped using \textbf{aakdo} plant's leaves, coconut and offered
sprout barley grains to eat.
\paragraph{Bin Baras.} On the twelfth day of Shravan sud, women made sand walls
on the lake-shores and offered betel-nuts, kumkum, coconut in worship. The
betel-nuts were sent to brothers.
\paragraph{Rakshabandhan.} The full-moon day of Shravan month is Rakshabandhan.
On this day, unmarried sisters at home and married sisters used to come to the
homes of their brothers to tie the \textbf{rakhdi} thread on the right wrist of
their brothers. If the brother is away in another town, they used to send rakhdi
thread by post. Sisters used to bless their brother after offering sweets. If
some brother doesn't have real sister then cousins used to tie rakhdi. Similarly, if
some sister  doesn't have a real brother they used to tie rakhdi to their
cousins. Brother used to give money or cloth to the sister who tied rakhdi.
Brahmins also came to home to tie rakhi and obtained gifts and/or money.
Businessmen used to tie rakhdi to their inkpot-pens. 
\paragraph{Kajali Trij.} Shravan vad third (Bhadarva vad third) was called
\textbf{vadi teej}. This day was considered a big festival. On this day
unmarried girls and middle-aged women also observed fasts. On previous night,
they applied henna and observed \textbf{ghamodi} (the breakfast of sweets on the
previous day of fast). They tied swings on the tree-branches and play swing
there. They used to go to sand dunes \textbf{dheba} in the evening to play. At
home, they played cards. Engaged girls were sent sweets like \textbf{penda},
\textbf{gundarpak}, \textbf{halva} or dry fruits from the in-laws. In the night,
men and women used to gather at a high place such as the terrace of a house and
waited for the moon to show up. As soon as a little edge of moon was visible,
they used to shout in joy to call others. At this time the moon was offered
\textbf{ardhya} of water and rice.

The leaves of Aakdo plant were brought home. One of them to sit on, one to put
on the head, one to put under the plate, and in one of the leaves, milk was
taken to drink. This was followed by a breakfast of sattu and tikli-shak. (Sattu
is a sweet made up of ghee and cooked flour, tikli is fried roti and shak is
vegetable spiced and cooked like curry.) Married women come to parent's home to
observe this fast and the breakfast ritual. If the moon is not visible because
of monsoon clouds, then the next day, after looking at the morning sun the fast
was broken. Unmarried girls observed teej fast and continued to do so after
getting married. After observing sixteen such teej-fasts in sixteen years, the
teej was celebrated in conclusion.
\paragraph{Ubh Chathth.} On the day of Shravan vad sixth (Bhadarva vad sixth),
women dress up and go to the brahmin's home to listen to  the stories. This was
followed by the ritual of tying a rag cloth on a finger and standing up. When
tired, the rag cloth was passed on to another women who stood up. After seeing
moon in the night, they used to take meals.
\paragraph{Randhan Chathth.} Shravan vad sixth (Bhadarva vad sixth) is also the
day of Randhan chathth along with the Ubh Chathth. Women started cooking in the
afternoon. To eat cold food the next day of Sheetla Satam, they cook
\textbf{tikli} of jaggery, crispy \textbf{puri} of wheat flour. Also cooked rice
and added \textbf{Raito} (cooked buttermilk in oil and spices) to the cooled
rice. Each food item were kept \textbf{abot} (untasted, untouched), so that it
could be used in the ritual the next day. After the night's meal when everyone
has finished eating, the stove in the kitchen was sprinkled with water to make
it cold. This ritual was called \textbf{chulho thadho karyo}. After that the
stove was not lighted in the night.
\paragraph{Shitla Satam.} Shravan vad seventh (Bhadarva vad seventh) was called
\textbf{Thadhi Satam}. That day, in the morning, at the temple of Shitla godess
or at a place where Shitla goddess is designated, women used to take a plateful
of cold food. Along with the food, they used to carry Barley flour-ghee-jaggery
made \textbf{kuler}, coconut and performed the rituals and offerings. After
that, the family members took the coconut-kuler offering and had their meals.
People used to eat cold food the whole day. No stove was lighted that day.  
\paragraph{Janmashtami.} Shravan vad eighth (Bhadarva vad eighth) is celebrated
as a big day of Janmashtami. It was also called \textbf{Gokala Aatham}. On this
birthday of Lord Krishna, everybody observed a fast. Lord Krishna's temples were
decorated. The festival was celebrated at midnight. An offering of
\textbf{panchamrut} (milk+honey+curd+ghee+sugar) and \textbf{panchajiri}(carom
seeds+ghee+sugar) were given at the temples.
\paragraph{Vachch Baras.} Shravan vad twelfth (Bhadarva vad twelfth) was the
festival of Vachch Baras. On this day, women worshipped cows which have newborn
calfs. Also listened to the stories of cow-calf. Instead of using iron utensils
like spoons and pans, clay utensils and pans were used. Green-grams and
barley-flour roti were cooked. Instead of cowmilk, ghee they used buffalo milk
and ghee. Also observed ektanu fast.
\paragraph{Gauna Oaas (fast).} From Shravan vad thirteenth (Bhadarva vad
thirteenth) till Bhadarva sud first these fasts were observed. On first day,
women washed their hair and put on new clothes and lighted lamps in the temples
and do rituals in the compounds. During the fasts, an unbroken lamp was lighted
at home. Every morning ant's hole (\textbf{nagro}) were filled with flour and
the lamps were filled with oil/ghee. Every night they slept on bare floor. They
could eat curd, sugar, melons, etc. (\textbf{faradi}) during the fast. On the
forth day, which is the first of Bhadarva, they circumambulated
(\textbf{parkamma}) the village and distributed money to little girls.
\paragraph{Ganesh Chaturthi / Gunes Choth.} Bhadarva sud fourth is the big festival of Ganesh
Chaturthi but it was not celebrated in Thar. Instead of this, women used to
observe a fast on every month's fourth day or on a few month's fourth day. After
seeing the moon and offering the water, they used to have meals. After the
wedding of their son/daughter, mothers wished and used to observe five
Ganesh Choth. So, on the next month's fourth day, herself and four other women
from the family or the neighborhood used to observe the fast of Gunes Choth. The
mother of newly wed son/daughter used to send fruits etc. to the homes of these
women and invited them in the evening to see the moon and have meal together.
\paragraph{Shradhh.} During the vad part of Bhadarva (Aso vad part) the Shradhhs
are observed for the peace of paternal souls. On these days, the abot food is
taken in small quantities and offered to the birds from the terraces of homes.
This was done by the males of the family. There was a tradition of preparing
\textbf{kheer} (milk+rice pudding-like sweet). Almost every family had
\textbf{dhinu} or the milking animal: cow, buffalo etc. If the milk is not
sufficient then milk from neighborhood was taken in lending and returned when
they had a shradhh to observe. Brahmins were offered meal and betel-nuts and
money were given as gifts. Sometime, cloth were also offered.

Shradhh were observed for the dead men-women of the family for upto 2-3
previous generations. Meaning, the living persons parent's and grandparent's
Shradhhs were observed. Last day means new moon day was called \textbf{sarva
pitri} amaas or \textbf{sol saradhi}. This day Shradhh was considered a Shradhh
for all the paternal souls. If brahmins were not invited then all the raw
materials for meal were given to the temples.

Deceased person's first Shraddh was the following year of the first anniversary
of death. The same day of the death by calendar was the day of Shradhh in the
fortnight. On the first Shraddh, sistes, daughters and relatives were invited
over. 
\paragraph{Navratri.} Aso sud first till ninth are called \textbf{navarta}.
Every home had incense and lamp. Seeds were planted in the clay trays in
temples. Their, goddess was worshipped and \textbf{khak} (holy ash) was put on
tongue and forehead. Women observed ektana fast for nine days. On the eighth day
a big religious ceremony was held on which they observed a fast.
\paragraph{Dashera.} Aso sud tenth is the Dashera festival. This is the day when
Lord Ram won over Ravan. It is also called Vijaya Dashami. Rajput's of Thar
worshipped their weapons on this day.
\paragraph{Sharad Poonam.} Aso month's full moon day is Sharad poonam. On this
day, in the night, on the terrace of homes, \textbf{Misri}(crystal sugar
nuggets) was kept in a bowl covered with a perforated lid. This was eaten in the
morning as an offering.

On the night of Sharad poonam, girls used to thread a needle 108 times in the
light of the moon. This was a job of perseverance and patience. It was said that
this improved the eye sight.
\paragraph{Dhan Teras, Kali Chaudash and Diwali.} Aso vad thirteenth (Kartak vad
thirteenth) was the beginning of Diwali festivities. Monsoon being ended, the
availability of foodgrains, milk and ghee used to be good. These festivities
were spent with much joy and fun. Homes were cleaned and dusted. Schools
observed a vacation. People who lived away from home for business etc. returned
home on these days. Kali Chudash was also called \textbf{Roop Chaudash}. That
day people washed their hair with oil and Fuller's Earth (Multani Mati). On Diwali
night, goddess Laxmi was worshipped. Books of accounts were also worshipped.
Rates and cost of commodities such as foodgrains, ghee, gold and silver were
recorded in the books and prayer were offered to goddess Laxmi to wish that she
gave double and quadruple the prosperity she gave in the last year. The offering
of the worshipping were distributed among relatives.

Engaged girls were sent sweets by her in-laws. Every home had lighted
earthen-lamps. Children celebrated with fire crackers. In meals, sweets,
lentils, puri and \textbf{lasso khich} (soft rice) were made.

In the night, children used to tie rags on the sticks of the Calotropis plant
and lighted them to form a torch and took them to the outskirts of the town.
These were called \textbf{Mereiya}.

Newborn kids were sent clay toy hut, sweets and dresses while singing songs from the maternal grandparents and paternal
aunt \textbf{bhua}. While bhua received a piece of cloth from her parents.
During the 4-5 days of Diwali festivities, children used to play with toy guns
(gudadia) made by town's ironsmith. They used to put little balls of gunpowder in the toy
guns and cracked them. Gunpowder was mixed with coal powder and wrapped in a
\textbf{sushi} and burned like a cracker. Also crackers were bought from market.
These days were passed with much fun.
\paragraph{Annkoot.} On the day of Kartak sud first, different kinds of sweets
were offered to gods in temple and were distributed as holi offerings.
\paragraph{Baliraja.} On the day of Kartak sud first, outside of the home a
figurine of Baliraja was made out of cowdung and was kept in the sleeping
position. This was called \textbf{bad raja}. Children cracked firecrackers there
and untie the rakhdis.
\paragraph{Bhai-beej.} Married girls came to their parents place for meals. At
home, women used to whitewash the walls and drew small footsteps out of red
sindoor. The number of footsteps were same as the number of brothers she had.
\paragraph{Tulsi na Upwas.} In Kartak sud \textbf{tulsi-ra-oaas} started. Women
observed fast from the eigth till the twelfth day. They only had water and used
to sleep on floor during these fasts. On full moon day when the Kartak month
ended, they used to donate things like clothes and household items. Parents
invited girls for meals.
\paragraph{Makar Sankranti.} English month January's 14th is celebrated as Makar
Sankranti. Earlier, it was celebrated on 12th and then 13th. When sun entered
Capricorn, this day was celebrated. Sweets made of sesame seeds were prepared
such as \textbf{tilodia} and sweet balls. Sesame seeds and raw rice+lentils were
sent to temple and daughter's home. Newly wed women used to take coconut,
clothpiece, money etc. in a new clay pot to the temple. This ritual was called
\textbf{Mandir kholyo}. Married sister-in-law was given cloth piece and
brother-in-law was given a towel.
\paragraph{Vasant Panchami.} Magh sud fifth was the day when everyone used to
visit temple. Winter was considered over. \textbf{Vasant panchami ra vaja
vagiya, naag-bichchu sab jagia.} (roughly translates as the springs arrive the
snakes and scorpions awake from their slumber.)
\paragraph{Maha Shivratri.} Magh vad thirteenth (Fagan vad thirteenth) was the
day of \textbf{Shivrat}. Everybody observed a fast on this day. Engaged girls
were send sweets by their in-laws. Festivities and fairs were organized at the
Shiv temples \textbf{Shivalo}. Elderly women chanted songs of devotion. While
the young girls and daughters in law played cards. Sea shells were also played.
They used to stay awake the whole night. In the early morning, everyone used to
go to the lake of town and collected 108 pebbles and brought them out of lake. 
\paragraph{Amali Agiyaras.} On Fagan sud eleventh, in the compound of the temple
of the town, women used to plant a branch of khejda tree. They did
circumambulate the tree four times and after fifteen days observed a
\textbf{saamdi Agiyaras}.
\paragraph{Holi.} Fagan sud full-moon day is the Holi festival. On the outskirts
of the village, after the dusk, a holi of dried wooden branches was lighted. New
born children and newly wed couple were taken for rituals. The holi was
circumambulated. Coconuts were broken. If the smoke of holi goes to the southern
direction, it was believed that the rains will be good in the monsoon season.
Everybody wished for a prosperous coming year. Brahmins and Maheshwaris used to
sing traditional songs at the temples. Brahmins used to sing erotic songs on the
previous day. Children used to tie musical straps in their legs and became
\textbf{Heer-Ranjha} and collected money. Women observed fast and had meal after
the holi worshipping rituals in the evening. Maternal grandparents and paternal
aunts (bhua) sent new clothes and sweets to their nephews-neices. These gifts were
called \textbf{Dhundhaniu}. On the other hand parents gave cloth piece to bhua.
\paragraph{Dhuleti.} In temples, god's statues were put in small swings and a
colored powder \textbf{gulaal} was kept in a dish in front. Everyone used to
sprinkle gulaal to the god and put some money as offering to another dish
nearby.

In the evening, engaged girl's and boy's parents (in-laws) used to celebrate by
sprinkling gulaal on each other. Brother-in-law, if married, gave one coconut to
his brother-in-law or two if unmarried but engaged. In the evening, married
daughter and son-in-law were invited over for dinner. Sisters used to put her
sugar-necklace over the head of her brother. These sugar necklaces were called
\textbf{harodia}. Brother gave money to sister.

