\chapter{Infrastructural Necessities of the Community}
\section{Dhatki (Thari) Language}
Many languages are spoken in India. Every region has a different language or should we say regions are made languagewise. Thing every person who uses to express his feelings is dialect. Inter-human relationships are different at different places and depends upon geography, business and community. And so is the dialect. Still dialect maintains the characteristics of its place of origin. How-ever one tries to hide but in the time of trouble one would send a call of distress in his own dialect.

It is said that every 12 miles the language changes. So the language at one end of a region might be considerably different than that of the other end, and sometimes it becomes even difficult to understand. Based on such languages, it is decided what part the speaker comes from. For example: In Gujarat, people from Kutchtch, Saurashtra, Mahesana, Surat etc. have distinct and identifiable accent and style of speaking.

Formal language means a language for general purposes, administration, education and social interaction. In that way, dialect is specific to a particular region but a language spans the whole country. Indian constitution has officiated several languages. After this introduction, let us see about the Thari/Dhatki language.

Thar Desert (The Great Indian Desert) is considered to be spread across South edge of Punjab to the west of Rajasthan to the Khairpur district till the south of TharParkar District upto the Great Rann of Kutchtch. Maheshwaris migrated from that region to the TharParkar region of Sindh and the dialect they spoke was so called Thari from the Thar Desert. People settled in the ``Dhat'' region called their dialect ``Dhatki''. As per the Encyclopedia Brittanica, vol. XVI, page 781:
\begin{quote}
 DHATKI, a dialect of Rajasthani is spoken in south-eastern TharParkar District.
\end{quote}
As per the 1931 census of India (Bombay Presidency):
\begin{quote}
Thari/Dhatki is regarded linguistically as a dialect of Sindhi but enumerated as a separate language in census. For this procedure, there is a clear authority as THARI is recognised in Sindh as a distinct from Sindhi and has an area of its own.
\end{quote}
George Gearson authored linguistic survey of India indicates that:
\begin{quote}
The language of TharParkar and Jaiselmer is mostly standard Marwadi. It has a mixture of Sindhi and Gujarati to a little extent only.
\end{quote}
According to Shri Bherumal Maherchand Advani Authored ``\textit{Sindhi Boli ji Tarikh}'', ``A new kind of language has been formed by a combination of Sindhi, Marwadi, and Gujarati. It is called Dhatki means language considered to be spoken in Dhat. This mixed dialect is considered an alternate to Rajasthani but is very close to Gujarati.''

According to what is indicated in the Gazetteer of Bombay Presidency, CUTCH, Feb, 1880, Chapter III, Population: Traers, page 50 \& 51, ``Maheshwaris arrived in Kutchtch approximately 500 years ago via Nagor--Thar and settled in the Abdasa Talluka. They spoke Thar-Gujarati language, used to put on turban like the Baniyas of Thar ... etc''. (Note: In the above writing, the mention of Thar is used in the sense of \textit{Greater-Thar} means the Great Indian Desert area and not the `Thar' TharParkar area where Maheshwaris settled late. In around A.D. 1300, Maheshwaris spoke ``Thar-Gujarati'' language which is likely to be a mixture of Thar's Marwari and Kutchtch's Gujarati.)

The dialect Maheshwaris brought from Marwar and the one spoken in dhat went under the influence of the dialects spoken in the surrounding region resulting in many gradual changes. These surrounding languages includes Gujarati in the east, Kutchtchi in the South Sindhi in the west and Rajasthani (Marwadi) in the North. Like this, variations of basic dialect resulted in Dhatki language.

For some years Thar was under the administration of Kutchtch state's ``political agent'', so the official language was Gujarati. This was also an influential factor on the Thari dialect.

In the school's of Thar, initially Gujarati, then Gujarati and Sindhi and later on in approximately A.D. 1940, only Sindhi was taught. Apart from that the Baniyas of Thar used to write ``Modi'' Gujarati (basic Gujarati characters without additional accents) in their books. This Gujarati was called ``Vaniki'' gujarati.

When Maheshwaris migrated from Marwar they came from Jaiselmer to Umarkot via Ratokot. After that they started living in Thar according to their convenience and started speaking Dhatki. But those who came from a different route from Jaiselmer via Sakhkhar to Sindh region, then Sahevan, Tando Allahyaar, Tando Adam, Badin etc. places or came after some time had influence of Sindhi language on their dialect. Maheshwaris living in Tando Allahyaar and Tando Adam were called ``Tandai'' and their dialect has clear influence of Sindhi. Table \ref{tbl:difftharitandaiguj} throws some light on this fact:
\begin{table}
\begin{center}
% use packages: array
\begin{tabular}{lll}
\hline
\textbf{Thari Dialect} & \textbf{Tandai Dialect} & \textbf{Gujarati} \\
\hline
Kahaan dyo & Chavan dyo & Kaheva dyo \\ 
leela gabhbhaa & aala kapda & bheena lugda \\ 
mi sambhalyo & mu budho & me sambhalyu\\
\hline
\end{tabular}
\end{center}
\caption{Differences between Thari, Tandai and Gujarati}
\label{tbl:difftharitandaiguj}
\end{table}
In the same way the dialect spoken in one end of Thar is different than that of the other end. This is shown in the table \ref{tbl:diffmithigad}.
\begin{table}
\begin{center}
% use packages: array
\begin{tabular}{lll}
\hline
\textbf{Dhatki in Mithi} & \textbf{Dhatki in Gadhado} \\
\hline
Paase mahin betho ahe & godhina betho ahe \\
puthyan aaye to & larinan aaye to \\
Dheba & Dhora\\
Tadha & Weri\\
\hline
\end{tabular}
\end{center}
\caption{Differences between Dhatki dialect as spoken in Mithi and Gadhado Villages}
\label{tbl:diffmithigad}
\end{table}
In the towns of Thar, Dhatki language was spoken by Maheshwaris, Brahmins, Bhojak, Shrimalis, Khatris, Malis, Sonaras, Rajputs (Sodha), Meghwal, Bheels, Bajeer etc.. But Lohanas and Muslims used to speak Sindhi however, they could comprehend Dhatki. In some villages, Muslims also used to speak Dhatki. Looking at these details, we can opine that: (1) Thari/Dhatki was basically spoken in Marwad which was brought by Maheshwaris and other communities during their migration. (2) Due to the influence of regional languages from all sides, their is some mixture. (3) Dhat's region that was closer to the other region's have more influence of their respective dialect. (4) School's language of teaching influenced the dialect. (5) Because of an increase in service class people, urban dialect differed from their rural counterparts.

Now let's see the technical and linguistic details of the Thari/Dhatki language:
According to Census of India-1911, Vol. 7, Bombay Presidency, page 168: Distribution of Total Population by Languages:\\
Family\hfill : Indo-European\\
Sub-Family\hfill : Aryan\\
Branch\hfill : Indian\\
Sub-Branch\hfill : Sanskritic\\
Group\hfill : North-Western\\
Language or Dialect \hfill : Thareli (Thari/Dhatki)\\
Total Population in TharParkar District = 3,95,235\\
Population Speaking Thari/Dhatki = 1,16,664\\
Male=64,794, Female=51,870\\
Total=1,16,664 ie. about 30\% of the district.
Now let us compare some Dhatki words with Sindhi and Gujarati (table \ref{tbl:words}).
\begin{table}
\begin{center}
% use packages: array
\begin{tabular}{l|l|l|l|l|l}
\hline
\textbf{Dhatki} & \textbf{Sindhi} & \textbf{Gujarati} & \textbf{Dhatki} & \textbf{Sindhi} & \textbf{Gujarati} \\
\hline
Ankh & Akh & Aankh & gaa & gaun & gaay \\
kann & kan & kaan & meens & meenh & bhains \\
nakk & nak & naak & vachhchhdo & gabho & vachchdo \\ 
dant & dandh & daant & chhoiyo & aadmi & purush \\ 
doodh & kheer & dudh & dosi & mai & stree \\ 
dahi & dahi & dahin & hek & hik & ek \\ 
makhkhan & makhan &  maakhan & bu & ba & be \\ 
gehun & kanak & ghau & tann & te & tran \\ 
mung & mund & mag & char & char & char \\ 
saag & bhaaji & shaak & panch & panj & paanch \\ 
chhah & jhan & chhas & dus & duh & dus \\ 
baap & piu & baap & meh & baarish & varsad \\ 
ma & amaa & maa & kirniyu & chhatti & chhatri \\ 
dikro & putt & dikro & kanglo & lagad & patang\\
\hline
\end{tabular}
\end{center}
\label{tbl:words}
\caption{Some words in Dhatki and their counterparts in Sindhi and Gujarati}
\end{table}

Some examples of sentences are shown in table \ref{tbl:sent}.
\begin{table}
\begin{center}
% use packages: array
\begin{tabular}{l|l|l}
\hline
\textbf{Dhatki} & \textbf{Sindhi} & \textbf{Gujarati} \\
\hline
tahjo naam ki ahe? & thunjo nalo chha aahe? & taru naam shu chhe? \\ 
maanh jo naam Mohan aahe & Mhunjo nalo Mohan aahe. & Maru naam Mohan chhe. \\ 
tu kith jaain to? & tu kithe vanji to? & tu kyan jaay chhe? \\ 
hun jaan mahin jaaun to. & maan jag me vanja tho. & hun jaanma jaun chhu. \\ 
taahje roti khaani ahe? & tokhe maani khappe? & tare jamvu chhe? \\ 
hun dhaapyal ahaan & mukhe dho aahe & hun dharai gayo chhu. \\ 
hek raja hanto. & hikdo raja ho. & ek raja hato. \\ 
ue re bu raane hante & tehnkhe ba raanyu huyu. & tene be rani hati. \\ 
hek rajkumar hanto & hikdo rajkumar ho. & ek rajkumar hato. \\ 
rajkumar vaddo thyo. & rajkumar vaddo thyo. & rajkumar moto thayo. \\ 
ooe ra lagan lya. & hunji shaadi kai, & tena lagna levana.\\
\hline
\end{tabular}
\end{center}
\caption{Some sentences in Dhatki and their counterparts in Sindhi and Gujarati}
\label{tbl:sent}
\end{table}
As seen in tables \ref{tbl:words} and \ref{tbl:sent}, the Dhatki language has been influenced by Gujarati somewhere and Sindhi elsewhere. Some dhatki words have been written in short form of Gujarati words. Means removing the `kaano' accent.

As per Thar's traditions and because of affection with each other, peopl's names were also shortened. We see some samples as presented in table \ref{tbl:names}.
\begin{table}
\begin{center}
% use packages: array
\begin{tabular}{l|l|l|l}
\hline
\textbf{Man's Full Name}  & \textbf{Shortened Name} & \textbf{Woman's Full Name} & \textbf{Shortened Name} \\
\hline
Ambaram & Ambo & Savitri & Saabi\\
Sukhdev & Sukho & Jashoda & Jassi\\
Maherchand & Mahero & Aasha & Aasi\\
Bhagwandas & Bhagu & Nirmala & Narmi\\
Hiralal & Hiro & Jaywanti & Jeti\\
Jethanand & Jetho & Draupadi & Dhuppi\\
Nandlal & Nandu & Rukshmani & Rukhi\\
\hline
\end{tabular}
\end{center}
\caption{Some Full Names in Dhatki and their Shortened Forms}
\label{tbl:names}
\end{table}
Articles appearing in Sindhi eg. jo, ja, ji and Gujarati eg. no, na, ni are replaced by marwadi style \textbf{ro}, \textbf{ra}, \textbf{ri}. For example:\\

Sindhi: hi chhatti keh ji aahe?\\
Gujarati: aa chhatri koni chhe?\\
\textbf{Dhatki: e kirniyu ke ro ahe?}\\

Dhatki have male and female gender but no neutral gender. Sindhi's `aahe' is `ahe' in Dhatki and its `tho' is `to'. Examples shown in table \ref{tbl:sent}.

There is no systematic literature available of Dhatki/Thari dialect. The language being colloquial, it transferred orally from generation to generation in the form of traditional songs, wedding songs, \textit{sawayas}, \textit{dhamalas}, \textit{shlokas}, festival songs, puzzles/riddles, proverbs etc.. These were spoken on occassions but are increasingly getting less spoken. Recently we heard that in Pakistan's Sindh state, ``The Sindhi Adabi Board'' tried to integrate, maintain and publish a collection of such sparse literature. In that publication's preface some such samples are provided. \textit{sawayas}, \textit{dhamalas} etc belong to the \textit{``pushtimargiya''} genre and so the Maheshwaris of Marwad must be belonging to that genre.

To include the Dhatki language in the Indian constitution, A Maheshwari Member of Parliament put forth a proposal in the Indian Parliament in A.D. 1992-93 but it was not accepted buy the parliament.

\section{Water}
Water is a primary need for humans, animals and plants. Thar being an arid land there was no river and it was not possible to bring up any canals. Here rainwater was the basis of life. Rainwater seeped into the soil was brought up by digging wells. In the ancient times, there was a river called ``hakdo'' that used to flow across the Thar which disappeared because of natural causes like earthquake. Because of this the water table went low in the north-east and up in the south. In Thar, the depth of a well is measured in terms of ``puras''. The length from the toe of a man (Purush) till the finger of elongated hand was considered to be one Puras. This is approximately six feet. The depth of well has been registered as shown in table \ref{tbl:well} in different places.
\begin{table}
\begin{center}
% use packages: array
\begin{tabular}{l|l}
\hline
\textbf{Area} & \textbf{Well Depth in Puras} \\
\hline
Samroti (Near Diplo) & 5 \\ 
Parkar (NagarParkar Talluk) & 10 to 15 \\ 
Kantho (North of Nagar and South of Chhachhro) & 15 to 35 \\ 
Dhat (area between Mithi, chhachhro and Umarkot & 40 \\ 
chhachhro, islamkot, mithi & 20 \\ 
Bhorillo & 30 \\ 
Kantyo & 20 \\ 
Chelhaar & 35 \\ 
gadhado & 60 \\ 
\hline
\end{tabular}
\end{center}
\caption{Depth of wells in Puras in the Thar Region}
\label{tbl:well}
\end{table}

Wells were known as ``tadha'' or ``tad''. Digging well was considered to be holy work. If a person funds to dig and build a well then the name of the well and the place was called after that person. For example, ``Meghe ro Tadho'', ``Dane ro Tadho'' etc..

Wells used to fetch water with different tastes. Different tastes had their names eg. \textbf{Kharo}, \textbf{Charko}, \textbf{Baalo}, \textbf{Kasaro}, \textbf{Ugro}, \textbf{Mitho}. If sweet water was not available, people made it do with the slightly salty water or the water with other tastes.

Wells being in the arid and sandy land in Thar, they were used to be built using special bricks. Such bricks were known as ``nav-terahi'' bricks. Such bricks were very useful in the circular built-up of the wells. Sometimes wells were also built in square shapes. Waters in the wells being deep, they were not used for agriculture but sometimes, vegetables were grown around the wells.

To fetch water from the wells coloured leather \textbf{``kos''} were used. They were tied through thick rope or leather to a pulley and pulled by animals like camels, ox or donkey. The water was filled into place called \textbf{``avada''}. The end that was put into the well was called \textbf{``saaran''}. It used to be approximately as long as the depth of the well itself. Two people used to operate the kos. One the person who orders the camel to pull (called ``khilio-khilivaro'') and the person who held the kos. While the camel pulled the pulley, the person holding the kos used to shout \textbf{``hau hau pachcha, mel pachcha''}. On hearing this the khilio used to stop the camel and pulled out the \textbf{nail} between the rope and kos in order to release the strain on the kos. This resulted into water being flown into the avada. The \textit{paaniharis} used to fetch water from here and the unused water used to go into drain. One such turn was called \textit{``vaaro kaadhyo''} and the people used to fetch water and operate the kos belonged to the Maali community.

Around the Gadhado town where waters were deep, two camels were employed to fetch water. When one camel reaches half way the nail was pulled out and the saran was re-tied to another camel and it used to pull the rest of the way. Thus, the length of saaran was half of the depth of the well.

Some \textbf{``vaishnu''} (Vaishnav) who have dislike for leather used canvas bags called \textbf{``chalsi''} for kos and cotton or \textbf{``akolia''} (cotton-like rope made up from ``aaklo'' plant's pulp) ropes for fetching the water. They used to do it themselves and the water was called \textbf{``bhrahma jal''}.

While the kos is worked, the women of village came to fetch water. Women put the pot of water on their head and used \textbf{``Sindhuni''} to support the pot. They used to put different types of pots (\textbf{``gaggar-morio''}) on sindhuni. These sindhuni's were decorated with mirrorwork, beads and beautiful embroiderries. The behind of the sindhuni had its decorated tail called \textbf{``chhugo''} or \textbf{``chhedo''}. Parents used to gift sindhuni to their married daughters as ``dahej'' (dowry). Poor people used earthen pots. Rich men's ladies did not go to the well to fetch water. They ordered the \textbf{``pakhal''}. The leather pakhal could carry eight pots of water that used to be brought by people called ``pakhali''. The water was emptied in the household pots and the remaining water was put into \textbf{``hodi''} or cement tanks.

Between Chhachhro and Gadhado, there were shallow wells which were called \textbf{``veri''} or \textbf{``par''}. The depth of water in such veris depended upon the rains. Normally, water was found at the depth of 5 to 15 puras. Like tadha's, such veris and pars were known by the names of people who built it. For example, jesse-ro-par, khime-ro-par, waghe-ri-veri, kumbhe-ri-veri, etc.. Some women also had built such tadhas like rupi-ri-veri, maanbaai-ro-tadho. In Mitthi, girls school teacher coming from Saurashtra were called baaisaaheb. She also built one tadho and it was called baaisaaheb-ro-tadho. The taste of water of such veris was similar to \textbf{palar} water.

In some villages, big tanks called \textbf{hod} were built for drinking or water for livestock. Some big tanks were built to store palar water also.

If some bucket or pot fell into a deep well, it was searched using the reflection of mirror (called \textbf{mirio}). The lost item was pulled out using \textbf{``billi''}, a hooked device made of iron and tied to a long rope. Sometimes the maali himself, used to tie himself to rope and went into the well to fetch the thing.

In monsoon, water used to get logged into small lakes in villages and was used for livestock and washing clothes. Such lakes were called \textbf{``tarai''} in Thar. Many towns had such water and its storage capacity was measured in terms of how many months the water will remain.

\begin{center}
% use packages: array
\begin{tabular}{l|l|l}
\hline
\textbf{Name of the Town} & \textbf{Name of Tarai} & \textbf{Month Capacity} \\
\hline
Mithi & nandhi, wadi tarai & 3-4 months \\ 
Chelhar & ranasar ri tarai & 5-6 months \\ 
Chelhar & Chhichhi ri tarai & 3-4 months \\ 
gadhado & pandhiyari ri tarai & 6 months \\ 
khiysar & - & 4 months \\ 
chhachhro & 2 mile dur tarai & 3 months \\ 
\hline
\end{tabular}
\end{center}

The depth of water in lakes was measures in terms of \textbf{``gode jitto''} (upto knees), \textbf{``chel jitto''} (upto waist), \textbf{``kulhe jitto''} (upto shoulders), \textbf{``mathode taar''} (a man would drown), \textbf{``othi bod''} (a camel would drown - 2-3 mathoda).

Lakes built by people around Gadhado and Bagal were called \textbf{``Garua''}. Rainwater was collected in such garuas. Such garuas were also known by the people who built them. Such as Manakia-ro-garuo, lalania-ro-garuo, vahua-ro-garo etc.. Garua's water was very sweet. There was a \textbf{``chhipo''} lake in Chhachhro where boys and men used to go for bath.

Because of problems associated with depth of well and water fetching, girl's parents hesitated to marry their daughters to such villages. Even in folk songs daughters tell their parents not to marry them in villages where the wells are deep.

In Maheshwaris, unmarried girls never used to go to fetch water. When in-laws sent married woman to fetch water for the first time, they decorated the pot with white paint (called \textbf{``sehdi''}) and used to make rec colored swastikas on the pots. Good decorated sindhunis were given and auspicious time was chosen to send the daughter-in-law to fetch water. This was called \textbf{``vahuari na pani uthiyari''}.

Wells gave water and hence prosperity to people but at the same time some unfortunate man or woman fed up of life used to jump into these wells (\textbf{``tipo deita''}) and commit suicide. Such wells were little used afterwards.
\section{Food}
It is a matter of pride that having been arrived from Marwar centuries ago and living amongst various communities in different region with the non-vegetarian eating habits, Maheshwari community practiced vegetarianism.

Ironsmiths, ``Khatri'', ``Maali'', ``Bajeer'', Goldsmiths, ``Meghwaal'', ``Bheel'', ``Koli'' and Islamic people had always been non-vegtarians but when Saaraswat brahmins Shuddh started consuming non-vegetarian food, at that time only brahmins of Pushkar, Maheshwaris, Bhojak and Shreemali brahmins stayed strict vegetarians.

Jains used to live in Nagarparkar. Except there at none of the places in Thar had Jains(Oswal) habited. None of the Maheshwaris lived in Nagarparkar. This prevented the influence of Jainism on Maheshwari community and the brahmins of Pushkar along with the Maheshwaris considered garlic and onion as non-consumable. The reason behind this could be that these communities being the followers of Vishnuism had limitation in the consumption of such Tamasik food.

Securing food by farming on their own fields, consuming milk, curd, butter milk, ghee from their own cattles(cows and buffaloes) in enough quantities, these people used to survive on simple but nutritious food. Almost at each Maheshwari's house, there used to be milk-producing livestock. They used to own one or more cows. Some Maheshwaris used to keep buffaloes too apart from cows. People of other communities used to keep goats.

To produce flour of grains, every household possessed flour mill using which woman used to grind the grains by themselves. While grinding the flour of Pearl millets (\textbf{bajri}), eating the fresh flour termed as \textbf{Baat} stuck to the \textbf{pulley} of flour mill used to give an immense pleasure. The women of the house used to have a good knowledge of all the parts of flour mill such as the \textbf{pulley}, \textbf{Makdi}, \textbf{kheel}, \textbf{kar}, \textbf{patli} etc.

The coarse, fine or medium texture of flour could be obtained by adjusting the height of nail. In 1944-45 at Mithi, Bhagchand Lohana installed a flour mill using diesel as fuel but none of the Maheshwaris used to go to his mill to grind their grains. In case of increased need of flour, two women used to sit face-to-face and grind the flour together using the flour mill.

With the help of a small flour mill also called as ``ghantulo", whole green grams used to be grinded to produce lentils and further grinded and peeled to produce \textbf{kormo} which was soaked in water and then kneaded in the wheat flour with spices to prepare special and delicious chapatis commonly known as \textbf{tikli}.

\textbf{The staple food of Maheshwaris:} For the breakfast, kids used to eat the roti of bajri prepared a night before with curd. In winter, the bajri rotis were warmed on coal-stove (angithi) and then crushed into pieces to mix with curd. Home made butter used to be spread on the roti of bajri which was called as Makhan-chakki and offered as breakfast to little kids. Adults did not use to eat breakfast.

Before eating their lunch, women used to offer first part of their meal to the fire. Separate chapatis for cows and dogs were prepared.(At the dawn, feed/grains for birds used to be sprinkled at the places meant for it and these places were called as hola-ro-chowk.

The lunch usually consisted of roti of bajri and thick chapatis of wheat flour, \textbf{rabdi} (curry of gram flour and curd), cooked vegetables, curd and buttermilk etc. \textbf{Khichdi} (boiled rice and lentil mix), rabdi and roti of bajri used to be the dishes for dinner. To prepare khichdi, rice and green grams(green lentils with peels) were mixed and then cooked. (As rice grew costlier, sometimes the proportion of rice in khichdi was kept less than that of green grams). In warm ghee, first cumin seeds and then the blend of buttermilk and gram flour were added to boil in medium fire to prepare the rabdi. It got cooked very quickly.

\textbf{``Lentil and rabdi had a fight, lentils consider itself superior, In less time rabdi gets cooked, never gets less in quantity."}
\section{Clothing}
There is an old proverb in Gujarati ``desh tevo vesh". But, Maheshwaris did not do any changes to the dressing styles they brought to Thar from Marwar. There were different dresses for children, women and men. Following is their description:
\paragraph{Small Children:} \textbf{``Jhablo"} and \textbf{``potro"} (a square piece of cloth without any stitches).
\paragraph{Young Boys:} \textbf{``cholo''} (shirt), \textbf{``suthan''}, patloon (pyjama with tying thread), \textbf{dhotli}, cap, shorts, \textbf{waistcoat}, coat, in feet, leather slippers made by local cobbler. In winter, sweater, monkey cap, muffler etc. was worn.
\paragraph{Unmarried Girls:} \textbf{Puthio}, ghaghro and after some age ``odhan''. Later some girls also used to wear Frocks and Patloon. On hands, \textbf{kafur} (rubber)/aaj upto elbow, \textbf{Bilhia} or bangles ( made of ivory) which was also called \textbf{Mahiyar}.
\paragraph{Men:} \textbf{Puthio, Dhotio, Potio} (turban), \textbf{Cholo}/shirt and in winters \textbf{baggalbandi}. Elderly people put on blankets on shoulders. While studying English, students and employed people used to wear half sleeve shirts. In case of full-sleeve, they had double-cuff buttons, suits (without blazer), blazer and occassionally neck-tie and hat were worn. Socks in legs, and to keep socks inplace, an elastic belt with hooks was tied. Some employed men also used to wear Dhoti and \textbf{Patko} (a kind of turban).
Dhoti used to be worn with double \textbf{laang} (the end to be tucked behind). In case of a relative's death the dhotli used to be plated and one laang was tucked on the front instead of back. During the wedding of a men, the plates were kept untucked.
For bathing, baafta (a thick cotton) \textbf{Anguchcho} (towel) was used. Some people used dhoti to dry their body and used to wear the same dhoti. Dhoti was changed every day and used to be put in to laundry. Cloth-stitched baniyan with a deep pocket near the belly was used to keep money safe. To keep money, a long plastic bag with threads on both side to tie were used. Such plastic bags were called \textbf{Vasni}. These Vasni's were tied along the waist.
\paragraph{Married Women:} Used to wear \textbf{zabbo} and \textbf{kurti}. Used to wear ``gherdar ghaghras'' (chaniyu or heavy flared skirt). Women used clothes type like chhint, gujj, cheero, kutchchi utlus, pent, kundhi etc.. To sew flared skirts, cloth was cut into plates or hook shape and for the tying thread a ``chheen'' used to be made around the waist area. Such skirts were made of upto 200 plates sometimes. Newly wed girls used to wear skirts with a special type of hand made tying threads made up of embroidered fabric and had two coins tied at either ends. For covering head, they used a 3 feet cotton cloth which was also used to cover face (ghoonghat). There were several names for such cloth such as \textbf{laherio}, \textbf{sadahu}, \textbf{pomcho}, \textbf{divtho} etc. In winters they used warm marino (a pink shawl). On almost all these clothes, they used to put artistic clothwork such as \textbf{maakhi}, \textbf{klawat}, \textbf{goto}, \textbf{mukko}, \textbf{surmo}, \textbf{sattaar}, \textbf{tildi}, silk embroiderries, diamonds etc. In the event of a death, ladies used to put a fold of their odhan behind the head. This practice was called to do ``pachcho pallo''. In hands, they used to wear kafur (rubber) or ajj (ivory), these usually covered the whole hands from sholder to elbow and sometimes upto wrist. On feet they used to wear local jootis or ``sapatas''.
\paragraph{Widows:} Widows used to wear kanchali (a plain gown upto waist) and sleeves upto wrist. These were called ``lambiye baahe''. Used to have black or red cloth as scarves. The bangles and other things from the arms were completely removed. Elderly women used to wear ghaghras made of ``fillingiai'' hand-colored by the local khatri, ghand etc. Some cloth used to be imported from gadhado that was also used.

Boys and men's dresses such as cholo, shirts, trouser etc. were made of baafta cloth which was taken as a big piece and 3-4 pairs were sewn. These pairs used to be very durable and got better on each wash. These were normally sufficient for an entire year. They were also economical. Later, a Japanese cloth was also used which was called ``kelo''. One pair of shirt-pyjama used to cost 1 Rupee. After the arrival of ``Hirakh'' type of cloth in the markets, clothes became whiter. Malmal and popplin was also used to sew shirts. Later China made double Horse Boski clothes also got popular. This cloth was used to sew shirt as well as used as turban. For trousers, ``duff'' cloth was preferred as it was smooth, durable and whiter.

Government workers used to sew shirts with detachable collors so that the collors for durability. If the trousers worn from hips, it was a fashion to fix it with fancy patch of cloth. Trouser's sleeves were doubled. For shirts check pattern was popular and for pyjamas, belt was used. Men's dhoti (a type of loincloth) was thick, with a red border and used to be imported from Patan, which was called ``Pattani Dhoti''. The whole piece was roughly 8 yard so one dhoti was 4 yard. Later on mill produced thin cloth dhoti also became popular.

For Turbans, clothes from Jodhpur was used. Elderly men wore white turbans. Fathers of marrying couples used to wear pink turbans. Boys used to sew silky trouser of a fabric called pent. Handkerchiefs were hardly used. Bushcoats were not very popular. In the year 1946 when first RSS (Rashtriya Swayamsevak Sangh or National Volunteer Organization) branch opened in Mithi, as a uniform of swayamsevak (volunteer), khaki shorts, white shirts and warm black caps were used while attending the branch.

Taylors were there in the villages but some sewing work was also done by women at home. Boys shirts were hand sewn by them. Sewing machines were not common at homes. Coat's and short's buttons were made of embroidered thread or by covering cloth over two layers of alluminium pieces. In olden times, women used to use a weaving wheel (``arrat'') to weave cotton to make threads, make cloth and color it. This kind of cloth was called ``gharecho'' and was used to sew ghaghras. Such wheels were very large and are still seen in some homes.

Old girls and women started to make tablecloth, covers, handkerchief, scarf borders etc. using ``aar''(ankodi). Woollen sweaters were also made of ``sua'' a kind of big, thick and non-sharp needle. These sweaters had different kinds of embedded designs and patterns. Gloves and caps for kids were also made of wool at home.

Women used to wash their cloth at home or at the local lakes. For washing hair or clothes, some places had suitable soil that was used. There were no washermen in Thar. Only in Mithi there was one family of Chhipa (washerman) but Maheshwaris hardly gave their clothes for washing or ironing. Some government employed people used the services of washermen.

Women used to do embroiderry on red ``hulwaan'' cloth. After removing some threads from the cloth they used silky thread to make designs. Such embroidered clothes were called \textbf{Bokani} or \textbf{Patko} which was worn in weddings.
\section{Jewelry-Makeup}
\section{Residence}
\section{Education}
\section{Utensils}
\section{Bedding}
\section{Business and Employment}
Maheshwaris used to do business, farming and livestock raising. When they arrived in Thar, they found soil and environment similar to Marwar, so they found it suitable and continued their original business and economical activity.

Thar's Maheshwaris used to put shops in the nearby villages of their own towns. Every town and village of Thar had such shops of Maheshwaris. Only one Maheshwari used to put a shop per village. They used to sell cloth, spices, jaggery, sugar, silver jewelry, utensils etc. on such shops. In leiu of these they used to take grains, ghee, wool, ``jiroi", ``kharad", ``khatha", blankets etc. under the barter system of trade. Additionally, rural people had little cash in those days. They used to lend money on interest from ``vaniyas" (Maheshwari) on occassions like wedding-death. They had to mortgage their jewelry against money and had to pay a good amount in interest. This interest business used to be profitable to Maheshwaris. Shopowners used to stay at the village shops for 4-6 months. The family stayed back in their towns. So they managed all their household activities like cooking themselves.

In some villages where there were no such shops, Maheshwaris used to go with a big bag of stuff to sell (called ``khadiyo" with 2 sections hanged on the either side of shoulders). They used to shout in the village: \textbf{``ahe koi maai, tol vatthan vari?"} -- which means, is there any lady who would like to buy jewelry?

Some Maheshwaris had fields on the outskirts of the towns. These were given to kolis and bheels on ``haarap" for cultivation. The grain thus produced was partly given to them, partly used for domestic use and the rest was sold away. In monsoon Chibhri, kaaring, Guvar etc. was available. Milking animals like cows, buffaloes, were domisticated and used to produce enough milk, butter, curd, butter-milk, ghee etc. for the domestic usage. Ghee was traded in big guantities with Kutchtch and Gujarat. Thar's ghee was famous. Some Maheshwaris used to sell oxes. Oxes from Thar used to go to Kutchtch for sale in return of silver, wooden beds, ``gaj", ``atlas", small wooden cots (from Sankhda, District Vadodara) etc.

Maheshwaris had shops of grains, cloth, cutlery, grocery etc. shops in cities. They used to import wheat and rice from Sindh. Colored cloth was imported from Gadhada, while pattani dhoti-jota were imported from Patan. Jaggery from Sindh and sugar was imported from Sumatra.

In the headquarters of TharParkar district called Mirpurkhas, Mithi's Tejomal Jagani had a big financial establishment. Jagani family was called ``lakhpati" in those days.

Some Maheshwaris had shops in Naukot also.

Umarkot's Tikamdas Ramjimal Kakkad was the owner of many acres of land. He was a big landlord. First English education started in A.D. 1912 in Thar. First batch of metric students came out in A.D. 1919-1920. Students passing metric  in those days and after used to work as clerk's in the government's departments such as customs, courts and revenue. Some got special training and became postmasters or teachers in schools. These service class people were around 2-3\% but were better respected than others in the community. Their lifestyle was visibly different than others.
\section{Festivals}
