\chapter{Infrastructural Necessities of the Community}
\section{Dhatki (Thari) Language}
Many languages are spoken in India. Every region has a different language or should we say regions are made languagewise. Thing every person who uses to express his feelings is dialect. Inter-human relationships are different at different places and depends upon geography, business and community. And so is the dialect. Still dialect maintains the characteristics of its place of origin. How-ever one tries to hide but in the time of trouble one would send a call of distress in his own dialect.

It is said that every 12 miles the language changes. So the language at one end of a region might be considerably different than that of the other end, and sometimes it becomes even difficult to understand. Based on such languages, it is decided what part the speaker comes from. For example: In Gujarat, people from Kutchtch, Saurashtra, Mahesana, Surat etc. have distinct and identifiable accent and style of speaking.

Formal language means a language for general purposes, administration, education and social interaction. In that way, dialect is specific to a particular region but a language spans the whole country. Indian constitution has officiated several languages. After this introduction, let us see about the Thari/Dhatki language.

Thar Desert (The Great Indian Desert) is considered to be spread across South edge of Punjab to the west of Rajasthan to the Khairpur district till the south of TharParkar District upto the Great Rann of Kutchtch. Maheshwaris migrated from that region to the TharParkar region of Sindh and the dialect they spoke was so called Thari from the Thar Desert. People settled in the ``Dhat'' region called their dialect ``Dhatki''. As per the Encyclopedia Brittanica, vol. XVI, page 781:
\begin{quotation}
 DHATKI, a dialect of Rajasthani is spoken in south-eastern TharParkar District.
\end{quotation}
As per the 1931 census of India (Bombay Presidency):
\begin{quotation}
 Thari/Dhatki is regarded linguistically as a dialect of Sindhi but enumerated as a separate language in census. For this procedure, there is a clear authority as THARI is recognised in Sindh as a distinct from Sindhi and has an area of its own.
\end{quotation}

\section{Water}
\section{Food}
\section{Clothing}
\section{Jewelry-Makeup}
\section{Residence}
\section{Education}
\section{Utensils}
\section{Bedding}
\section{Business and Employment}
\section{Festivals}