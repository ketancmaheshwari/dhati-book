\chapter{Amazing Life-Journey of the Maheshwari Community}
Quitting the Kshatriya religion and accepting the pen and weighing-balance is as courageous and amazing story of the King's officers as is the story of the life-journey of the Maheshwari community.

The origin of Maheshwari community is Marwad region, but being a deserted region and as lack of enough rains they had to face droughts year after year. Such a situation made the life of Maheshwari families very difficult. In such conditions, life became a challenge for them and they decided to face it with exceptional courage. They made small groups and went out of their region in order to search for their bread-butter and employment.

So, some went to Mewad and others to Jaipur via Ajmer. Yet others went to Bikaner, crossing Jodhpur border to Pokhran, falaudi, Jaisalmer, Barmer, Sindh, Kutchch, Jamnagar etc. places.

According to some Historians, families from Jaisalmer migrated and settled in Gujarat in the 13th century A.D.. Families from Mewad went ahead towards Maharashtra via Gujarat, Jaipur families went towards Delhi, and that ofBikaner went to Calcutta crossing many many borders. People from Marwar also went to Mumbai and Maharashtra. This flow continued for a long time and many family went and settled in Bengal and current Bangladesh as well. People from Jodhpur region went towards Bihar and Assam and then to Utkal-Assam and Nagaland. Some of the Maheshwari people from the Purania district of Bihar started going to the weekly market at Viratnagar (currently Nepalgunj in Nepal) and got settled there.

One group of Jaisalmer went to Malwa in Central India and Vidarbh via east Madhya Pradesh's Gondwana (Jabalpur etc.) and another group went and settled in Uttar Pradesh's Mathura, Aligarh, Kaasgunj, Meerut and Saharanpur. Such was the journey of the community and went on to spread in a large portion of the country, providing a courageous example like other Vaishya communities of Rajasthan.

For the above mentioned expansion, apart from livelihood, self and family's safety and security was also a reason. Muslim era -- from Allauddin Khilji to Aurangzeb -- from around year 1300 A.D. till year 1700 A.D. -- 400 years and Maratha civil war were also responsible for such migrations.

Today Maheshwari community is not only settled in India but have crossed international borders as well. With today's transportation and communication facilities it is not very much surprising that people migrate to different countries but imagining how our ancestors used to protect their families of thieves-dacoits and enemies using those primitive tools will give goose-bumps to the most courageous of today's people. We can but only imagine how by foot, camel and ox-carts, they migrated to unknown regions, mixed with strange people, adapted their unknown language and customs and demonstrated great courage. Such people with their self-courage and firm determination accepted and faced all kinds of adversities and eventually reached on top successfully.

Today we are tasting sweet consequences of our ancestors holy courage. It is very satisfying that even today we remain firm in different situations and face various challenges in order to keep the name of our community high and have a bright future. Fresh instances of this courage is our maigrations during the 1947 partition and then 1971 war and migrations between 1988-92.

Currently, many Maheshwaris live outside of India of which following are main countries:
\begin{center}
% use packages: array
\begin{tabular}{ll}
America, Canada and Britain & Approx. 150 to 200 families \\ 
Nepal & Approx. 200 to 250 families \\ 
Bangladesh & Approx. 300 to 400 families \\ 
Sindh (Pakistan) & Approx. 600 to 700 families
\end{tabular}
\end{center}

In the evolution of Maheshwari community it is important to note that they started with 72 branches that have increased because of various reasons and now they are approximately 80.

Depending upon the contemporary requirements, keeping away from the fame, using wisdom, farsighted decision-making is still a lesson for the new generation as much as it is a requirement of the future.