\chapter{Content and Happy Thari}
In today's modern world, the conveniences and facilities that we enjoy were not even imagined by the people of Thar 50-60 years ago. In those days Maheshwaris (and other communities as well) lived in a very simplistic manner and their needs were also very little. There were hardly one or two millionares and only a very few people had more than a thousand Rupees in savings but were content. Wealth was obtained and accumulated with frugal lifestyle and was spent on wedding and demise occasions. Comparing those times with today's, it is crucial to understand and know what were the materialistic objects that were lacking in those times. However, some things were invented later.

\begin{center}
% use packages: array
\hspace*{-2cm}\begin{longtable}{p{5cm}|p{10cm}}
\hline
\textbf{Material objects which were done without in Thar} &
\textbf{Note} \\
\hline
(1). Electricity and electrical equipments such as bulb, tubelights, fans, bell, iron, heater, geyser, mixer, mills, radio, television, tape-recorder, washing machine, hair dryer, computer etc. &
Lamps, lanterns were used. Hand held fans were used. Water was boiled on stoves. Grains were milled at home.\\
\hline
(2). Toothbrush and paste. &
Some government employees used. Other than that all would used Neem or Baol sticks called \textit{Datun}.\\
\hline
(3). Water tap. &
Women folks would fetch water from well or lake.\\
\hline
(4). Water handpumps, dunky, bore and motors. &
Water tanks were filled with rainwater.\\
\hline
(5). Calculator. &
Students were so fluent in multiplication tables that regular calculations did not require a calculator. Calculators were not invented then.\\
\hline
(6). Ball point pen. &
Students would dip holder in ink and write. Government employees used fountain pens. Ball point pen did not enter the market.\\
\hline
(7). Xerox copy. &
Carbon papers were used. A carbon paper was put between 2-3 papers and written with pressure.\\
\hline
(8). Gas stoves, Cylinders, lighters and kerosene stoves. &
Simple stoves with wood and dried cowdung were used. Burning coals were buried in stove for reuse. If the coal are not burning then some burning coals were brought from the neighborhood. Matchsticks were used frugally. Smokers kept matchboxes.\\
\hline
(9). Milk in bottles/bags. &
Milk was obtained directly from cattles. Milk was not available for sale. Buttermilk was given away for free.\\
\hline
(10). Flour Mills. &
Women milled grains at home.\\
\hline
(11). Cooking oil. &
Oil was not used in cooking. Ghee was used.\\
\hline
(12). Tea, coffee, ice, icecream, cold drinks etc. &
There were no restaurants. Tea and coffee were not consumed at homes. There were no icecreams without ice.\\
\hline
(13). Various savory food items. &
Thick savory fried noodles (sev) and fried dumplings were available. Roasted chickpeas, pipermints, salted peanuts were given to children. Last J.B. Mangharam and Company's biscuits would be consumed. Puffed rice or mouthfreshners were not available.\\
\hline
(14). Ghee boxes. &
Hardly seen. Ghee was stored and transported in leather bags.\\
\hline
(15). Cloth bags. & 
Household grocery were brought in a pouch made up of shirts. If more things are there they would use a piece of cloth. \\
\hline
(16). School bag/Aluminium bag/water bottle &
Children would take books in cloth bag called \textit{bujki}. They drank water at school itself.\\
\hline
(17). Washing powder, soap, shampoo, etc. &
Laundry was done using clay or ordinary soap. Fuller's earth was used to wash hair. Few people used fragrant soaps. Woman would put some soap with clothes for their fragrance.\\
\hline
(18). Shirt/T-shirt. &
Simple half- or full-sleeved double-cuff shirt was worn. Readymade clothes were not used.\\ 
\hline
(19). Saree. &
9 feet sarees were used. \\
\hline
(20). Leather or rubber footwear &
Regular plain slippers were available. Few wore laced shoes. Locally made shoes were available. \\
\hline
(21). Spectacles/Goggles. &
Because of good food and environment, there were no glasses for sight correction. As an exception, 1-2 people used glasses and were criticized as ``blinds". However, elderly people wore glasses.\\
\hline
(22). Bags to store clothes. &
Clothes were stored in large clothbags. Iron trunks were used. \\
\hline
(23). Newpapers-Magazines. &
Hardly one or two newspapers would be found in high school. Nobody subscribed to newspapers or magazines personally.\\
\hline
(24). Calendar. &
Women folks remembered dates-days. If required Gor would look up the calendar.\\
\hline
(25). Bangles. &
Women wore rubber or ivory bangles. Married woman would change them every 3-4 years. \\
\hline
(26). Telephones/Mobile phones etc. &
Postcards/envelopes were sent for communication. Inland letters were not available. Telegrams were sent when absolutely required (specially to send news of someone's demise).\\
\hline
(27). Vehicles such as bicycles, scooters, motorcycles, car, bus, autorickshaw, truck etc. &
Camels and horses were used. Those who visited Sindh had seen buses. Sindh also had horsecarts. \\
\hline
(28). Train. &
There were two towns with railway stations: Chhod and Gadhado. If someone had to catch other trains then one had to go to Naukot or Kunri. People traveled by train to Jodhpur for hospitals and pilgrims such as Nathdwara, Pushkar, Haridvar.\\
\hline
(29). Bathroom-Latrine. &
Small wall was made in a corner which was used as a bathroom. Men and women would go into the forest for latrine.\\
\hline
(30). RCC houses. &
Bricks were used for walls but no RCC was used in roofs.\\
\hline
(31). Plastic made household and other things. &
There were no items made of plastic. There were no plastic bags either. \\
\hline
(32). Different types of contemporary toys. &
Wooden toys were used. Simple but durable toys from Japan and Germany were also used. \\
\hline
(33). Sofa, dining tables etc furniture. &
Some people had wooden furniture and government employee had 2-3 chairs. Wooden cots were used to sit in the yard of house.\\
\hline
(34). Plywood, fevicol. & 
Strong wooden panes were used. Nails were used to tie them. \\
\hline
(35). Steel utensils. &
Copper, brass, bronze, and German Silver utensils were used. Poor people used clay utensils. Some communities were served food in silver utensils (because of untouchability).\\
\hline
(36). Childbirth-delivery. &
Experienced midwife would come to home for childbirth. There were no maternity hospitals. There were no ceaserian section operations.\\
\hline
(37). Ambulance. &
Coffin was taken over shoulders. Sick people were taken over camels by tying special ``kajaa" on camel back.\\
\hline
(38). Roads/Streets. &
Paved, tar roads were usually 25-30 miles from towns. Small footpaths were used in order to travel through dunes from one village to other. People would walk or used camels.\\
\hline
\end{longtable}
\end{center}

Some items might be missing from this list.

We were basically the farming and business people. However, our standard of living improved over the 200-250 years of history. Marwars farmer and farmer-women were content and happy with minimum needs. Here is a couplet showing how content these farmers and farmer-ladies were:


\begin{table}
\begin{center}
% use packages: array
\hspace*{-2cm}\begin{tabular}{p{7cm}|p{7cm}}
\hline
\textbf{Farmer} & \textbf{Farmer Lady}\\
\hline
\textbf{nave manju ri khat, ke na chuve tapri, bhensadle do char ke duze baapdi, bajra handa bat, dahi me olana, itra de kirtar, fer nahin chavna.} & 
\textbf{uthe hi piyar hoy, uthe hi saasro, athuno hoe khet chuve nahin aasro, nala khet najik, jathe hal kholna, itra de kirtar, fer nahin chavna.}\\
\hline
Ohh god! Give me a nine-thread cot, non-leaking roof, 2-4 milking buffaloes, Curd and pearl millet roti. I do not wish more. &
Ohh god! Let my parent's and in-law's be in the same town. My field be in the west direction (so that sun stays in the back during morning trip to and evening trips from the field), a non-leaking roof and a small place near field to put equipments. I do not wish more.\\
\hline
\end{tabular}
\end{center}
\caption{Content people of Marwar}
\label{tbl:farm}
\end{table}

