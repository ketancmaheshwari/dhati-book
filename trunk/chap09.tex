\chapter{Society and Social System}
\chaptermark{Society}
Society is not simply a mob or a group of people. People gathered at a public
place such as bus or railway station or a park can not be said to be a society
but people gathered at an organized function or a conference or a meeting may
be called a society. So, society is not just the addition of individuals but a
system and elementary link that connects individuals.  

Society means a group of people who have a distinct and identifiable goal in
front of them. People whose hopes and aspirations are same and they have an
intention of cooperating with each other in the quest of fulfilling those
hopes and aspirations. People who are engaged in constant quest and potent
efforts towards the goals. Such a group of people is a society.

Such a society made up of brave and focused men must define and march on a path
towards their goal.

There are three stages of progress: will, planning and execution of plans. For
this purpose, all people of the society must make good use of their energy for
the self-development and the development of their fellow human beings. In such
a cause, energies will not struggle against each other but cooperate and move
forward at an ever faster speed.

When there are people with different thoughts and nature, it is inevitable that
some conflicts in thoughts and actions might occur. Therefore, any person who
is devoted to the development of the society must find a way to resolve such
conflicts and channel the energies in a fruitful cause complementing each
other. 

When there is a lack of balance in the powers that lie within a society and if
no one puts an effort to balance off these powers, then similar to how nature
causes storms when there is a imbalance among elements, there would be storm
in the society among the powers. All the social revolutions happening and
happened in the past in the society are nothing but the movements caused in the
quest to bring about the balance among the various elements in the society.
However, it still brings many wounds to the society. Therefore let us not wish
that such revolution occur but work together towards causes that avoid the
centralization and imbalance of such powers.

Just as with personal and group efforts society progresses and can become
strong and resistant towards storms, the building of such a society should be
so strong that it can provide support, safety and security to individuals.

With this background information, now let us see what are the various factors
that play key role in the social systems.
