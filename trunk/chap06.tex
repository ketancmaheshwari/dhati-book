\chapter{Rituals and Social Traditions}
\paragraph{Rituals.} Many of us believes in rituals but not many think of the
root element of a ritual. We ask ourselves, are we as much humble as
talkative? What would others say about us? We are always afraid of this but
still we never have a good opinion of others.

Now let us think about the rituals. Every ritual happened one time as a result
of people's process and people's arrangement. People means, what is society's
view, what is society's situation and what is society's intention, is important
to understand.

I, you, our relatives, neighbours, and going further, our community and
town-city and country, why do we live together? Why do we accept the rule of
living together? All theories, all beliefs are linked to each other one after
the other. Our each step is decided by a pre-established arrangement.

Discussing this ponderable question in more detail, we find that every deed of
our life has been organized into reasonable or unreasonable by a social system.
Every arrangement has been deemed reasonable by a comprehensive principle. New
arrangement is born from old arrangement. New principle takes fruition from old
one. This is the chain of society. 

For instance, when a child is born, there are some norms decided for his
upbringing. Some system is decided to train him. His marriage is put under some
limits and with regard to his progeny, some responsibilities are decided and
others are denied. These systems, these standards and these responsibilities are
deemed reasonable under some principles of life. A series of these principles
makes a kind of social chain. Each link of this chain fits perfectly with the
other link. This whole chain could be called a social view of life but a main
point is that to create this view of life, some basic points are required.
These, in the absence of a better phrase, could be called the basic elements of
social life. In English, there is a word for this: ``data". A theory can not be
established without sufficient data. But this data is not a thing of logic,
intelligence of humbleness. This data can only be the interaction of facts with
the natural laws. \textbf{These facts and unanimity can not be forever. They
have the limits of time and place. This fact is so clear that it is not required
to put any more arguments in its support.}

Based on your situation and circumstances you think of how will you prosper most
in the future and live within your means is a thing of commonsense. Similarly,
society also plans and programs its future based on the situations and
circumstances. These programs are rituals and the initial thought behind these
rituals are theories.

At some point, some one can say, every ritual might be deemed reasonable based
on some theory, afterall, our ancestors were not foolish to implement them.

Let is see the above point in more detail. Every theory needs some data. The
data available at the time of forming a theory remains forever. The theory
remains valid as long as the data remains valid. If the data changes, then,
based on natural laws and systematic depreciations on theories and opinions, the
nature of these rituals change. As the nature of above mentioned data changes,
the theories behind the data must change and by consequence, the rituals formed
from  the theories must change too, or else the rituals becomes meaningless and
baseless.

\textbf{Inspite of these, if a ritual is followed just because it is coming along from a
long time, the society becomes ritualized and a ritualized society is aimless
society which is like having a roof without walls.}

After this introductory discussion, let us see Thar's main rituals and
traditions.

\begin{table}
\begin{center}
% use packages: array
\begin{tabular}{|lll|lll|}
\hline
(1) & gold: & tuss & (12) & rupees: & visa/khodo\\
(2) & opium: & mung/rati & (13)  &sand  & fistful/handful\\
(3) &mixture to make curd: & chhanto & (14)  &grass  &bundle\\
(4) &butter: & fingers & (15) &kana  &bhakar\\
(5) &snuff: &pinchful & (16) &wood  &log\\
(6) &magat: &chugtho & (17) &melon  & pie\\
(7) &sangri/beans: &fistful & (18) &roti  & piece\\
(8) &buttermilk: &lup & (19) &laddu  &kado\\
(9) &sugar: &mouthful & (20) &ghee  & spoonful\\
(10)&wheat/barley: &gaaro & (21) &sweets  & daboro\\
(11)&dates: &satto & (22) &distance  & vikha\\
\hline
\end{tabular}
\end{center}
\caption{Measurements of different things in common usage}
\label{tbl:measure}
\end{table}


\section{Birth}
\section{Janoi (Sacred Thread)}
\section{Engagement}
\section{Marriage}
\section{Death}
\subsection{Dirges}
