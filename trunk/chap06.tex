\chapter{Rituals and Social Traditions}
\paragraph{Rituals.} Many of us believes in rituals but not many think of the
root element of a ritual. We ask ourselves, are we as much humble as
talkative? What would others say about us? We are always afraid of this but
still we never have a good opinion of others.

Now let us think about the rituals. Every ritual happened one time as a result
of people's process and people's arrangement. People means, what is society's
view, what is society's situation and what is society's intention, is important
to understand.

I, you, our relatives, neighbours, and going further, our community and
town-city and country, why do we live together? Why do we accept the rule of
living together? All theories, all beliefs are linked to each other one after
the other. Our each step is decided by a pre-established arrangement.

Discussing this ponderable question in more detail, we find that every deed of
our life has been organized into reasonable or unreasonable by a social system.
Every arrangement has been deemed reasonable by a comprehensive principle. New
arrangement is born from old arrangement. New principle takes fruition from old
one. This is the chain of society. 

For instance, when a child is born, there are some norms decided for his
upbringing. Some system is decided to train him. His marriage is put under some
limits and with regard to his progeny, some responsibilities are decided and
others are denied. These systems, these standards and these responsibilities are
deemed reasonable under some principles of life. A series of these principles
makes a kind of social chain. Each link of this chain fits perfectly with the
other link. This whole chain could be called a social view of life but a main
point is that to create this view of life, some basic points are required.
These, in the absence of a better phrase, could be called the basic elements of
social life. In English, there is a word for this: ``data". A theory can not be
established without sufficient data. But this data is not a thing of logic,
intelligence of humbleness. This data can only be the interaction of facts with
the natural laws. \textbf{These facts and unanimity can not be forever. They
have the limits of time and place. This fact is so clear that it is not required
to put any more arguments in its support.}

Based on your situation and circumstances you think of how will you prosper most
in the future and live within your means is a thing of commonsense. Similarly,
society also plans and programs its future based on the situations and
circumstances. These programs are rituals and the initial thought behind these
rituals are theories.

At some point, some one can say, every ritual might be deemed reasonable based
on some theory, afterall, our ancestors were not foolish to implement them.

Let is see the above point in more detail. Every theory needs some data. The
data available at the time of forming a theory remains forever. The theory
remains valid as long as the data remains valid. If the data changes, then,
based on natural laws and systematic depreciations on theories and opinions, the
nature of these rituals change. As the nature of above mentioned data changes,
the theories behind the data must change and by consequence, the rituals formed
from  the theories must change too, or else the rituals becomes meaningless and
baseless.

\textbf{Inspite of these, if a ritual is followed just because it is coming along from a
long time, the society becomes ritualized and a ritualized society is aimless
society which is like having a roof without walls.}

After this introductory discussion, let us see Thar's main rituals and
traditions.

\begin{table}
\begin{center}
% use packages: array
\begin{tabular}{|lll|lll|}
\hline
(1) & gold: & tuss & (12) & rupees: & visa/khodo\\
(2) & opium: & mung/rati & (13)  &sand  & fistful/handful\\
(3) &mixture to make curd: & chhanto & (14)  &grass  &bundle\\
(4) &butter: & fingers & (15) &kana  &bhakar\\
(5) &snuff: &pinchful & (16) &wood  &log\\
(6) &magat: &chugtho & (17) &melon  & pie\\
(7) &sangri/beans: &fistful & (18) &roti  & piece\\
(8) &buttermilk: &lup & (19) &laddu  &kado\\
(9) &sugar: &mouthful & (20) &ghee  & spoonful\\
(10)&wheat/barley: &gaaro & (21) &sweets  & daboro\\
(11)&dates: &satto & (22) &distance  & vikha\\
\hline
\end{tabular}
\end{center}
\caption{Measurements of different things in common usage}
\label{tbl:measure}
\end{table}

\section{Birth}
Almighty has created a system of birth in this world for the progeny in humans,
animals and plants, etc. For a systemic management of world and society, seers
and saints institutionalized marriage under which woman gets pregnant and as a
result new generation is born leading to progeny of generation. Now let us see
the prevalent rituals related to birth in Maheshwari community.

When a pregnant woman is in her seventh month of pregnancy, the \textbf{hair
washing ritual} is performed.  When the woman is at her in-laws place, the hair
washing was done on \textit{sud} day fifth, seventh, ninth or thirteenth.  At
this time \textbf{khetwar's or kshetrapal's juhar} was done. A brick is washed
and worshipped as a symbol of kshetrapal.  The girl is adorned with
\textit{kumkum chandlo}. \textit{Abil}, \textit{gulaal}, kumkum, 7 cardamoms, 7
cloves were used with a \textit{diya} light and coconut was broken.
\textit{Sawa ser} (750 grams) wheat laduo were made and was worshipped. With
another sawa ser flour, five big \textit{rotas} were made and worshipped with
powdered sugar and ghee. These rotas were eaten by the woman. Along with the
woman, her younger brother-in-law or elder brother-in-law's son used to sit to
eat. The food hence cooked must be consumed before night, and could not be left
for second day. Other things used in worshipping were given to cow. Sweets were
distributed among the relatives. If for some reasons the hair wash ceremony was
not done in seventh month, it is done in ninth month.

In Maheshwari community, since a woman's first delivery is traditionally done
at her parent's place, they are informed in advance. After the hair washing
ceremony, woman's uncle or brother came to pick her up from her in-law's place.
The in-laws used to give a large \textbf{dohti} which was called
\textbf{sadra-ri-reet}. After reaching the parent's place, this dohti was
distributed among relatives, so that people knew the woman is pregnant. After
the ritual of hair washing, the woman has to stay at home. If required to go
out, she had to come home quickly. She was not allowed to go out at late
evening or night at all.

There were no public hospitals for childbirth in Thar. There were no maternity
homes. Experienced mid-wife (who likely used to be the town's barber) assisted in
childbirth. In cases of problems, she knew of home-based solutions but in
serious cases such as if child is inverted, the child or woman would die
because of lack of doctors and hospitals.

Experienced mid-wife and some senior women would predict about the gender of the
child based on the mother's stomach and other characteristics. These
predictions used to be true most of the times. No advanced tools such as
sonography were available in those days.

Birth control devices and abortions were not prevailing in those days. As and
when the child was born, it was accepted as almighty's gift. However, if after
first daughter, second and third daughter were born, dissatisfaction spread
into family. However, daughter is born, so there will be an exchange in
marriage used to be prevailing thought. Sons were more desirable. This was
because, son was the heir of family and was the one who could perform the death
rituals of parents. There are instances of upto eleven daughters born back to
back \textbf{naronar} in the hope of a son out of which 4 died and 7 were married off.

Mid-wife periodically visited to check on the woman. She would also inform of an
approximate date of birth. Childbirth was done at home in a separate room. A
jute bed-cot was used for childbirth. After the childbirth, woman was given
warmth by burning dried cowdung under the bed-cot.

If the newborn is a boy (and that too the first) then the mid-wife would
immediately go th woman's parents to congratulate them and would receive some
prize in addition to regular fees.

In the event of the birth of a boy, people used to beat steel plates with
rolling pin (\textbf{thali vagadvi}) to announce in the neighborhood of the
birth of the boy (if some family had only girls born, people used to taunt:
\textbf{taye ghare thaliye kon vagi ahe}). Woman's in-laws were informed by
telegram or letter. Some elder person from family would go to a holyman
(\textit{maharaj}) and informed the time of birth and asked for child's birth
planetary positions. If there is some troublesome planetary positions, then
rituals were done as per the advise of maharaj. Oil was donated in bronze bowl.

On the occassion of son's birth, relatives used to visit to congratulate the
family. Grandfather offered party (\textit{rihan}) and gave a rupee and coconut
to guests. If no rihan was offered, still coconut and rupee was given.
Celebratory songs were sung at grandfather's place which were called
\textbf{lara}. Such laras were sung for the newborn son. Child's paternal uncle
and aunt also organized singing of lara one day each. Dates and sugar candies
were given to the singing women.

On the sixth day of birth, child's maternal aunt wrote the child's
\textbf{chhatthi}. It was drawn on the wall of mother's room using colored
lines and boxes. Aunt lovingly used to bring clothes and gold ring for the
child while the parents would give some more valuable gift in return. In the
night, a blank paper and pen was kept near child's head so that the god of
destiny could write child's destiny.

After eight to ten days, at an auspicious time, the woman would wash her hair
which was called \textbf{melo matho dhoyo}. Until then, she would not go out of
her room. The child is given a name on this day. The name was given based on
the birth \textit{raashi's} birth-letters. As far as possible name other than
according to raashi were not given. If the name is different than birth raashi,
that name was called \textbf{arak naam}. Maharaj would perform a holy ritual
with holy-fire and would tell the child's name with a blow in its ear. This was
called \textbf{gur funk}. Maharaj was rewarded with appropriate remuneration
\textbf{daxina}.

If the child is born under one of the four lunar constellations
(\textbf{Ashlesha}(Hydrae), \textbf{Mula}(V Scorpionis),
\textbf{Magha}(Regulus) and \textbf{Jyeshtha}(T Scorpionis)) then after 27
days, constellation ritual was performed (\textbf{nakhtar naaita}). At that
time twenty-seven well's water, 27 tree's wood, 27 sea shells, betelnuts,
grapes, drumsticks, copper coins etc. were gathered. Maharaj performed holy
fire ritual. Woman and her husband with the child in lap sit together in the
ritual with heads covered and relatives put color on them. Maharaj would give
name based on the lunar constellation such as Mulchand, Jethanand, Magharam
etc. Twenty-seven children were offered meals.

After the child-birth, woman of the town would visit the mother and child and
would bring sugar nuggets (\textbf{sangan misri}). This visit was called
\textbf{bolan aayi}.

At the time of child's birth, nectar of a green leafy vegetable
(\textbf{maido}, tandalja) and jaggery was given as \textbf{suti} or first
food. Woman was given sweets with spices such as dried ginger and sweets with
mussels cumin (\textbf{churi-ro-wato}). Such sweets were called \textbf{goli}.
At the time of son's name giving ceremony, sweet balls of jaggery were made and
distributed. These sweet balls were called \textbf{baruo}. Other sweets were
also distributed among relatives.

Woman after birth was taken care of by the mid-wife. She would wash her's and
the child's clothes for one month or more. She would give oil-massage to both
and would tie the hands and legs of the child with its body properly which was
called \textbf{tanjyo}. This \textbf{tanjan} continued till six months. For
child's diaper, a square cotton cloth was doubled and folded in a triangle.
This diaper was called \textbf{potro}. When the mid-wife left her duties, she
would get money and mother's old clothes.

After a month, woman would wash her head again. This was called \textbf{achcho
matho dhoyo}. If the child died of some reason, then, the head was washed after
twenty-one days. At this time child and son were fed with water from the
Ganges. After that the woman would take over the household work and would work
in kitchen. Feeding mother would take special care of her own diet. She would
not eat sour, spicy and difficult to digest food so that child remains away
from problems. special care was taken in winters and summers. Child was
exclusively fed with mother's milk. External milk was not given. Even after the
child starts eating food, mother would feed her milk for about one and a half
years. Often mothers would give little opium to child so that it sleeps and
would be convenient for mother to finish chores.

After achcho matho dhoyo woman's husband would come to pick her up from her
parent's place. At this time parents would do a ritual called
\textbf{suat-ri-reet}. Woman would get 10 pairs (\textbf{mudd}) of clothes and
child would get 18-20 pair of clothes and golden and silver ornaments. Child
would get a cradle (\textbf{pingho}) or a small bed-cot (\textbf{michli}).
Mostly the first delivery happened at the parent's place. Second, third and
successive deliveries happened at the in-law's place and parent's would give
two pairs of clothes to mother and son each.

Child's \textbf{khetwar juhar} (kshetrapal's ritual) after birth was done in
the forest. On first Deewali festival of child, its paternal aunt and maternal
grandparents would prepare clay pots and would give sweets and two pairs of
clothes. They would bring this with celebratory drum beating. Similarly, on
first Holi festival, they would bring sweets-clothes. This was called
\textbf{dhundhaniyu}.

If the first son is died and a son is born after that, he was called
\textbf{miyachchiwalo} and his nose was pierced. If a son is born after three
daughters, it was called \textbf{tipokar}. For such children, an additional
sister was named from Maheshwari or Brahmin community so that he has four
sisters. If many children died during childhood, a new born was given clothes
from neighbors and given odd names such as \textit{luno, mirchu, bhugdo} etc.

At home, the father would not talk to or hold the child for 6-8 months as
decorum. Other family members would play with the child lovingly and
enthusiastically.

After three-four years when \textit{bhat} visited, he would record the names
and dates of birth of new born children in the family. If a first son was born
he would get a nice gift (\textbf{sheekh}), sometimes a golden ornament.

\section{Janoi (Sacred Thread)}
Brahmans would certainly weat Janoi in Thar. Earlier, Maheshwaris would also
wear but later on that system decayed. Only before getting married, maharaj
would put the Janoi on which was taken off and put on the basil plant after 2-3
days.

In reality Janoi means a second birth of a child because of which Brahmans are
called \textbf{dvij}. But when Maheshwaris worn Janoi (which was worn over left
shoulder and would go to the right and behind), it was used to tie keys. A man
with Janoi would put the thread over his left ear when he goes to toilet and
would remove when done.

\section{Engagement}
Engagement is called \textbf{veenhaan} in Maheshwari community. These
engagements were usually done at a small age. Engagements could be done in any
bloodline (\textbf{nukh}) other than ones own paternal. (Marwadi Maheshwaris
avoids maternal bloodline in addition to paternal ones). Where there is a close
blood relations, such as uncles and cousins, engagements are not possible.

Owing to a skewed male:female ratio in Maheshwari community, there is often a
scarcity of girls for engagement and thus often boy's engagements were
difficult to do.
The male/female ratio of Tharparker district is shown below: This information has been taken from the Census of India 1931, Vol. VIII, part 1, Bombay Presidency. For every thousand people the male and female population was as follows:

\begin{table}
\begin{center}
% use packages: array
\begin{tabular}{|l|ll|ll|ll|ll|ll|ll|}
\hline
\multirow{2}{*}\textbf{Year/Area} & \multicolumn{2}{|c|}{\textbf{1881}} & \multicolumn{2}{|c|}{\textbf{1891}} & \multicolumn{2}{|c|}{\textbf{1901}} & \multicolumn{2}{|c|}{\textbf{1911}} & \multicolumn{2}{|c|}{\textbf{1921}} & \multicolumn{2}{|c|}{\textbf{1931}} \\
%\hline
&male&fem&male&fem&male&fem&male&fem&male&fem&male&fem\\
\hline
Sindh&545.5&454.5&546.2&453.8&548.7&451.3&552.0&448.0&560.2&439.8&561.1&438.9\\
Gujarat&515.0&485.0&514.1&485.9&511.4&488.6&518.6&481.4&522.3&477.7&524.9&475.1\\
\hline
\end{tabular}
\end{center}
\caption{Male/Female ratio in Sindh and Gujarat between 1881 and 1931}
\label{tbl:malefemaleratio}
\end{table}
On taking averages, it is found that in Sindh, the ratio was male:female = 552.3:447.7 and in Gujarat it was male:female = 517.7:482.3. Taking into account the numbers from six decades, it seems the number of females have been decreasing in Sindh steadily.

Because of this, the system of \textit{sato}, might have become popular from early on.

\section{Marriage}

\section{Death}

\subsection{Dirges}
