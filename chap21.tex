\chapter{Superstitions and Untouchability}
\section{Superstitions}
Rural people believed in many superstitions. They would always look for
auspicious time before going anywhere. If someone has fallen ill, they would
immediately think about some superstitious cause. Maheshwaris were also
influenced by this.

If someone sneezes or if a cat crosses over, it was considered as a bad omen.
Man's right and woman's left eye twitch indicated something good is going to
happen. If the other eye twitched, then it was considered a bad omen. If there
is an itch on right palm, it was considered an omen to gain money. If a
footwear gets over other, it was considered an omen for travel.

If new clothes are stitched then they were not wore on a Tuesday. On other
days, they were first put on a water pot before wearing. Shaving could not be
done on a Tuesday or a Saturday. Clipped nails had to be wrapped in a paper
before throwing away. If a boy grind teeth in sleep then it is good but bad if
it is a girl. Shewing one's fingernails was bad omen.

Some men-women were considered to have an evil eye and if they see something or
someone then it was considered that they have cast an evil eye. If such a
person passes from nearby then some sand from their footprint's left heel was
taken and was ritually revolved around the head of a child though to have
affected by evil eye.

If a child is sick then some salt was ritually rotated seven times around his
head and thrown away near crossroads or in the stove. During this ritual the
person doing this must remain silent.

After readying child with bathing and dressing, a black dot with the
\textbf{kajal} was put over the forehead so as to protect it from an evil eye.
A child was not kept in open during the dusk because it was believed that some
evil beings are passing through the skies at that time. That evil being, called
\textbf{uparli} would cause the child to grow weak. This was called the child
being shadowed.

In the afternoon, during the dusk and on the day of \textbf{kaali chaudash},
young woman would not go outside. This is because evil beings can control her
at this time. If this is suspect then a wheat roti with oil is made and put at
a road fork. Some rituals would also be done by holymen.

If a child is suffering from whooping cough then a coin from someone's
\textbf{vaikunthi} procession was tied in the child's neck. 

If someone's children are dying frequently, then they would loan clothes from
others for their newly born and would name them some unpleasant names such as
\textbf{luno}, \textbf{mirchu}, \textbf{bhugdo}, etc.

If a donkey hee-haws in street or a dog is howling in the night, it was a bad
omen.

If a crow is cawing on roof top, some guests would be arriving.

In monsoons, if there is no rains and if a sparrow is wallowing in sand then it
will rain soon.

In small diseases, Brahmins, muslim holymen, etc would be approached for
threads, lockets, water with mantras etc.

If a boy child had died and another boy is born then his nose was pierced. If a
boy child is born after three girls, he was called \textbf{tipokar} and a
special ritual was done.

If traveling, only some days of the week were considered auspicious to go
towards particular directions. If one wants to go to other directions, the
special ritual were done with honey, betel nuts, money etc put at some distance
in that direction the previous day. Married daughter would never go to her
in-law's place on a Wednesday. It was said that ``\textit{budh beti, kadi na
bheti}", meaning "a daughter let go on Wednesday would never meet again".

Presence of a widow was considered inauspicious during a wedding or other
auspicious occasion. If someone is going to another town for some important
business, and if he meets a cow, a girl or a woman with pots full of water, it
was considered to be a good omen.

Home was not swept after someone has left home or during dusk. No monetary
transactions were done during dusk. Newborn boy's sixth day ceremony was not
done on a Wednesday. If someone has passed away, people would not go on a
Wednesday or a Sunday for the first time.

If someone has inadvertantly caused the death of a cat, he had to make a golden
token cat and donate it at the Narayan Sarovar temple. If someone is sleeping
no one would step over him/her.

If something is lost or someone is anticipated to come back from other place,
then special ritual with lines and numbers was done to speculate. If a single
line appears in the end then a favorable outcome was speculated and unfavorable
if a double line appears in the end. 

Maheshwaris too believed in many such superstitions. Women folks would believe
more than the men folks.

\section{Untouchability}
There are four main castes in Hinduism. Out of these,
Shudras were considered the lowest caste. Other caste people would stay away
from Shudras. They would be careful not to touch them or won't let their shadow
fall on them. 

In Thar, if Maheshwaris and/or Brahmins would touch people from communities
such as Meghwal, Bhil, Koli, Bajir, Bhungi etc. then they would go home and
take bath or dip some gold in water and sprinkle it on themselves. 

If a Maheshwari invites someone from non-vegetarian community such as Luhana,
Muslim, Khatri, etc, then the meal was served in silver or ceramic vessels.
Silver was considered unaffected by anybody's touch. If brass vessel was used
then it was washed with hot charcoals.

If someone from other community is to be given water, the vessel was held by
the host and he would pour the water over other person's palm and the other
person would drink it from the palm. The vessel was kept at a height so that
there is no droplet touching the vessel. If buttermilk is offered, it was
poured in the receiver's vessel from a distance above.

If someone has touched a bone by mistake, they had to take a shower or had to
sprinkle gold dipped water.

After cremation, the participating person had to take bath without removing
clothes.

A woman in her menstrual period could not touch anyone in the household. She
could not touch anything in the kitchen and anything related to water. After
two days, she would wash her head following which she could work in the
kitchen.

