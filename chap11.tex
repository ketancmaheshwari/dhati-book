\chapter{Assembly}
Small conflicts among Maheshwaris were resolved by community's mukhis. If more
than one village/town are involved then the \textit{panchayats} from these
towns were met and resolved the issues. Additionally, sometimes the whole
Maheshwari community was invited into an assembly. There, a discussion was held
on complains about weddings, relations etc. 4-5 people from each town would
attend. So, 25-30 people would assemble. Dhat community's president would
decide the date and venue and declare it among the towns. All Maheshwaris were
allowed to attend these assemblies. Many complains were discussed there and if
some changes are required in rules and regulations, those changes were brought
about. Guilty people were thrown out of the community and/or fined. Even a
one-rupee fine from the community was considered very bad/insulting. Such
families who have been thrown out of the community were not invited to any
occasions and they were generally boycotted which was considered a very bad
thing for the family. 

First assembly in Dhat was held in the Mithi town whose chair was Tando
Allahyaar's \textit{sheth} Lagharam Zaamandas Maalpani.

Second assembly was held in Gadhdado where \textit{sheth} Mathradas Pragchand
was the chair.

Chelhar's famous assembly was organized in the year 1940. There, the chair was
Umarkot's \textit{sheth} Punamchand Kevalram Karmani and Chhachhro's Mr.
Motiram Valjiram Munhta was the minister. In this assembly, Surat's 13th All
India Maheshwari Convention's chair Mr. Ramkrushna Dhyut (From Hyderabad,
Deccan) was invited. Two lawyers from Mithi, Mr. Jethanand Leelaram Ramvani and
Mr. Vaghjimal Gunesmal Jagani went to Surat to invite him. With them, Mr.
Dhyut, with Ms. Gyaanidevi Heda and one more lady also arrived. They reached
till Gadhado by railway and from there, they travelled 75-80 miles very
difficult journey by camel.

In order to communicate the systems, custom and other procedure of Dhat, the
two lawyers accompanied them on stage.

In that assembly, Dhyut's comments significantly influenced Chhachhro's
\textit{sheth} Mulshankar Khetaram. He asked her wife to remove her
\textit{baanhi} and asked her to put in bangles. It was not appropriate to do
this to a married lady with husband alive. This was subject to a lot of
criticism in the community and people would curiously come to see her.

Another person who returned from a life term after committing a murder was
thrown out of the community. He apologized with community people's shoes over
his head and a pair in his mouth. When Mr. Dhyut came to know about this, he
brought that person back into the community.

Such assemblies were held in Chhachhro and Chhod. Last assembly before
partition (between India and Pakistan) was held in Kaantyo town in A.D. 1944-45
where Umarkot's \textit{sheth} Punamchand Kevalram Karmani was the president.

In those times, the unity in community, respect for elderly and leaders and
obeying of community's orders which are non-existent in these times. There were
many Maheshwari assemblies in India after that but the results are not seen.
