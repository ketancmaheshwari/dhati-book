\chapter{Rihan}
When there is a happy occasion in one's home, such as birth of a boy, boy's
engagement or marriage, passing away of an elderly person, \textit{Akhatreej}
festival etc. Maheshwaris would organize \textit{Rihan} (a party). Maheshwaris
used opium freely during rihans. Fennelseeds, betelnuts, sugar, cigarettes were
offered alongside. There were some people specialized in cutting betelnuts with
betelnut cutter. Betel nut cutters from Kutch and Jamanagar's brass cutters
were popular. Special care was taken that no one is missed in the Rihan and if
so they would be fetched from their home with an invitation.

In preparations of Rihan, small and large carpets made of cotton, wool, etc
were spread out in the Ottak. They would bring large plate, water pot, bowls,
cotton etc. Opium wrapped in muslin cloth was brought and was dissolved in some
water. A thick paste resulted which was called \textit{kasumbo}.

Then, if it was a Monday, then, some cotton was wrapped in a match-stick, it
was held up in a corner and some kasumbo was offered as a token to Lord Shiva.
On other days, some kasumbo was sprinkled with index finger.

Kasumbo was taken from the bowl into a cotton plug and was squeezed on palm and
the first one was offered to holymen after saying \textbf{pehli hathadi guran
ri}. After this, it was offered to other people participating in the rihan.
After some friendly negotiations, some one would accept it in their palm and
would sip it taking palm to mouth. This way, after sipping 3-5 palms of
kasumbo, he would drink water followed by some sugar. This was called
\textbf{Thungo}. The first person would offer the kasumbo to next person in his
palm. Similarly, sometimes it was offered from palm to palm. Sometimes people
would engage in extended friendly arguments over who should have the kasumbo
first and how it is too much for them.

Following is an imaginary dialog in dhatki (with approximate translation in
English) between people involved in such an extended friendly argument:

* \textbf{hey seth hathali dyo -}

(now sir give me your palm -)

- \textbf{ae to jado galgalto amal ahe, kihnk thoro karo}

(this is very strong extract of opium, do a bit less)

* \textbf{iye mi ki thoro karan mi age thoro ghaty ahe}

(how can I do less in this, it is less already)

- \textbf{hu age gharaun le aayo haan}

(I already had some at home)

* \textbf{mataji ro soons, he na mu kaho}

(god swear, now please do not say no)

- \textbf{mana maran puthya thay aho ki seth}

(do you want to kill me or what sir)

* \textbf{marin avhanja dusman, mahnjo hath pachho varso ki}

(may your enemies be killed, don't return my hand)

- \textbf{beli mahnjo kink kahyo karo, ito sahe ma dyo}

(pray sir please listen to me, not this much)

* \textbf{avhin ito age pya lyo ta. he sanchai kaho ta ka jane}

(you always take this much, now tell the truth)

- \textbf{hu avhahun dikro thyan jo piyan to}

(I be your son if I take so much)

* \textbf{avhana na piyala to hu avhahu dikro thyan}

(I be your son if I do not let you take this much)

When things go too far that it becomes the question of one's honor, someone
will interfere and will give their palm to both parties. This kind of friendly
insistence is called \textbf{manohaar}. If someone returns someone's palm, it
was considered an insult.

People who are addicted to opium will only have it with other people to get
best effect (get high). Such addicts would carry a small balance, and weights
such as silver coins and seeds and would take opium in specific weights. Some
addicts would take as much as 5g in one sitting.

In 1932, \textit{Sheth Shri} Tulsidas Karamchand Akhani wrote the following
mantra in order to send a message to the society to get rid of the bad habits
of opium consumption:

\textbf{have karvi chhe amal ni vaat,}

(now talk about opium)

\textbf{rupiya lutave roj na saat.}

(which causes loss of 7 rupees everyday)

Meaning at the rate of about six annas per 10g, about 180 grams of opium was
used which used to cost Rupees 7 in those days which was considered to be a big
amount. At the time of this writing the cost of opium would be more than 1000
Rupees per 10g.

Cities had licensed shops which used to sell opium.

When the rihan ended, and people started to leave, the host would honorably
recognize this by saying "sheth, uthiyo ta".

\begin{framed}
\begin{center}\textbf{``Dhati Dhoro"}\end{center}
Here is a story of how self-respecting the Maheshwaris were in those days:

Some Dhati Maheshwari was going towards Jaiselmer to sell ghee loaded on top of
camels. On the way near a \textit{dhora}, he saw a huge container to cook. He
stopped by and curiously asked about it. The local Maheshwaris taunted him
saying would he be doing a party for the community or what? On hearing this, he
took permission of the local community, unloaded all the ghee from the back of
the camels, cooked sweets in the container and offered a party to the local
community. Because of this event, the dhora was named as ``Dhati Dhora".

\end{framed}
