\chapter{Infrastructural Necessities of the Community}
\section{Dhatki (Thari) Language}
Many languages are spoken in India. Every region has a different language or should we say regions are made languagewise. Thing every person who uses to express his feelings is dialect. Inter-human relationships are different at different places and depends upon geography, business and community. And so is the dialect. Still dialect maintains the characteristics of its place of origin. How-ever one tries to hide but in the time of trouble one would send a call of distress in his own dialect.

It is said that every 12 miles the language changes. So the language at one end of a region might be considerably different than that of the other end, and sometimes it becomes even difficult to understand. Based on such languages, it is decided what part the speaker comes from. For example: In Gujarat, people from Kutchtch, Saurashtra, Mahesana, Surat etc. have distinct and identifiable accent and style of speaking.

Formal language means a language for general purposes, administration, education and social interaction. In that way, dialect is specific to a particular region but a language spans the whole country. Indian constitution has officiated several languages. After this introduction, let us see about the Thari/Dhatki language.

Thar Desert (The Great Indian Desert) is considered to be spread across South edge of Punjab to the west of Rajasthan to the Khairpur district till the south of TharParkar District upto the Great Rann of Kutchtch. Maheshwaris migrated from that region to the TharParkar region of Sindh and the dialect they spoke was so called Thari from the Thar Desert. People settled in the ``Dhat'' region called their dialect ``Dhatki''. As per the Encyclopedia Brittanica, vol. XVI, page 781:
\begin{quote}
 DHATKI, a dialect of Rajasthani is spoken in south-eastern TharParkar District.
\end{quote}
As per the 1931 census of India (Bombay Presidency):
\begin{quote}
Thari/Dhatki is regarded linguistically as a dialect of Sindhi but enumerated as a separate language in census. For this procedure, there is a clear authority as THARI is recognised in Sindh as a distinct from Sindhi and has an area of its own.
\end{quote}
George Gearson authored linguistic survey of India indicates that:
\begin{quote}
The language of TharParkar and Jaiselmer is mostly standard Marwadi. It has a mixture of Sindhi and Gujarati to a little extent only.
\end{quote}
According to Shri Bherumal Maherchand Advani Authored ``\textit{Sindhi Boli ji Tarikh}'', ``A new kind of language has been formed by a combination of Sindhi, Marwadi, and Gujarati. It is called Dhatki means language considered to be spoken in Dhat. This mixed dialect is considered an alternate to Rajasthani but is very close to Gujarati.''

According to what is indicated in the Gazetteer of Bombay Presidency, CUTCH, Feb, 1880, Chapter III, Population: Traers, page 50 \& 51, ``Maheshwaris arrived in Kutchtch approximately 500 years ago via Nagor--Thar and settled in the Abdasa Talluka. They spoke Thar-Gujarati language, used to put on turban like the Baniyas of Thar ... etc''. (Note: In the above writing, the mention of Thar is used in the sense of \textit{Greater-Thar} means the Great Indian Desert area and not the `Thar' TharParkar area where Maheshwaris settled late. In around A.D. 1300, Maheshwaris spoke ``Thar-Gujarati'' language which is likely to be a mixture of Thar's Marwari and Kutchtch's Gujarati.)

The dialect Maheshwaris brought from Marwar and the one spoken in dhat went under the influence of the dialects spoken in the surrounding region resulting in many gradual changes. These surrounding languages includes Gujarati in the east, Kutchtchi in the South Sindhi in the west and Rajasthani (Marwadi) in the North. Like this, variations of basic dialect resulted in Dhatki language.

For some years Thar was under the administration of Kutchtch state's ``political agent'', so the official language was Gujarati. This was also an influential factor on the Thari dialect.

In the school's of Thar, initially Gujarati, then Gujarati and Sindhi and later on in approximately A.D. 1940, only Sindhi was taught. Apart from that the Baniyas of Thar used to write ``Modi'' Gujarati (basic Gujarati characters without additional accents) in their books. This Gujarati was called ``Vaniki'' gujarati.

When Maheshwaris migrated from Marwar they came from Jaiselmer to Umarkot via Ratokot. After that they started living in Thar according to their convenience and started speaking Dhatki. But those who came from a different route from Jaiselmer via Sakhkhar to Sindh region, then Sahevan, Tando Allahyaar, Tando Adam, Badin etc. places or came after some time had influence of Sindhi language on their dialect. Maheshwaris living in Tando Allahyaar and Tando Adam were called ``Tandai'' and their dialect has clear influence of Sindhi. Table \ref{tbl:difftharitandaiguj} throws some light on this fact:
\begin{table}
\begin{center}
% use packages: array
\begin{tabular}{lll}
\hline
\textbf{Thari Dialect} & \textbf{Tandai Dialect} & \textbf{Gujarati} \\
\hline
Kahaan dyo & Chavan dyo & Kaheva dyo \\ 
leela gabhbhaa & aala kapda & bheena lugda \\ 
mi sambhalyo & mu budho & me sambhalyu\\
\hline
\end{tabular}
\end{center}
\caption{Differences between Thari, Tandai and Gujarati}
\label{tbl:difftharitandaiguj}
\end{table}
In the same way the dialect spoken in one end of Thar is different than that of the other end. This is shown in the table \ref{tbl:diffmithigad}.
\begin{table}
\begin{center}
% use packages: array
\begin{tabular}{lll}
\hline
\textbf{Dhatki in Mithi} & \textbf{Dhatki in Gadhado} \\
\hline
Paase mahin betho ahe & godhina betho ahe \\
puthyan aaye to & larinan aaye to \\
Dheba & Dhora\\
Tadha & Weri\\
\hline
\end{tabular}
\end{center}
\caption{Differences between Dhatki dialect as spoken in Mithi and Gadhado Villages}
\label{tbl:diffmithigad}
\end{table}
In the towns of Thar, Dhatki language was spoken by Maheshwaris, Brahmins, Bhojak, Shrimalis, Khatris, Malis, Sonaras, Rajputs (Sodha), Meghwal, Bheels, Bajeer etc.. But Lohanas and Muslims used to speak Sindhi however, they could comprehend Dhatki. In some villages, Muslims also used to speak Dhatki. Looking at these details, we can opine that: (1) Thari/Dhatki was basically spoken in Marwad which was brought by Maheshwaris and other communities during their migration. (2) Due to the influence of regional languages from all sides, their is some mixture. (3) Dhat's region that was closer to the other region's have more influence of their respective dialect. (4) School's language of teaching influenced the dialect. (5) Because of an increase in service class people, urban dialect differed from their rural counterparts.

Now let's see the technical and linguistic details of the Thari/Dhatki language:
According to Census of India-1911, Vol. 7, Bombay Presidency, page 168: Distribution of Total Population by Languages:\\
Family\hfill : Indo-European\\
Sub-Family\hfill : Aryan\\
Branch\hfill : Indian\\
Sub-Branch\hfill : Sanskritic\\
Group\hfill : North-Western\\
Language or Dialect \hfill : Thareli (Thari/Dhatki)\\
Total Population in TharParkar District = 3,95,235\\
Population Speaking Thari/Dhatki = 1,16,664\\
Male=64,794, Female=51,870\\
Total=1,16,664 ie. about 30\% of the district.
Now let us compare some Dhatki words with Sindhi and Gujarati (table \ref{tbl:words}).
\begin{table}
\begin{center}
% use packages: array
\begin{tabular}{l|l|l|l|l|l}
\hline
\textbf{Dhatki} & \textbf{Sindhi} & \textbf{Gujarati} & \textbf{Dhatki} & \textbf{Sindhi} & \textbf{Gujarati} \\
\hline
Ankh & Akh & Aankh & gaa & gaun & gaay \\
kann & kan & kaan & meens & meenh & bhains \\
nakk & nak & naak & vachhchhdo & gabho & vachchdo \\ 
dant & dandh & daant & chhoiyo & aadmi & purush \\ 
doodh & kheer & dudh & dosi & mai & stree \\ 
dahi & dahi & dahin & hek & hik & ek \\ 
makhkhan & makhan &  maakhan & bu & ba & be \\ 
gehun & kanak & ghau & tann & te & tran \\ 
mung & mund & mag & char & char & char \\ 
saag & bhaaji & shaak & panch & panj & paanch \\ 
chhah & jhan & chhas & dus & duh & dus \\ 
baap & piu & baap & meh & baarish & varsad \\ 
ma & amaa & maa & kirniyu & chhatti & chhatri \\ 
dikro & putt & dikro & kanglo & lagad & patang\\
\hline
\end{tabular}
\end{center}
\label{tbl:words}
\caption{Some words in Dhatki and their counterparts in Sindhi and Gujarati}
\end{table}

Some examples of sentences are shown in table \ref{tbl:sent}.
\begin{table}
\begin{center}
% use packages: array
\begin{tabular}{l|l|l}
\hline
\textbf{Dhatki} & \textbf{Sindhi} & \textbf{Gujarati} \\
\hline
tahjo naam ki ahe? & thunjo nalo chha aahe? & taru naam shu chhe? \\ 
maanh jo naam Mohan aahe & Mhunjo nalo Mohan aahe. & Maru naam Mohan chhe. \\ 
tu kith jaain to? & tu kithe vanji to? & tu kyan jaay chhe? \\ 
hun jaan mahin jaaun to. & maan jag me vanja tho. & hun jaanma jaun chhu. \\ 
taahje roti khaani ahe? & tokhe maani khappe? & tare jamvu chhe? \\ 
hun dhaapyal ahaan & mukhe dho aahe & hun dharai gayo chhu. \\ 
hek raja hanto. & hikdo raja ho. & ek raja hato. \\ 
ue re bu raane hante & tehnkhe ba raanyu huyu. & tene be rani hati. \\ 
hek rajkumar hanto & hikdo rajkumar ho. & ek rajkumar hato. \\ 
rajkumar vaddo thyo. & rajkumar vaddo thyo. & rajkumar moto thayo. \\ 
ooe ra lagan lya. & hunji shaadi kai, & tena lagna levana.\\
\hline
\end{tabular}
\end{center}
\label{tbl:sent}
\caption{Some sentences in Dhatki and their counterparts in Sindhi and Gujarati}
\end{table}
As seen in tables \ref{tbl:words} and \ref{tbl:sent}, the Dhatki language has been influenced by Gujarati somewhere and Sindhi elsewhere. Some dhatki words have been written in short form of Gujarati words. Means removing the `kaano' accent.

As per Thar's traditions and because of affection with each other, peopl's names were also shortened. We see some samples as presented in table \ref{tbl:names}.
\begin{table}
\begin{center}
% use packages: array
\begin{tabular}{l|l|l|l}
\hline
\textbf{Man's Full Name}  & \textbf{Shortened Name} & \textbf{Woman's Full Name} & \textbf{Shortened Name} \\
\hline
Ambaram & Ambo & Savitri & Saabi\\
Sukhdev & Sukho & Jashoda & Jassi\\
Maherchand & Mahero & Aasha & Aasi\\
Bhagwandas & Bhagu & Nirmala & Narmi\\
Hiralal & Hiro & Jaywanti & Jeti\\
Jethanand & Jetho & Draupadi & Dhuppi\\
Nandlal & Nandu & Rukshmani & Rukhi\\
\hline
\end{tabular}
\end{center}
\label{tbl:names}
\caption{Some Full Names in Dhatki and their Shortened Forms}
\end{table}
Articles appearing in Sindhi eg. jo, ja, ji and Gujarati eg. no, na, ni are replaced by marwadi style \textbf{ro}, \textbf{ra}, \textbf{ri}. For example:\\

Sindhi: hi chhatti keh ji aahe?\\
Gujarati: aa chhatri koni chhe?\\
\textbf{Dhatki: e kirniyu ke ro ahe?}\\

Dhatki have male and female gender but no neutral gender. Sindhi's `aahe' is `ahe' in Dhatki and its `tho' is `to'. Examples shown in table \ref{tbl:sent}.

There is no systematic literature available of Dhatki/Thari dialect. The language being colloquial, it transferred orally from generation to generation in the form of traditional songs, wedding songs, \textit{sawayas}, \textit{dhamalas}, \textit{shlokas}, festival songs, puzzles/riddles, proverbs etc.. These were spoken on occassions but are increasingly getting less spoken. Recently we heard that in Pakistan's Sindh state, ``The Sindhi Adabi Board'' tried to integrate, maintain and publish a collection of such sparse literature. In that publication's preface some such samples are provided. \textit{sawayas}, \textit{dhamalas} etc belong to the \textit{``pushtimargiya''} genre and so the Maheshwaris of Marwad must be belonging to that genre.

To include the Dhatki language in the Indian constitution, A Maheshwari Member of Parliament put forth a proposal in the Indian Parliament in A.D. 1992-93 but it was not accepted buy the parliament.

\section{Water}
\section{Food}
\section{Clothing}
\section{Jewelry-Makeup}
\section{Residence}
\section{Education}
\section{Utensils}
\section{Bedding}
\section{Business and Employment}
\section{Festivals}