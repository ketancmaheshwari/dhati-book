\chapter{Brief Geography and History of Thar}
\chaptermark{Geography and History}
South East of Sindh is the TharParkar district. Two parts of this district were
sandy and dry. Third part was flat and hence accessed by Sindhu river's canal
for irrigation. So this part was prosperous. In the East end of the sandy area
was the Karunzar Hill.

Sanskrit word for dry land is ``sthal''. From sthal to thal and eventually name
Thar emerged for this region. The ``great desert of Thar'' is mostly Thar and
TharParkar district's Dhat region and some nearby land is basically all called
``Thar''. It is pronounced as \textit{Thaar} in English which is wrong. Real
pronunciation of the word is \texttt{Thar}.

In the east of Thar desert is the Aravalli mountain range, in the west the
Khairpur kingdom followed by the prosperous region. In South Kutchtch and in
North are the Southern Borders of Punjab-Haryana. The population of this
region is sparse because of arid land and large mounds of sand. Large cities
are less and villages are far apart from each other.

Many centuries ago, there was sea in some parts of Rajasthan and the Desert of
Kutchtch. This sea was connected to the cambay. In those days one of the
branches of the Sindhu river, called ``Haakdo Nadi'' emerging from Punjab's
\textit{Panchnad} area, flowed through Bahavalpur's east and near Umarkot
before submerging into the sea near the desert of Kutchtch. At that time there
was a big port called Parinagar and big ships used to sail along. Eventually by
the sea movement or earthquakes the Haakdo river stopped flowing and the Sindhu
river also changed its course to the west. After that the whole region became a
dry and arid desert.

To reach NagarParkar, Kutchtch desert had to be crossed (lit. ``\textit{par
kari ukarvu}''). From this phrase, the name of this town was ``parkar''.
NagarParkar is situated at the base of Karunzar Hill. ``Thar'' + ``Parkar'' =
TharParkar is the name of district.

The mounds of sand were called ``\textbf{Dheba}'' or ``\textbf{Bhitt}'' in
Thar. The side that gets sunshine at the time of sunrise was called
``\textbf{Tirkol}'' and the otherside of the mound was called
``\textbf{Gochchar}''. The place where these two Bhitt meets was called
``\textbf{bukkad}''. The upper part was called ``\textbf{Mathaari}'' and the
end part is called  ``\textbf{Pochchando}''. Large plain between bhitt's was
called ``\textbf{Dohar}'', which was suitable for habitation. Villages used to
get established here, well used to get dug and farming took place in rains.
Small Bhitt's were called ``\textbf{Daro}''. This is where the names
``Mohan-Jo-Daro'' and ``Kahu-Jo-Daro'' comes from.

There were around 40 big sand mounds between Kantyo and Umarkot spanning around
20 miles. Largest mounds were in the Mithi Tehsil. Because of these mounds
there was no river in Thar and there was no vehicles with wheels moving around.

Based on its geology, Thar's area were given different names like Kha'ad,
Kantho, Parkar, Vat, Samroti, Vango, Maherano, Naro, Achchoter and Dhat.
``Dhat'' was the main central part of Thar. This included some parts of
Umarkot, Chhachhro and Mithi Tehsils. Because of this the whole Thar was known
as Dhat and the Maheshwaris there and elsewhere were known distinctly as Dhati
Maheshwaris. Some people called them ``Thari'' Maheshwaris.

Tharparkar district lies between $24^\circ-13'$ to $25^\circ-22'$ north
latitudes and $68^\circ-40'$ to $71^\circ-11'$ east longitude. Its total
surface area was 13,690 square miles out of which 8,496 square miles was arid
sandy and hilly terrain. The fertile land cover was 5,194 square miles.

In the east were the Gujarat and Rajasthan states, Sanghad district in the
North, Hydrabad district in the west and the desert of Kutchtch was in the
south direction.

The district was devided into three divisions for administrative purposes:
\begin{enumerate}
\item Mirpur-Khas Subdivision: The land was completely under irrigation here. There were railways and roads.
\item Naro Subdivision: Because of umarkot's arid and sandy region here there was no irrigation. Railway line went to Jodhpur through new Chhod and gadhado.
\item Thar Subdivision: This was completely arid and sandy terrain. In the east was the hilly region. There was no facility of irrigation. There were no roads. Agriculture completely depended upon rains.
\end{enumerate}
In Thar subdivision, Mithi, Diplo, Chhachhro and NagarParkar Tehsils were
located where a Tehsildar (Mamlatdar) was appointed. Mithi also had a Deputy
Collector and Deputy Superintendent of Police (Dy.S.P.). Until A.D. 1906 the
headquarter of the district was Umarkot which was moved to Mirpur-Khas in 1907.
Since A.D. 1992, the main TharParkar district has been modified with addition
of some more parts from the Thar Subdivision and the headquarter has been moved
to Mithi.

Let us now focus on Thar's history. Thar's ancient history is not available.
Umarkot, NagarParkar etc. were ancient cities. Parinagar was a big port.
Jainism was spread in NagarParkar and Viravah. Baudha and Jain temples were
also there. Godi's famous Jain temple was also there.

Prior to that Thar was ruled by Parmar Rajputs (a princely caste). After that
came the Sumra Rajputs. In A.D. 1125 Sodha Rajputs conquered the Ratokot area
and gradually till A.D. 1226 moved towards Parkar after conquering Umarkot. In
those days the human settlement was negligible in the Thar's Dhat area. Thieves
and Dacoits used to take shelter here.

After that Bheels arrived to live in Thar. These warrior communities owned
land, dug wells and established their own villages. Started agriculture.

From A.D. 1330 till A.D. 1439 Sumaras and then till A.D. 1609 Sodhas ruled
Thar. After that until A.D. 1736, Thar was under Sindh rulers and directly
under Delhi rule intermittently. Ocassionally Sodhas declared themselves
rulers. In A.D. 1736 when Kalhoda stormed Umarkot and acquired it, all Sodhas
spread out across the region.

Approximately between A.D. 1936 till A.D. 1755, from Marwar, Maheshwaris,
Brahmins, Sonara, Naai, Chaaran, Suthar, Maali, Koli, Bajir, Meghwaal etc.
Hindu communities, who came previously because of Sodhas, also came again in
the small villages of Thar.

From A.D. 1782 Talpur obtained Thar from Kalhodas and ruled it for 61 years.
They built a lot of forts for safety reasons. In A.D. 1843 British occupied
Sindh and then the British rule began and they established the TharParkar
district.

From A.D. 1844 till A.D. 1856, out of People's wish, Thar's some area -- Parkar
and Kantho--Balihari, Diplo, Mithi, Islamkot, Singaro, fithapur, viravah,
Adhigam, Mamchero, Bahrano, Chudio and Sakarvero etc. villages were under
Kutchtch assistant political agent who used to reside in Kutchtch-Bhuj.

These political agents sometimes in Monsoon, lived in the bungalows built in
Mithi. In place of this bundalow, later was the residence of Mr. Maheshwari
Uttamchand Khetaram Bachani (near the Muralidhar temple and opposite Dayaram's
Dharamshala). It is in ruins now.

At the time of independence of India from the British, it was decided to
partition the country. After hearing that the Sindh region will go to Pakistan,
Sindh's TharParkar district's head and known people met in  Mirpurkhas and
decided that the ``Lower-Sindh'' region where the Hindu population is more
should be merged with \textit{Hindustan}. This was opposed by the
``Upper-Sindh's'' Hindus which were relatively less in number. Still the
proposal was sent to the government. But unfortunately, before anything could
be done about this proposal, the British government already signed and stamped
the orders of partition and the prepared maps. Because of this, Thar's
Maheshwaris and Hindus decided to leave their homes and villages and come to
India through Migration (called ``Ladpalaan'' in local language).

