\chapter{Transportation}
There are large sand dunes in the arid land of Thar. Plain area between two such
dunes was called \textbf{dohar} which was suitable for human settlement. Small
villages would form in these places and wells were dug, lakes were made and
agriculture would also start.

The dunes were from North-East to South-West with the west side of them being
very steep. Such dunes were usually 500-600 feet higher than the land level. It
was hard to climb them straight so people and animal would climb in a slant
over these dunes. Because of fine sand, often the feet would get buried under
them or the sand would slip from under the feet.

Thar's Mithi, Diplo, and Chhachhro towns had dunes while the Nagar-Parker area
was hilly. The Karunjhar hill located in the area was 1000 feet high and it
covered an area of about 20 square miles. Nagar-Parker was located in the
valley of this hill.

Naukot, Kunri, NewChhod, Umarkot etc. had plain land. They received water from
Sindhu river canals which resulted in good agriculture. The area had roads and
railway lines. 

In the area with dunes where no wheeled vehicle could go, all transportation
would be with camels and horses. To go from one town to other, if the distance
is 30-40 miles, one would need to stay overnight for a few hours. To carry
water on the way, leather \textbf{deeli} (sandari) or brass pot with cloth covered
called \textbf{baadlo} were used. To eat, wheat \textit{tikli} was taken
from home.

To go from Mithi to Naukot (where there was a railway station), one had to rest
at the Vijutra village. Similarly, Jhangro was between Mithi and Nagar-Parker,
Kantyo between Chhachhro and Kunri, Dahli was between Gadhado and Nagar-Parker
in addition to other villages.

People not habitual with traveling on camelback would find even the short, few
hours journey troublesome. In order to ride camel, a \textbf{pakhdo} would put
on its back, sheets were spread and \textbf{Gaasiyo} was made over which two
people could sit back and front. The person in the front would control the
camel with the help of a rein (\textbf{mohar}). \textbf{Pagoda} were used to
support legs.

It used to be scary to new person when the camel would rise or sit.
Additionally, the thighs would start hurting in a few hours of journey.
However, thankfully, when the person would rest in the sandy dunes for a short
while, such pain would quickly go away and would become fresh.

In such a system of camel back travelling, the most difficult part was to move
someone sick to hospital. For this purpose, a tool called \textbf{kajao} was
used on which a patient would be able to sleep.

There were no streets or roads in Thar with a straight line or equal width. The
only streets were the ones formed by the feet of horses and camels. These kinds
of ways were used which were called \textbf{vadho} or \textbf{gus} or
\textbf{Dhadu}.

Businessmen would load their mercantile in \textbf{khadias} which were then
mounted atop a camel's back. If the mercantile was grains, sugar, or ghee they
would use special type of bags called \textbf{taakiya}. Sometimes things like
iron girder, almirah etc. would also be carried atop a camel's back. 

Thar had a new railway line in year 1900 which was called "Jodhpur-Bikaner
Railway". This name was changed on 1-11-1924 to "Jodhpur Railway". Maheshwari
towns such as New Chhod, Gadhado, Lilmo were some of the stations on this line.
People from Mithi, Chhachhro, Chelhar, Kantyo etc. towns would catch train from
Naukot or Kunri station.

Railway had first, second, intermediate and third class compartments. People
would travel in third class. Job class people would get intermediate class
tickets. Many Thari people had never seen a train in their lifetime.

