\chapter{Marriage Procession of the Bridegroom~(\textit{varghoda})}
\chaptermark{varghoda}
It is a custom to have a marriage procession for bridegroom during wedding which
also gives an estimate of the family's prestige. However, the modern day
marriage processions were originally called ``ghodi chadhan" (horse climbing).

After migrating from Marwar, wherever they stayed, Maheshwaris organized their
children's weddings but they were much simpler because of a painful migration.

After some time, about after a couple of generations in Thar, Maheshwaris
settled down and remembered the old style marriage procession. With some
savings and established business, they started the custom again.

Bridegroom's marriage procession was one such custom which was resumed by one
Chhatmalji from the Rathi family. In order to start a marriage procession, one
must get permission from the panchayat of the town. Then, hire some camels, tie
drums on each side of these camels and with musical sounds, take the procession
from bridegroom's town to bride's town. Offer parties and gifts to the community
families on the towns in the way and offer 30 camels to the holymen or an
equivalent amount of money.

This custom was understood by Chhatmalji after inviting knowledgeable people
from Marwar and organized first marriage procession in Dhat, of his son Dhanji.
From this anecdote, a new proverb started as follows:

\begin{quote}
\textbf{Chhiti kari Chhatmal, varghode ri vaat,\\
Rathi thari jaat, jaayo na jaapse}
\end{quote}

Narsingdas belongs to this Chhatmalji's family and their ancestry was called
Narsingra. Chhatmalji's great grandson was Ramji Shah and his ancestry was
called Ramjira.

A second marriage procession was organized by Maherchand Karmani which was from
Umarkot to Badin. This procession was also as pompous as the previous one. All
villages' Maheshwaris were offered parties, were given clothes and poets were
also given donations.

In the year 1909, Nabisar's Sitaram and Lagharam Chaudhari's two procession
reached Vastani's house on same day (presumably for the wedding of two of the
Vastani daughters).

In the year 1914, Narsinghray Jogomal took a wedding procession from Chhachhro
to Nabisar.

Sarupchand Saanjhira started a procession from Chhod with cheerful drum band.
He offered parties to 5-7 Maheshwari towns. Mithi's Mukhi Malani's home was the
destination.

In the year 1968 a procession was organized on the wedding of Vastiram
Amarchand Ramvani. His marriage was fixed with Ramchand Meghraj Baluani's
(Gigal) sister. After offering party to the community, 1.25 KG sugar, 2.25 KG
dried dates and 1 rupee cash per family was distributed in the community.

Serving Brahmins were offered 18 camels.

After the formation of Pakistan, in the year 1963, Murlidhar Punamchand Dedha
wedded into the family of Mithi's Gordhandas Manhardas Harani. After the
marriage, with permission of the community, holymen started a procession by
beating the drums. They roamed in  the town's temple and despite being a dhura,
offered party to the whole community. Distributed 500g magat and 4 rupees cash
per home. Serving brahmins had meals and took about 400g of magat along. 220 KG
flour was used in total for magat.

As per the best of our knowledge, the last marriage procession in Thar was
organized in the Kantyo town. After the formation of Pakistan, in the year
1965, \textit{sheth} Gordhandas Kevalram Laghad who moved from Dahli to Kaantyo
had organized a procession from his home to the home of \textit{sheth}
Kishenchand Chaudhary (Ghurya).

People whose 5-7 previous generations had organized such processions were
legible to wear gold necklace and a wrist bracelet. If worn otherwise, people
would taunt him as \textbf{``taahje ke baap daade varghoda kadhya ahin, se
kanthi-kado paheryo ahe?"}. Such necklaces and bracelets were made up of solid
gold and could weigh as much as 200g each.

If there were more than one wedding procession arrived in a town on the same
day then, in order to organize first \textit{rihan} (party), a permission had
to be obtained from the community. Those who have organized such a procession
in the past would get a permission to be first to organize the rihan. Family
who have organized such a procession were considered superior in the community.
Their social transactions were also high-valued. Everyone would always remember
the family of procession organizer fondly: so and so person had organized a
wedding procession and offered meals to Mahajan-Maastaan.
\begin{framed}
\begin{center}\textbf{``Shah"}\end{center}

Maharana Pratap was the king of Mewad. He was very brave, thoughtful and
patriotic. In order to protect the freedom of Mewad, he went to the
battleground. However, he had to leave Mewad as his resources started to run
out. 

At this time a prosperous businessman called Bhamasha devoted all his riches to
the king's service.

When Bhamasha offers all his riches to Maharana Pratap, he says, ``Oh Bhamasha !
Mewad's land is radiant because of the gems like you !! The land which beers
such courageous people will never be under siege of anyone !!! It will always
in the paradise of freedom.

``Oh Bhamasha, I do not have much to offer you in return of your loyalty and
patriotism, but as a token of gratitude, I call you ``Bhama Shah" and the title
of ``Shah" will stay for generations and would remind the coming generations of
you courage and patriotism."

This brave Bhamashah was a Maheshwari. Since then, Maheshwaris use ``Shah" as a
prefix or postfix title as a custom.

\end{framed}
