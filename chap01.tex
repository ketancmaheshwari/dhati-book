\chapter{The Origin of the Maheshwari Community}
Authentic historical literature about the origin of the Maheshwari community is not available. The sole source of information available is in the form of a book ``Itihas Kalpdrum Maheshwari Kulbhushan'' authored by late Shivkaranji Darak of Mundwa. Based on the book, following is the description of the origins of the Maheshwari community:

Suryawanshi King Khadgalsen of Chauhan dynasty was ruling over Khandelanagar state. He was very kind and just king. People lived happily and peacefully in his kingdom. He was always worried of not having a son.

One day the king had invited Brahmins and paid great honour to them. The Brahmins were very happy with the king's courtesy and asked him for a boon. The king then expressed his desire for a son. Brahmins said, ``if you worship Lord Shiva, you will be blessed with a very brave and adventurous son, but do not allow him to go towards the north and take bath in the Surya-kund there until he turns 16 years old". If the prince respects brahmins, he will become a great king else will be reborn in the same kingdom. Hence being blessed by the brahmins, the king rewarded them with fine clothes and jewelry and respectfully saw them off. The king worshipped God Shiva and was blessed with the boon.

King Khadgalsen had 24 queens. After some time, one of the queens, Champawati gave birth to a baby boy. The king was very happy and named the prince as Sujan Kunwar. The prince learned horse-riding, weapons etc. by the age 7 years. When he reached the age of 12, enemies were afraid of him. The king was quite satisfied with his work. He was careful not to let the prince go towards the North.

Once a Jain sadhu came and preached the prince about Jain religion luring him into anti-Shiva beleifs and showed the faults of brahmins. At the age of 14, the prince opposed Shiva and started practicing Jain religion. He campaigned the Jain religion in East, West and South and banished idol worshipping. He used to harass Brahmins and broke their sacred threads (\textit{janoi}). He forbade all religious activities including yajna and hawana. Out of the King's fear, he never went towards the North direction, but who can stop the destiny.

Once he went towards the North to the Suryakund with his 72 officers. He had grown angry when he saw 6 Rishis performing a yajna. He ordered his officers to destroy the yajna and harassed the rishis. Seeing this the rishis cursed them to become stone-like. So the prince including all his officers and horses became stone-like. This news spread very quickly in all directions.

The King and citizens became worried after hearing this news. King Khadgalsen died of the shock. 16 of his queens became \textit{sati} with him. With no protector of the kingdom, neighbouring enemies attacked the state. They divided the state into many regions and merged them into their own states.

Even as this happened, the prince's widow and 72 officer's widows cried and went to the rishis. They humbly requested and begged for the lives of their husbands. Seeing this rishi went soft. However, they said they are not capable enough of revert the curse. They advised the ladies to go to a nearby cave and worship God Shiva so that the curse can be taken off. All the ladies went to a cage and religiously meditated for appeasement of God Shiva.

After some time, God Shiva and Parvatiji came around the place where the prince and officers were lying stone-like. Parvatiji asked what happened and Shivji told the whole history.

At this time the prince's queen and the officer's wives fell on the feet of Parvatiji and expressed their plight. Seeing this, Parvatiji requested Shivji of taking off the curse. God taken off the curse and freed them of the stone-like state giving them a new life. Everybody fell on the God's feet.

As the prince became conscious, his mind filled with lust seeing Parvatiji's beauty. Seeing this Parvatiji cursed the prince like this: ``O bad man! you will always beg for food and your all coming generations shall beg for food''! These people were called ``jaaga'' (bhat) later on.

The 72 officers said: ``O God! Now we do not have a place to live. What should we do now''? So Shivji informed them, that they quit their Kshatriya religion in a previous birth so now they are liable to accept the Vaishya religion. Go to the Suryakund and have bath. As they bathed, their sword became pen, sword-case became stick and shields became weighing balance. All officers became Vaishya. As God Mahesh gave them lesson, they were called ``\textbf{Maheshwari}'' Vaishya.

When rishis came to know that everybody has been freed of curse, they asked God: ``O God! how will our incomplete Yajna will be completed''? Hence God preached to the officers that now onwards these rishis are your guru and you accept them as such. God told the rishis that they do not have anything as of now but when they have some occasion in their home they will give you material things to the best of their capabilities. You should teach them to follow their religion. Rishi accepted them as their pupils and each rishi accepted 12 pupils. Following is their description: 
(1) Parik from Parashar rishi (2) Dadma from Dadhichi rishi (3) Adigol from Gautam rishi (4) Khandelwal from Kharik rishi (5) Sukuwal from Sukumarg rishi (6) Saraswat brahmin/purohit from Sarasur rishi.

After some time of leaving Khandela all settled in Didwana. From these 72 officers, 72 \textit{nukhs (clans)} came into existence and from these \textit{nukhs}, depending upon the business, \textit{peta-nukhs} came into existence.

This day was the ninth day of \textit{Jeth-sud} month. This day is celebrated as \textbf{``Mahesh Navami''} in all over India by Maheshwaris. Maheshwari community is progressing continuously.
