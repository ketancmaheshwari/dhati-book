\chapter{Woman's Position in Society}
\chaptermark{Woman}
In Hindu scriptures, the caste system and phase (\textit{ashram}) system has
been patriarchal. Woman is considered as the one who follows man and is a
companion of a man. Woman does not have their separate duties but her duties
are what the man commands.

Woman would be dependent on father after birth, then under the shadows of
husband after marriage and finally dependent upon her son if she becomes a
widow. Women would normally receive education in household chores, kitchen
work, etc, from her mother. They did not have opportunities for education and
freedom similar to men.

To start, at the time of birth itself, an inferior attitude was held against
them. If son is born, family would celebrate by beating plates and offering
\textit{rihan} in the community whereas if it is a daughter, they would say a
``stone" is born. Although, there was some consolation that the girl born would
be useful at the time of boy's wedding in exchange for a bride. There used to
be discrimination in feeding, dressing, and bringing up in general.

Boy was always given importance at the time of engagement. Traditionally, for
boy's engagement, his sister was given in exchange as mentioned above. In such
a situation, it was always seen that the boy gets a good bride and his sister
was almost always sacrificed. It was never thought that select good groom and
family for girl and any bride is ok for the boy. In fact, some poor parents
would also sell their daughter in marriage for money as if she is a thing on
sale.

Marriage would take place in the childhood normally. Immediately after wedding,
the child bride would be surrounded by the veil \textit{ghunghat} system. Among
the inlaws, she had to be in veil even from her brother-in-law who is younger
than her husband but elder than her. She had to stay as much away from men as
possible. If she is wearing shoes and has to walk near men, she has to take
them away and hold in her hands while walking near by men. Among women, she had
to be in veil from her mother-in-law. She never had a chance to talk with her
husband during the day.

Because of child marriage, many women would die during childbirth. So, the
widower men would remarry a second, third (sometimes fourth) time and the newly
wed bride would be in trouble. Elderly men would die and their young wives
would become widow in a very young age. Men would be able to marry more than
once but women would not be allowed to marry again even if their husband dies
immediately after marriage. There was no custom of widow marriage.

Woman would always stay at home. Since all the financial affairs were upon men,
no woman from the gentry would be allowed to step out alone. At the in-laws
home, she had to tolerate taunts from mother-in-law and/or sister-in-law.
Mother-in-law would remember how she had to tolerate such behavior from her
mother-in-law and would vent this on her daughter-in-law. Men would go to their
`hut' shops in villages or would live in another town where they had jobs. When
they returned home, they would be tattletaled by their mother and sister
against their wives. Bitter arguments would ensue and husband would beat wife.
Domestic violence would happen often and woman would be at the receiving end.
She would often be hit with punches and kicked. In such times in-laws would ban
the daughter-in-law from going to her parent's place. After getting fed up of
such problems, woman would either immolate self or jump in well.

Now if we see the daily routine of Thar's woman, she had to wake up at a very
early hour when sun has not yet risen in the morning. She started with
household and kitchen related chores such as grinding grains, cleaning floor,
repair mud floor if required, churn curd to make butter, clean the cattle shed,
feed the cows-buffaloes, etc. If cowboy is not hired, she would milk the
cows-buffaloes herself. Then, after bathing, would fetch 3-4 or more of water
pots from the town well.
