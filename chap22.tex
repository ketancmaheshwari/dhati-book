\chapter{Bhat, Choba, Panda, Gor}
\chaptermark{Bhat}
Every community has special people who are responsible for record keeping of
the community and hence maintaining its history over time and keeping its
heritage safe and secure. Such communities are known as Charans, Bhats, Barots,
Gadhvi, etc.

Such Bhats or Jaagas are described in the evolutionary tales of the Maheshwari
community. These Bhats (aka Bhat Raja) live in Rajasthan. They record the
family ancestry and history of Maheshwaris in their record keeping books. These
books are written in unaccentuated (without any accents) Hindi language.

When Maheshwaris migrated from Marwar to Thar, at that time, Bhat rajas also
migrated and depending on convenience and hosts, they also distributed families
among themselves. The names and description of their last generation and towns
they were in charge as of latest information is as shown in table~\ref{tblbhats}


\begin{table}
\begin{center}
% use packages: array
\hspace*{-2cm}\begin{tabular}{p{5.5cm}|p{5.5cm}|p{5.5cm}}
\hline
\textbf{Name of Bhat Raja and their brothers along with their town} &
\textbf{Their Children} & \textbf{Which towns did they visit in Thar}\\
\hline
Kaluramji, Narandas, Gangadhar. Town: Nandsha Jagir, district Bhilwada, Rajasthan. &
Bhawarlal Kaluram, Chatrabhuj Bhawarlal &
Umarkot, Veralo, Sundro, Nadlasar, Banra.\\
\hline
Devilalji. Town: Singapura, district Bhilwara, Rajasthan. & Bhavanishankar Devilal, Babulal Bhawanishankar, Madanlal Devilal, Girishkumar Madanlal, Bhagwatiprasad Devilal, Kalpeshkumar Bhagwatiprasad &
Mithi, Bhorilo, Tando Allahyaar.\\
\hline
Mukundlalji &
- &
Lilmu, Parnom, Chelhar, Kantyo.\\
\hline
Jagannathji and Prabhulalji &
- &
Chhachhro, Bugul, Arabliyar.\\
\hline
Khubchand Ravji. Town: Jhandol, district Bhilwara, Rajasthan. &
Amrutlalji &
Chhod etc.\\
\hline
\end{tabular}
\end{center}
\caption{Bhats and their details}\label{tblbhats}
\end{table}
%
Bhats would visit Thar towns every 3-4 years and would stay in a
\textit{dharamshala}. Then he would visit each Maheshwari family and record any
new births and demises in the family. What kind of rituals were done after the
passed away person, eg. offered meals to \textbf{Mahajan-Maastaan}, organized
vaikunthi, donations of cows and camels, organized \textit{varghodo} in
weddings etc. would be recorded. He would also sit all the family members
together and read their ancestry and have meal with them. He would receive
``\textbf{seekh}" from the family. This ``seekh" would include cash money,
golden ring or bud or other jewelry and some kind of cloth. Every family would
give something appropriate to their financial reach and Bhat raja would
gracefully accept. If someone has the occasion of boy child birth he would get
some more.

Some Maheshwaris migrated to Kutchch from Marwar. Initially, some Bhats used to
visit them but later, because of a difficult course to reach between the
desert, they were not able to reach and many family's details could not be
recorded. Some Maheshwaris merged with the Jains of Kutchch and were called
`dasa' or `visa'. However, they have maintained their ancestry. Apart from
Kutchch, some Maheshwari families also live in Mumbai and Panvel but the
contact with Bhat Raja is broken.

After the partition of India and Pakistan, Thar's Maheshwaris live in 350-400
towns and villages of Gujarat and Rajasthan. Bhats still visit and do
record-keeping in towns where population is significant. After the partition,
Bhats have also visited Pakistan's Thar area villages and towns with
Passport-Visa. 

In addition to Bhats, for pilgrimage related services of Maheshwaris,
\textbf{Chobas/Pandas} lived in various pilgrimage places. These Chobas/Pandas
still exist and their details are described as follows:

\textbf{Mathura}: Holy men serving in Mathura are called Choba. As soon as some
pilgrim drops down at the station, 8-10 of them would surround and ask about
their hometown, ancestry etc. After knowing the right Choba for their town,
they will enable a contact with them. This Choba will help them find a place to
stay and will take them along for sight seeing and to various temples of Gokul
and Vrindavan. The visitor will offer them a suitable donation.

Earlier, one Mr. Rughnath Dasaram was the Choba. Now one Mr. Viththalnathji
Vinodkumar Chaturvedi is the Choba. They consider themselves as teerth purohit.

\textbf{Haridvar}: They are called Panda here. They also find the visitor and
take them along. Especially, in Haridvar, people come on the eleventh day to do
the ossification of their relatives who have passed away. After finishing the
rituals and offering meals, Panda would receive some donation. If the host
could not visit in person then the ossified remains would be sent by
post-parcel. Alongside, some money as donation were also sent via Money Order.
Panda would do the rituals on behalf of the host.

Earlier Mahanand Baldev was the Panda. Nowadays, Makhan Chakhan is Maheshwaris
Panda.

\textbf{Badrinath}: There are Pandas for rituals for the visiting pilgrims.
They also have a separate Panda for Dhati people.

Currently, there is Pandit Dinanath Panchpuri. He stays in Badrinath, (district
Garhwal) PIN 246422 between 10th May and 15th November. The temple at Badrinath
remains open only in these days or the year otherwise the temple remains closed
due to winter's cold. Other times the Panda live in Devprayag (district
Garhwal) PIN 249301. 

Such Chobas-Pandas would frequently visit Maheshwaris in Thar towns and get
some donations. The hassling between these Chobas and Pandas at Mathura and
Haridvar station often becomes a nuisance for the visitors.

\textbf{Shri Nathdwara}: Sometimes Shri Nathdwara's Gusain would also visit.
They would drop down at Chhod or Naukot station. They would not travel over
horse or camelback. Since there were no chariots in Thar at the time, people
would carry them over their shoulders to take them from one town to other.

\textbf{Gor}: In Thar's towns where Maheshwaris lived, each town had a Gor
serving a particular ancestry. Gors would do appropriate rituals during
occasions such as birth, wedding and death. Maheshwaris Gor were Pushkarna
Brahmins Vaus (3 ancestries), Chhangani (5 ancestries) and Paliwad - Dhamat (7
ancestries). People would consult with them for auspicious days for organizing
various ceremonies as well as general enquiries about upcoming special days
according to lunar calendar. They would also tell religious stories in a
congregation.

Sometimes, other holymen would visit towns. Specially, sadhus with small pots.
They would stand at the gate of homes and chant ``\textbf{bhar de lota}"
meaning ``fill the pot". People would fill the pot with grains or flour. These
sadhus were often called ``bhugat wala" by the local folks.

\begin{framed}
\begin{center}\textbf{``Definition of the word
Maheshwari"}\end{center}

We all Maheshwaris have born and evolved from the divine powers and special
blessings by lord Shiwa. Every syllable of the Maheshwari word has its own
special meaning. This meaning is still relevant in the modern systems and
situations.

\textbf{Ma}: The first syllable of the word Maheshwari is an indicator for
respect to self and others. We respect others and gain respect in return. In
practice we receive what we give.

\textbf{He}: This syllable indicates to progress despite any situations.
Meaning find convenience in the face of inconvenience and make progress. Get
rid of the food and behavior which is understood to be harmful.

\textbf{Sh}: This syllable is a symbol of peace and coolness. Obtain the Shiwa
element in peace and simplicity. If speech is sweet as sugar and behavior is
clean then all will be assimilated well. Because of this quality, wherever we
went, we became a part of their culture and systems.

\textbf{Wa}: This syllable indicates towards our main karma, that is business.
We have obtained wealth with our hard work and perseverance in all kinds of
adverse conditions. We have served our duties and helped in strengthening the
economy of the country.

\textbf{Ri}: This syllable is the symbol of pure customs and policies. Our
society is an ideal one with a rich culture. Because of a spirit of love,
selflessness, unity and respect, while following one's duty, we have been able
to make and maintain our own distinct identity.

\end{framed}
