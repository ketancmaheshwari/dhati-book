\chapter{Fairs}
We humans seek time out of our routine to activities that will give us pleasure
and help us forget our sorrows and tiredness. In addition to daily such
activities, society has organized annual festivities and fairs that we can
enjoy in our leisurely time. Fairs are organized on religious and social
festivals and reflect society's culture and traditions. One forgets themselves
and gets enjoyment for a while.

Many fairs were organized in Thar and people--men, women and children would
enjoy irrespective of caste-community or creed. Some such fairs are described
below which were particularly organized around the towns where Maheshwaris
lived:

(1) On the \textit{otta} (outside of house) of Hardas Bhagat in the eastern
Mithi a fair is organized over the dune. This fair is organized on
\textbf{Bhadarva, sud} second day and participated by many people from towns
all around. Earlier this was a one-day fair which became a 3-day fair after the
partition.

A Pushkarna Brahmin named Hardas lived in Chhachhro. In the year 1781, one
night, he saw in dream that some kind of catastrophe is going to befall upon
the town of Chhachhro. There will be riots and bloodbath. Next day he told
about his dream to people of the town and mentioned that he is planning to
leave the town. Anyone is free to join him in this exodus. With Hardas Bhagat,
Maheshwari community's leader Mukhi Malji Mansukhani (Rathi)'s ancestor moved
to Mithi with family. Soon the predicted catastrophe happened. Madadkhan Pathan
attacked and badly wrecked Chhachhro.

Hardas Bhagat was a saint. A fifth generation of his brother Asandas, Roopchand
lives in Khithal in Alwar district of Rajasthan. Descendents of Rupchand's
uncle Farasram, Harishkumar and brothers live in Palanpur and Kadi in Gujarat.
As per tradition, the fair is organized in Palanpur and Mithi.

Malji Mansukhani's descendents are living in Mithi, Palanpur, Vadgam, Bhuj and
Ahmedabad.

(2) Six miles south of Chelhar near the way to Mithi a fair is organized in the
town of Harehar in honor of goddess Malhan Mata. The fair is organized on the
7th day of \textit{Magh sud}. As per the anecdote:

Bundi's Hada lineage king got married to a Rajput Malhan woman. Afterwards, a
holyman informed them that their ancestry belongs to the eighth century
Chauhans which is also the origin of the Hada lineage. Thus, both husband and
wife belong to the same ancestry. King thus renounced the queen and decided to
repent the mistake. The lady became a sati and Rajputs worship her as a goddess
till date. Everyone from the nearby village come to attend this fair.

(3) Four miles from Kantyo town on the way to Chhachhro and Umarkot on the
second day of \textit{Bhadarva sud} Chandopeer's fair is organized. A platform
is built their and people have immense faith in the peer. In addition to nearby
towns, Maheshwaris also visit from Chelhar and Chhachhro.

(4) There is a memorial (a place where someone has meditated unto death) of
\textit{dada} Parbrahm near the Verizap town of Diplo district. There a fair is
organized between the 10th and full moon day of \textit{Jeth sud} month. The
fair is organized in the night under moonshine. There is a \textit{khabad} tree
half a kilometer from the memorial. A flame is seen in the tree once or twice
during the fair. Dada Parbrahm meditated against the ongoing tyranny on the
orders of a Saint from Kutch named Mekandada. Mekandada was a saint from the
times of guru Gorakhnath. Dada Parbrahm went on a pilgrimage to the godess
Hinglaaj and brought a trident which he set up near his meditation place.
During the fair, beleivers come from all over the place with token tridents and
erect them in place.

(5) Saint supreme Nenuram Saaheb's Ashram and his resting place is located
about 28 miles from Mithi in the town of Islamkot. A fair is organized their on
the second and third day of \textit{Bhadarva sud}. Nenuram was born in the
household of Meghuram and Meghabai in the year 1898. At the age of 9 he adapted
a path of sainthood. Kuniram was his guru. Nenuram was a yogi, knowledgeable,
celibate and saint. He campaigned for knowledge and devotion. Cold water and
warm food is always available at his Ashram. Thousands of people and animals
were blessed with food and water from the Ashram during a drought. He passed
away on 15th September 1973.

(6) Two Miles East of the Chhachhro town, there is a village called Radli where
Ramgerdada's fair is organized on the 7th day of \textit{Magh sud}. Many people
living around participate in this fair. Camels and horse races are organized. A
lot of sweet is sold in 3-4 hours during the fair.

(7) There is a Shiwa temple near Khorad village near Umarkot. A fair is
organized on Shivratri. People from surrounding villages and from Maheshwaris
from New Chhod come to participate.

(8) A fair is organized around Mukhi Tarachand's Shiwala (a small Shiwa temple)
in Mithi.

(9) A fair is organized around the Saradhra's Shiwa temple in Karunjhar hills
near Nagar Parkar. Many people participate in this fair from far away. The
route to the fair venue is hard to travel. 

(10) Peer Pithoro's fair is organized near the Pithoro railway station on
\textit{Chaitra} new moon day. Peer Pithoro appeared on a horse for the
protection of religion. He disappeared in the place where there is a memorial.
Thousands of pilgrims from different communities participate. Maheshwaris also
participate with great faith. 

(11) \textbf{Goddess Hinglaaj:} The beautiful and splendid place of this
goddess is 200 KM from Karachi, over the Kech Makran mountains and across the
Hingol (Aghor) river. It is located in a cave 300 feet high in the mountains.
Journey is hard without some familiar person. From Karachi, one can go via Hub,
Othal and Bela. Many saints such as Shri Ramchandra, Guru Gorakhnath, Oghadnath
and others have visited with great faith. The fair is organized on 3-4 April.
The place is very ancient.

