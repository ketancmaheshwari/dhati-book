\chapter{Migration from Marwar to Thar}
\chaptermark{Migration}
As mentioned in earlier chapter, the Maheshwari community migrated from
Jaisalmer to different parts of India between A.D. 1300 and A.D. 1700 because
of different causes. Main reasons for these migrations might be Muslim tyranny,
Maratha civil war and continuous drought etc. (Only those who had to leave
their age-old home and country can know the pain and agony they had to
undergo!).

While migrating, where to go was a question. A safe and known place would be a
natural choice. Royal states of Jaisalmer and Amarkot (Umarkot) were connected
through marriage relationships as the following examples shows:

\begin{enumerate}
 \item Jaisalmer's king Chachakdev first married Umarkot's princess of King Roopsingh in the year A.D. 1197.
\item King Lakhansingh married in Umarkot in the year A.D. 1270.
\item King Jaysingh married Jadawkunwar, daughter of Umarkot's Sodha Naharsang Amarsang.
\end{enumerate}

Since the establishment of Jaisalmer in the year V.S.\footnote{Vikram Samvat, a
hindu lunar calendar} 1212 (A.D. 1155) until V.S. 1915 (A.D. 1858) the
financial officers were Maheshwaris. So Maheshwaris were also main among the
service people of the state governance. While in war, handling rations etc.,
staying together during wedding ceremonies and in business and royal
administration, Maheshwaris used to enjoy the positions of \textit{``ghadvai''},
\textit{``choudhary''} and \textit{``mun'hata''}. So Umarkot was familiar to Maheshwaris.

In those days, rows of camels (caravan) with many camels used to transport the
luggage and goods from the Jaisalmer state. Such rows went to Umarkot via
Ratokot towards south (Ratokot was a big city in Thar-Parkar district near the
Marwar border. This city was destroyed later.). Another route went through
rohdi to sakhkhar. So, that was a known route as well. Both these routes were
called \textit{Trade Routes}. Depending upon the situation, familiarity and
opportunity, people used one of these routes for migrations. These times were
approximately between the years A.D. 1736 and A.D. 1755.

In this way, Maheshwaris settled in the state of Sodha's in Umarkot. Those who
travelled on a different route settled in places like Bukera, Tando Alahyaar,
Tando Aadam, Sehwan, Badin etc. There were around 300 Maheshwari families in
Sehwan at that time (there are none today). Maheshwaris living in Sindh used to
speak Thari with an influence of the Sindhi language.

Maheshwaris were strictly vegetarian. Even onion and garlic were considered
uneatable. So they could not live with the non-vegetarian culture of Sindh.
Apart from that, Marwar was a dry region. On the other hand, the Sindhu river
basin was not comfortable because of high humidity and mosquitoes. So they
marched forward to Thar desert in search of alternatives. Though Thar was
similar to Marwar, they liked it because there was no political tension as was
prevalant in Marwar. In such conditions they started looking for their
relatives and family members in those areas including the Kutchtch region. Some
families that came along Sindh and nearby Mithi and Bagal region were called
Sindhi. In 1736 AD, when Mian NoorMohammed attacked Umarkot, Sodhas spread
across Thar. With them Maheshwaris too settled in different villages in Thar.
These families settled in Dahali, Chhod, Bagal, Chhachharo, Nabisar, Chelhar
etc. In 1875 A.D., there was heavy rains in this region and a lot of
domesticated animals died in floods. Because of this several Maheshwaris went
to the high and dried sandy regions of Thar. Thus, people from same ``Akaah''
(extended family) settled in one place and in the time of political stability
started searching for their families and hence the populations of these places
increased.

Along with Maheshwaris, other community people like Pushkarna Brahmin,
Saraswat, Shrimali Brahmin, Maali, Sonara, Sutar, Darji, Kumhar etc also got
settled.

These migrant Maheshwaris were known by the places they migrated. For example
people from Kutchtch were called ``Kutchtchi'', people from Thar were called
``Thari or Dhati'' and people from Jamnagar who came from Nagor (in Marwar)
were called ``Nagori''.

This way, Maheshwaris got settled and started developing their business and
employment. Gradually they built homes and started marriage etc. rituals. In
Thar's various villages 16 out of 72 Maheshwari clans (As described earlier)
settled as follows:\\ Rathi, Kela (sarada), Kadva, Hadkut, Gigal, Chandak,
Bhutada, Baththar, Malhar, Masania (Baheti), Panpaliya, Lohia, Kachoria,
Kasumbia, Malpani, Laghad. Kela's are called ``Ghurya'' in Thar and ``Maandan''
in Kutchtch.

Additional to Thar Villages, some villages in Marwar, such as Sundaro,
Mahajalar, Khuhadi, Jaysindar, and Lilmu (These villages are in Indian
Territory now) were also associated with giving daughters' hand in marriage.
Whereas other villages of Marwar were only related with general friendship.
