\chapter{\textit{Mukhi} and \textit{Dhura}}
After starting to live in Thar, Maheshwaris settled down gradually and started
inviting others from nearby towns and villages from Marwar to stay together. In
this manner, every town/village had its own social system. \textit{panchayat}
and \textit{Mukhi} were appointed in order for a smooth administration of the
system. Normally, people who are leading and of high status would be elected as
mukhis. Mukhis would listen to the complaints of community people and try to
resolve them. Mukhis would also oversee that the customs made by the system are
followed appropriately.

Mukhis from different towns/villages would get together. In order to make sure
that all Maheshwaris have a similar system, they would make sure that the
customs are uniform and would make amendments if there are differences. They
would elect one person among themselves as a president of the community. He
would preside over issues related to the whole society and after due
deliberation and hearing out all sides, he would offer his assistance in
finding out resolutions of such issues. Such mukhis were highly respected in
all of the society and it was seen that no one speaks against them. 

As per information, at different times or simultaneously, 2-3 mukhis would live
in a town. Their names and towns are as shown below (With insufficient
information, please excuse if there are any omissions):

\textbf{Mithi}: Maljiram Mansukhdas Rathi / Somjimal Maljiram / Lachhmandas Somjimal.\\
                Maljiram Radhakirshan / Meghraj Radhakirshan / Lachhmandas Meghraj Karmani
                Radhakirshan Amarchand Kullar

\textbf{Umarkot}: Khushaldas Khataumal Karmani

\textbf{Chhachhro}: Mathradas Motiram Maherani
                    Jagroopdas Narsingra
                    Maljiram Jagrupdas
                    Nanjomal Maljiram
                    Motiram Khushaldas Mathrani
                    Somjimal Amarchand Munhta


\textbf{Chelhar}: Damodar Vastani / Gokaldas / Ranomal
                  Meghraj (Eldest of the Sajnani Family)

\textbf{Kantyo}: Khataumal Langhni (Kachoria) and their family

\textbf{New Chhod}: Jivraj Maankaani
                    Kanji Amarchand Saanjhira
                    Punamchand Kewalram Karmani (Mukhi of Dhat).

During one wedding related conflict, in order to bring about a resolution a
Maheshwari community panchayat met in a town called Hingora, 6 miles east of
Umarkot. Their the guilty party was resolved to be thrown out of the community.
Here Umarkot's Mukhi Khushaldas Karmani and other leaders signed the
resolution.

Mithi's Mukhi Maljiram Mansukhdas Rathi could not reach in time their for some reason. He was also considered a very influential person. First space was not reserved for him to sign. As soon as he reached, he was given the resolution document to sign. He saw it and asked: \textbf{"Sahi kith karan?"} (Where should I sign?). Somebody said sign somewhere in the side. On listening this, he said \textbf{"He sahi paseri j rehise"} (refused to sign) and left.

Since then, two polar groups of Mukhis were formed in Dhat. One was called \textbf{Karmania-ro-dhuro} and the other \textbf{Malania-ro-dhuro}. All towns and villages of Dhat were similarly divided in two groups of Mukhis. In a similar manner, because of some or the other conflicts, in Mithi, Chhachhro etc. places, many more \textit{Dhuras} (literally polar groups) were formed. The details of these dhuras is as follows:

\textbf{Mithi}: (1) Mukhiyaan-ro-dhuro (2) Karmaniyaan-ro-dhuro (3) kularaan-ro-dhuro (4) laluraan-ro-dhuro.

\textbf{Chhachhro}: (1) Narsingraan-ro-dhuro (2) Mathraniyaan-ro-dhuro (3) Maheraaniyaan-ro-dhuro (4) Munhta-ro-dhuro.

\textbf{Chelhar}: (1) vado (big) dhuro included Vastani, Girdharani, Damani, Lalura, and Kara. (2) Nandho (small) dhuro included Damoshah Devani and Narsing Mehrani. From the small dhuro, 4-5 families separated and formed their own dhuro.

\textbf{New Chhod}: (1) Saanjhiraan-ro-dhuro (2) Maankaaniyaan-ro-dhuro

\textbf{Gadhado}: (1) Tikmanian-ro-dhuro included Chandani, Gopani, Kachromal Mukhi, and Narumal Tulsidas (2) Mulaaniyaan-ro-dhuro.

There were no dhuras in Kaantyo and Umarkot.

During the occasions such as wedding and passing away, invitations were given in the same dhura only. If invitees in wedding have relatives in other dhuras, they were also invited. However, there were no cases of violent conflict or quarrels because of dhuras.
