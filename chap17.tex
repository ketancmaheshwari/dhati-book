\chapter{Seasons and Production}
Thar had three seasons--Winter, Summer and Monsoon.

Because Thar is a sandy and arid land, the sand heats and cools down fast.
Dunes would be cold in winters. December-January were very cold (\textbf{si
pade}). Since there was no sea nearby, no sea effect was seen in Thar's climate.
People would wrap \textbf{loi}, \textbf{khatha} and \textbf{kambal} to protect
themselves from cold. A brazier would be lit in home every morning and evening
where family would gather to get some heat. Improptu bonfires would be held in
villages. People would work less in winters. If it is too much of cold people
would say ``\textbf{ruth pe to}".

Season would change after the Holi festival. Summers would start. Summers are
very strong in Thar. Earth heats up a lot. Body would sweat a lot. In the day
time, sun shines hot and hot winds blow with small grains of sand. People would
get sick of these winds which are called \textbf{loo}. However, nights were
relatively cool. In summer, all the greenery would dry away from the dunes and
in the valleys and the area would look very desolate. Lakes and stepwells would
dry up. Waters would go deeper in wells.

As the summer progresses, the heat would increase. May-June were very difficult
to pass. Clouds would form and people would eagerly wait for rains. It was
believed that rain will fall when clouds move from North-South. Everyone would
look towards the skies. Initial raindrops would be absorbed by the sandy land
which was called \textbf{rej}. If rain has fallen somewhere from where the cool
breeze is blowing, it was known as \textbf{vuthe-ro-va}. Wet sand fragrance
would spread around.

Thari people were wholly dependent on rains. There was a saying as:
``\textbf{Vutho to Thar, na to bur}" meaning if it rains, Thar is worth living
otherwise it is a desert. Rains would bring nature's blessings. Dunes and land
would have a green cover. Lakes, small and big would overflow with fresh
waters. People and cattle would get a new life. Leaves would get fresh and
green.

Thar receives relatively less Monsoon rains. On average, it would be 10 to 12
inches of rain per season. Monsoon would last between 15th June and 15th
September. Every third or fourth year would be dreaded as a drought year.
People would get happy even on a small amount of rain. Experienced people would
look into calendars and other circumstances and based on ancient knowhow, would
tell when will the rain arrive.

If the lightning happened from the North-East corner, rain would certainly
happen, they said. It was a saying: ``\textbf{khivan khivi ishani, nodi ghare
visani}" (khivan means lightning and ishani is the name of direction
North-East). If there is too much of lightning, it was called \textbf{bakrar}.
Rains without thunder and lightning was called \textbf{gungo meh} (gungo means
mute). If it rains with sun out, it was called \textbf{ughado meh}. At this
time colorful rainbow (\textbf{dangalo}) was seen in the sky. If there are rain
clouds for too many days without sunlight, then it was said:

\textbf{tod pinchha, kar tara; si ta mari, garib vichara}

Meaning, ``ohh god please remove the clouds and show stars, because poor people
are dying of cold".

If it rains heavily, it was called \textbf{soker pai}. Flooding was feared with
too much rains. Such rains happened in the years 1913, 1927, and 1929. During
the heavy rains of the year 1927, people went to live on dunes or tents because
their homes were damaged by rain. It is remembered as \textbf{chorasiye-ro-meh}
(chorasi means 84) because the year in local calendar was 1984. Too much rain
fell in the year 1874 in Sindh because of which there was flooding in the Nara
river. Maheshwaris who lived in Nabisar and Umarkot were affected by these
floods and moved to places with higher elevation such as Mithi, Chhachhro,
and Kantyo. If there is incessant rains and not stop for days, people would
fill small ghee spoons with rain water in an attempt/ritual to pray for rain to
stop. When rains stop, it was called \textbf{meh okajyo}. 

There were often droughts in Thar. One such drought was in the local calendar
year 1942 (AD. 1885). Another big drought happened in the year 1956 (AD 1899).
This drought is known as \textbf{chhappaniya dukal}. It is heard that people
would sell off their children for a handful of grains. One drought of 1996 (AD
1939) is known as \textbf{Chhannuana dukal}. In such small and big draughts
lakes would dry up and waters deepen in the wells. Cattle would die without
grass and water. People would go to Sindh for labor work. Poor people would use
leaves and dried, crushed bark of tree because of a shortage of foodgrains. 

In the monsoon season, Thar's fields would bear pearl millet, sesame, moong,
sorghum, beans, castor etc. This produce would grow in abundant and businessmen
would store this in roomful of their homes. 

Other than these, different kinds of melons and gourds would grow aplenty.
Additionally, where well water is available, other vegetables such as spinach,
radish, eggplant, carrot, etc were also grown in small quantities. (More
details are available in the `food' section.) There were no parks or gardens in
Thar. There were some gardens in Umarkot. There, fruits like Mango,
Pomegrenate, Lemon, Banana, berries, grapes were grown. They had the benefit of
plain land and water from Sindhu river.

Many Maheshwaris owned fields in the nearby villages which were given to
\textit{Bheels-Kolis} for cultivation. At the time of harvest, sufficient
grains and produce were kept for themselves and the rest was sold off. 

Other seasonal fruits in Thar were berries, \textbf{pipa}, \textbf{kundhir},
\textbf{tagra}, \textbf{Ghangheta}, \textbf{pakka}, \textbf{Khokha},
\textbf{fogesi}, \textbf{ratabor}, and \textbf{sangri}. These kinds of fruits
were brought by children who were out playing or wandering people who would
exchange them for some foodgrains or buttermilk.

Some fruits such as Apples, Banana, fresh Dates, dried berries, sugar-cane etc
were brought from Sindh. Green and ripe Mangoes were also available. In dried
fruits, Cashew, Almonds, Resins, Pistachios and \textbf{neja} nuts were
available. Only some upper class people would be able to afford such dried
fruits. Among fresh vegetables from Sindh, sometimes cabbage, cauliflower and
okra would also be brought. Average person would rarely consume them.

Back in the day rice-paddy was brought from Sindh and it was crushed in a
special Mill to produce rice. This rice was red in color. This was crushed in a
mill and the powder or paste was given to cattles as fodder. White rice was
used in \textit{khichdi} and red rice, being sweet was used in making
\textit{roti}. Wheat was not produced in Thar and was brought from Sindh. 

Sugar was imported from Java-Sumatra. The sugar was brownish in color. This
kind of sugar was called \textbf{daboto} or \textbf{bachcha} sugar. During the
world war II around the years 1939-40 the sugar stopped coming and the
government decided to distribute Indian sugar made in the mills on rationing
basis. Jaggery would come from Sindh. Jaggery was used more than sugar.

